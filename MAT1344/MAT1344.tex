
\documentclass[x11names,reqno,14pt]{extarticle}
% Choomno Moos
% Portland State University
% Choom@pdx.edu


%% stupid experiment %%
%%%%%%%%%%%%% PACKAGES %%%%%%%%%%%%%

%%%% SYMBOLS AND MATH %%%%
\let\oldvec\vec
\usepackage{authblk}	% author block customization
\usepackage{microtype}	% makes stuff look real nice
\usepackage{amssymb} 	% math symbols
\usepackage{siunitx} 	% for SI units, and the degree symbol
\usepackage{mathrsfs}	% provides script fonts like mathscr
\usepackage{mathtools}	% extension to amsmath, also loads amsmath
\usepackage{esint}		% extended set of integrals
\mathtoolsset{showonlyrefs} % equation numbers only shown when referenced
\usepackage{amsthm}		% theorem environments
\usepackage{relsize}	%font size commands
\usepackage{bm}			% provides bold math
\usepackage{bbm}		% for blackboard bold 1

%%%% FIGURES %%%%
\usepackage{graphicx} % for including pictures
\usepackage{float} % allows [H] option on figures, so that they appear where they are typed in code
\usepackage{caption}
\usepackage{hyperref}
%\usepackage{titling}
\usepackage{tikz} % for drawing
\usetikzlibrary{shapes,arrows,chains,positioning,cd,decorations.pathreplacing,decorations.markings,hobby,knots,braids}
\usepackage{subcaption}	% subfigure environment in figures

%%%% MISC %%%%
\usepackage{enumitem} % for lists and itemizations
\setlist[enumerate]{leftmargin=*,label=\bf \arabic*.}

\usepackage{multicol}
\usepackage{multirow}
\usepackage{url}
\usepackage[symbol]{footmisc}
\renewcommand{\thefootnote}{\fnsymbol{footnote}}
\usepackage{lastpage} % provides the total number of pages for the "X of LastPage" page numbering
\usepackage{fancyhdr}
\usepackage{manfnt}
\usepackage{nicefrac}
%\usepackage{fontspec}
%\usepackage{polyglossia}
%\setmainlanguage{english}
%\setotherlanguages{khmer}
%\newfontfamily\khmerfont[Script=Khmer]{Khmer Busra}

%%% Khmer script commands for math %%%
%\newcommand{\ka}{\text{\textkhmer{ក}}}
%\newcommand{\ko}{\text{\textkhmer{ត}}}
%\newcommand{\kha}{\text{\textkhmer{ខ}}}

%\usepackage[
%backend=biber,
% numeric
%style=numeric,
% APA
%bibstyle=apa,
%citestyle=authoryear,
%]{biblatex}

\usepackage[explicit]{titlesec}
%%%%%%%% SOME CODE FOR REDECLARING %%%%%%%%%%

\makeatletter
\newcommand\RedeclareMathOperator{%
	\@ifstar{\def\rmo@s{m}\rmo@redeclare}{\def\rmo@s{o}\rmo@redeclare}%
}
% this is taken from \renew@command
\newcommand\rmo@redeclare[2]{%
	\begingroup \escapechar\m@ne\xdef\@gtempa{{\string#1}}\endgroup
	\expandafter\@ifundefined\@gtempa
	{\@latex@error{\noexpand#1undefined}\@ehc}%
	\relax
	\expandafter\rmo@declmathop\rmo@s{#1}{#2}}
% This is just \@declmathop without \@ifdefinable
\newcommand\rmo@declmathop[3]{%
	\DeclareRobustCommand{#2}{\qopname\newmcodes@#1{#3}}%
}
\@onlypreamble\RedeclareMathOperator
\makeatother

\makeatletter
\newcommand*{\relrelbarsep}{.386ex}
\newcommand*{\relrelbar}{%
	\mathrel{%
		\mathpalette\@relrelbar\relrelbarsep
	}%
}
\newcommand*{\@relrelbar}[2]{%
	\raise#2\hbox to 0pt{$\m@th#1\relbar$\hss}%
	\lower#2\hbox{$\m@th#1\relbar$}%
}
\providecommand*{\rightrightarrowsfill@}{%
	\arrowfill@\relrelbar\relrelbar\rightrightarrows
}
\providecommand*{\leftleftarrowsfill@}{%
	\arrowfill@\leftleftarrows\relrelbar\relrelbar
}
\providecommand*{\xrightrightarrows}[2][]{%
	\ext@arrow 0359\rightrightarrowsfill@{#1}{#2}%
}
\providecommand*{\xleftleftarrows}[2][]{%
	\ext@arrow 3095\leftleftarrowsfill@{#1}{#2}%
}
\makeatother

%%%%%%%% NEW COMMANDS %%%%%%%%%%

% settings
\newcommand{\N}{\mathbb{N}}                     	% Natural numbers
\newcommand{\Z}{\mathbb{Z}}                     	% Integers
\newcommand{\Q}{\mathbb{Q}}                     	% Rationals
\newcommand{\R}{\mathbb{R}}                     	% Reals
\newcommand{\C}{\mathbb{C}}                     	% Complex numbers
\newcommand{\K}{\mathbb{K}}							% Scalars
\newcommand{\F}{\mathbb{F}}                     	% Arbitrary Field
\newcommand{\E}{\mathbb{E}}                     	% Euclidean topological space
\renewcommand{\H}{{\mathbb{H}}}                   	% Quaternions / Half space
\newcommand{\RP}{{\mathbb{RP}}}                       % Real projective space
\newcommand{\CP}{{\mathbb{CP}}}                       % Complex projective space
\newcommand{\Mat}{{\mathrm{Mat}}}						% Matrix ring
\newcommand{\M}{\mathcal{M}}
\newcommand{\GL}{{\mathrm{GL}}}
\newcommand{\SL}{{\mathrm{SL}}}

\newcommand{\tgl}{\mathfrak{gl}}
\newcommand{\tsl}{\mathfrak{sl}}                  % Lie algebras; i.e., tangent space of SO/SL/SU
\newcommand{\tso}{\mathfrak{so}}
\newcommand{\tsu}{\mathfrak{sl}}


% typography
\newcommand{\noi}{\noindent}						% Removes indent
\newcommand{\tbf}[1]{\textbf{#1}}					% Boldface
\newcommand{\mc}[1]{\mathcal{#1}}               	% Calligraphic
\newcommand{\ms}[1]{\mathscr{#1}}               	% Script
\newcommand{\mbb}[1]{\mathbb{#1}}               	% Blackboard bold


% (in)equalities
\newcommand{\eqdef}{\overset{\mathrm{def}}{=}}		% Definition equals
\newcommand{\sub}{\subseteq}						% Changes default symbol from proper to improper
\newcommand{\psub}{\subset}						% Preferred proper subset symbol

% Categories
\newcommand{\catname}[1]{{\text{\sffamily {#1}}}}

\newcommand{\Cat}{{\catname{C}}}
\newcommand{\cat}[1]{{\catname{\ifblank{#1}{C}{#1}}}}
\newcommand{\CAT}{{\catname{Cat}}}
\newcommand{\Set}{{\catname{Set}}}

\newcommand{\Top}{{\catname{Top}}}
\newcommand{\Met}{{\catname{Met}}}
\newcommand{\PL}{{\catname{PL}}}
\newcommand{\Man}{{\catname{Man}}}
\newcommand{\Diff}{{\catname{Diff}}}

\newcommand{\Grp}{{\catname{Grp}}}
\newcommand{\Grpd}{{\catname{Grpd}}}
\newcommand{\Ab}{{\catname{Ab}}}
\newcommand{\Ring}{{\catname{Ring}}}
\newcommand{\CRing}{{\catname{CRing}}}
\newcommand{\Mod}{{\mhyphen\catname{Mod}}}
\newcommand{\Alg}{{\mhyphen\catname{Alg}}}
\newcommand{\Field}{{\catname{Field}}}
\newcommand{\Vect}{{\catname{Vect}}}
\newcommand{\Hilb}{{\catname{Hilb}}}
\newcommand{\Ch}{{\catname{Ch}}}

\newcommand{\Hom}{{\mathrm{Hom}}}
\newcommand{\End}{{\mathrm{End}}}
\newcommand{\Aut}{{\mathrm{Aut}}}
\newcommand{\Obj}{{\mathrm{Obj}}}
\newcommand{\op}{{\mathrm{op}}}

% Norms, inner products
\delimitershortfall=-1sp
\newcommand{\widecdot}{\, \cdot \,}
\newcommand\emptyarg{{}\cdot{}}
\DeclarePairedDelimiterX{\norm}[1]{\Vert}{\Vert}{\ifblank{#1}{\emptyarg}{#1}}
\DeclarePairedDelimiterX{\abs}[1]\vert\vert{\ifblank{#1}{\emptyarg}{#1}}
\DeclarePairedDelimiterX\inn[1]\langle\rangle{\ifblank{#1}{\emptyarg,\emptyarg}{#1}}
\DeclarePairedDelimiterX\cur[1]\{\}{\ifblank{#1}{\emptyarg,\emptyarg}{#1}}
\DeclarePairedDelimiterX\pa[1](){\ifblank{#1}{\emptyarg}{#1}}
\DeclarePairedDelimiterX\brak[1][]{\ifblank{#1}{\emptyarg}{#1}}
\DeclarePairedDelimiterX{\an}[1]\langle\rangle{\ifblank{#1}{\emptyarg}{#1}}
\DeclarePairedDelimiterX{\bra}[1]\langle\vert{\ifblank{#1}{\emptyarg}{#1}}
\DeclarePairedDelimiterX{\ket}[1]\vert\rangle{\ifblank{#1}{\emptyarg}{#1}}

% mathmode text operators
\RedeclareMathOperator{\Re}{\operatorname{Re}}		% Real part
\RedeclareMathOperator{\Im}{\operatorname{Im}}		% Imaginary part
\DeclareMathOperator{\Stab}{\mathrm{Stab}}
\DeclareMathOperator{\Orb}{\mathrm{Orb}}
\DeclareMathOperator{\Id}{\mathrm{Id}}
\DeclareMathOperator{\vspan}{\mathrm{span}}			% Vector span
\DeclareMathOperator{\tr}{\mathrm{tr}}
\DeclareMathOperator{\adj}{\mathrm{adj}}
\DeclareMathOperator{\diag}{\mathrm{diag}}
\DeclareMathOperator{\eq}{\mathrm{eq}}
\DeclareMathOperator{\coeq}{\mathrm{coeq}}
\DeclareMathOperator{\coker}{\mathrm{coker}}
\DeclareMathOperator{\dom}{\mathrm{dom}}
\DeclareMathOperator{\cod}{\mathrm{codom}}
\DeclareMathOperator{\im}{\mathrm{im}}
\DeclareMathOperator{\Dim}{\mathrm{dim}}
\DeclareMathOperator{\codim}{\mathrm{codim}}
\DeclareMathOperator{\Sym}{\mathrm{Sym}}
\DeclareMathOperator{\lcm}{\mathrm{lcm}}
\DeclareMathOperator{\Inn}{\mathrm{Inn}}
\DeclareMathOperator{\sgn}{sgn}						% sgn operator
\DeclareMathOperator{\intr}{\text{int}}             % Interior
\DeclareMathOperator{\co}{\mathrm{co}}				% dual/convex Hull
\DeclareMathOperator{\Ann}{\mathrm{Ann}}
\DeclareMathOperator{\Tor}{\mathrm{Tor}}


% misc symbols
\newcommand{\divides}{\big\lvert}
\newcommand{\grad}{\nabla}
\newcommand{\veps}{\varepsilon}						% Preferred epsilon
\newcommand{\vphi}{\varphi}
\newcommand{\del}{\partial}							% Differential/Boundary
\renewcommand{\emptyset}{\text{\O}}					% Traditional emptyset symbol
\newcommand{\tril}{\triangleleft}					% Quandle operation
\newcommand{\nabt}{\widetilde{\nabla}}				% Contravariant derivative
\newcommand{\later}{$\textcolor{red}{\blacksquare}$}% Laziness indicator

% misc
\mathchardef\mhyphen="2D							% mathomode hyphen
\renewcommand{\mod}[1]{\ (\mathrm{mod}\ #1)}
\renewcommand{\bar}[1]{\overline{#1}}				% Closure/conjugate
\renewcommand\qedsymbol{$\blacksquare$} 			% Changes default qed in proof environment
%%%%% raised chi
\DeclareRobustCommand{\rchi}{{\mathpalette\irchi\relax}}
\newcommand{\irchi}[2]{\raisebox{\depth}{$#1\chi$}}
\newcommand\concat{+\kern-1.3ex+\kern0.8ex}

% Arrows
\newcommand{\weak}{\rightharpoonup}					% Weak convergence
\newcommand{\weakstar}{\overset{*}{\rightharpoonup}}% Weak-star convergence
\newcommand{\inclusion}{\hookrightarrow}			% Inclusion/injective map
\renewcommand{\natural}{\twoheadrightarrow}				% Natural map

% Environments
\theoremstyle{plain}
\newtheorem{thm}{Theorem}[section]
%\newtheorem{lem}[thm]{Lemma}
\newtheorem{lem}{Lemma}
\newtheorem*{lems}{Lemma}
\newtheorem{cor}[thm]{Corollary}
\newtheorem{prop}{Proposition}
\newtheorem*{claim}{Claim}
\newtheorem*{cors}{Corollary}
\newtheorem*{props}{Proposition}
\newtheorem*{conj}{Conjecture}

\theoremstyle{definition}
\newtheorem{defn}{Definition}[section]
\newtheorem*{defns}{Definition}
\newtheorem{exm}{Example}[section]
\newtheorem{exer}{Exercise}[section]

\theoremstyle{remark}
\newtheorem*{rem}{Remark}

\newtheorem*{solnx}{Solution}
\newenvironment{soln}
    {\pushQED{\qed}\renewcommand{\qedsymbol}{$\Diamond$}\solnx}
    {\popQED\endsolnx}%

% Macros
\newcommand{\restr}[1]{_{\mkern 1mu \vrule height 2ex\mkern2mu #1}}
\newcommand{\Upushout}[5]{
    \begin{tikzcd}[ampersand replacement = \&]
    \&#2\ar[rd,"\iota_{#2}"]\ar[rrd,bend left,"f"]\&\&\\
    #1\ar[ur,"#4"]\ar[dr,"#5"]\&\&#2\oplus_{#1} #3\ar[r,dashed,"\vphi"]\&Z\\
    \&#3\ar[ur,"\iota_{#3}"']\ar[rru,bend right,"g"']\&\&
    \end{tikzcd}
}
\newcommand{\exactshort}[5]{
		\begin{tikzcd}[ampersand replacement = \&]
			0\ar[r]\&#1\ar[r,"#2"]\& #3 \ar[r,"#4"]\& #5 \ar[r]\&0
		\end{tikzcd}
}
\newcommand{\product}[6]{
		\begin{tikzcd}[ampersand replacement = \&]
			#1 \& #2 \ar[l,"#4"'] \\
			#3 \ar[u,"#5"] \ar[ur,"#6"']
		\end{tikzcd}
}
\newcommand{\coproduct}[6]{
		\begin{tikzcd}[ampersand replacement = \&]
			#1 \ar[r,"#4"] \ar[d,"#5"'] \& #2 \ar[dl,"#6"] \\
			#3
		\end{tikzcd}
}
%%%%%%%%%%%% PAGE FORMATTING %%%%%%%%%

\usepackage{geometry}
    \geometry{
		left=15mm,
		right=15mm,
		top=15mm,
		bottom=15mm	
		}

\usepackage{color} % to do: change to xcolor
\usepackage{listings}
\lstset{
    basicstyle=\ttfamily,columns=fullflexible,keepspaces=true
}
\usepackage{setspace}
\usepackage{setspace}
\usepackage{mdframed}
\usepackage{booktabs}
\usepackage[document]{ragged2e}
\usepackage{epsfig}
\usepackage{dynkin-diagrams}

\pagestyle{fancy}{
	\fancyhead[L]{Fall 2024}
	\fancyhead[C]{MAT1344F}
	\fancyhead[R]{John White}
  
  \fancyfoot[R]{\footnotesize Page \thepage \ of \pageref{LastPage}}
	\fancyfoot[C]{}
	}
\fancypagestyle{firststyle}{
     \fancyhead[L]{}
     \fancyhead[R]{}
     \fancyhead[C]{}
     \renewcommand{\headrulewidth}{0pt}
	\fancyfoot[R]{\footnotesize Page \thepage \ of \pageref{LastPage}}
}
\newcommand{\pmat}[4]{\begin{pmatrix} #1 & #2 \\ #3 & #4 \end{pmatrix}}
\newcommand{\A}{\mathbb{A}}
\newcommand{\B}{\mathbb{B}}
\newcommand{\fin}{``\in"}
\newcommand{\mk}[1]{\mathfrak{#1}}
\newcommand{\g}{\mk{g}}
\newcommand{\h}{\mk{h}}
\newcommand{\J}{\mc{J}}
\newcommand{\tphi}{\tilde{\phi}}
\DeclareMathOperator{\Perm}{Perm}
\DeclareMathOperator{\pdim}{pdim}
\DeclareMathOperator{\gldim}{gldim}
\DeclareMathOperator{\lgldim}{lgldim}
\DeclareMathOperator{\rgldim}{rgldim}
\DeclareMathOperator{\idim}{idim}
\DeclareMathOperator{\SU}{SU}
\DeclareMathOperator{\SO}{SO}
\DeclareMathOperator{\Ad}{Ad}
\DeclareMathOperator{\ad}{ad}
\DeclareMathOperator{\gr}{gr}
\DeclareMathOperator{\Sig}{Sig}
\newcommand{\Rmod}{R-\text{mod}}
\newcommand{\RMod}{R-\text{Mod}}
\newcommand{\onto}{\twoheadrightarrow}
\newcommand{\into}{\hookrightarrow}
\newcommand{\barf}{\bar{f}}
\newcommand{\dd}[2]{\frac{d#1}{d#2}}
\newcommand{\pp}[2]{\frac{\partial #1}{\partial #2}}
\newcommand{\gl}{\mk{g}\mk{l}}
\newcommand{\spew}{\Sp(E,\omega)}
\newcommand{\jew}{\mc{J}(E,\omega)}
\renewcommand{\P}{\mathbb{P}}
\renewcommand{\E}{\mathbb{E}}
\DeclareMathOperator{\Ext}{Ext}
\DeclareMathOperator{\Rank}{Rank}
\DeclareMathOperator{\Sp}{Sp}
\DeclareMathOperator{\ann}{ann}
\DeclareMathOperator{\Lag}{Lag}
\DeclareMathOperator{\Riem}{Riem}
\DeclareMathOperator{\Span}{span}
\newcommand{\exactlon}[5]{
		\begin{tikzcd}
			0\ar[r]&#1\ar[r,"#2"]& #3 \ar[r,"#4"]& #5 \ar[r]&0
		\end{tikzcd}
}

\title{MAT 1344}
\author{John White}
\date{Fall 2024}


\begin{document}

\section*{Lecture 1 - 3/5/24}

A good starting point is Newton's equation $V(q_1, \dots, q_n)$ for a particle: 
\[
m\ddot{q_i} = -\pp{V}{q_i}
\]

The first observation is that if energy is 
\[
E = \frac{m}{2}\dot{q}^2 + V(q)
\]
then $E$ is constant along solution curves (take the $t$ derivative). 

A classic physics trick is to reduce $n$th order to first order by letting higher derivatives be introduced as new variables. Introduce $p_i = m\dot{q}_i$. We have the equations
\begin{align*}
\dot{q_i} & = \frac{1}{m}p_i  & \dot{p_i} = -\pp{V}{q_i}
\end{align*}
The energy becomes ``Hamiltonian." 
\[
H(q, p) = \frac{1}{2m}\sum_{i=1}^n p_i^2 + V(q)
\]

We can write these equations from earlier quite nicely in terms of the Hamiltonian (if you know the potential you know the Hamiltonian, and vice-versa) as 
\begin{align*}
\dot{q_i} & = \pp{H}{p_i},  & \dot{p_i} = -\pp{H}{q_i}
\end{align*}

Hamilton's equation. This looks similar to $\dot{X_i} = -\pp{V}{x_i}$, the equation of a gradient flow. 

One advantage of these Hamiltonian equations is that we have \underline{\underline{lots}} of symmetry, i.e. for coordinate changes 
\begin{align*}
\tilde{q_i} = f_i(q, p), \,\,\,& \tilde{p_i} = g_i(q, p); & \tilde{H}(\tilde{q},\tilde{p}) = H(q, p)
\end{align*}
then in new coordinates, $\dot{\tilde{q_i}} = \pp{\tilde{H}}{\tilde{p_i}}, \dot{\tilde{p_i}} = -\pp{\tilde{H}}{\tilde{q_i}}$

\exm

$\tilde{p_i} = -q_i, \tilde{q_i} = p_i$

\exm

$\tilde{q_i} = q_i, \tilde{p_i} = p_i + \varphi_i(q_1, \dots, q_n)$

We think of this Hamiltonian as having a very large infintie dimensional symmetry group, in contrast to the earlier graident flow, which has a very small symmetry group. 

A \underline{Hamiltonian vector field} 
\[
X_H = \sum_{i=1}^n \left(\pp{H}{q_i}\pp{}{p_i} - \pp{H}{p_i}\pp{}{q_i}\right)
\]

If we take the exterior derivative, 
\[
dH = \sum_{i=1}^n\left(\pp{H}{q_i}dq_i + \pp{H}{p_i}dp_i\right)
\]
We can write Hamilton's equation as $\iota(X_H)\omega = - dH$, with $\omega = \sum_{i=1}^ndq_i\wedge dp_i$ where $\iota$ means contraction. 

Often we take this equation as the definition of the Hamiltonian vector field, which defines a differential equation, which defines a flow, et cetera. 

\underline{Remark:} The word ``Symplectic" was introduced by Hermann Weyl in the theory of Lie groups. 

Symplectic manifolds were introduced by Charles Ehrsmann and Paulette Libermann around 1948. 

In the 60s and 70s, Souriau, Kostant (SP?), others did more work such as trying to phrase classical mechanics in this language. Many others, such as Arnold, Thurston, who showed that there symplectic and complex manifolds are not the same. Arnold initiated a program of symplectic topology in 74(?). Weinstein, Steinberg, Guillemain...

\section*{\underline{Part 1: Symplectic Linear Algebra}}

\defn

A \underline{symplectic structure} on a finite dimensional (real for now) vector space $E$ is a bilinear form
\[
\omega:E\times E \to \R
\]
which is
\begin{enumerate}[label=(\roman*)]

\item Skew-symmetric, meaning $\omega(v, w) = \omega(w, v)$

\item Nondegenerate, meaning $\ker \omega \eqdef \{v \in E \mid \omega(v, w) = 0$ for all $w \in E\}$ is trivial. (Every vector has a friend).

\end{enumerate}

In terms of $\omega^\flat:E\to E^*$, $v \mapsto \omega(v, \cdot)$. 

Skew-symmetry means $(\omega^\flat)* = -\omega^\flat$. 

Non-degeneracy means $\ker(\omega^\flat) = 0$

\exm
\,

\begin{enumerate}

\item ``Standard symplectic structure" For $E = \R^{2n}$ with basis $e_1, \dots, e_n, f_1, \dots, f_n$, if we set 
\begin{align*}
\omega(e_i,e_j) = 0,\omega(f_i,f_j)=0,\,&\omega(e_i,f_j)=\delta_{ij}
\end{align*}

\item For $V$ any finite dimensional vector space, setting $E = V \oplus V^*$, and 
\[
\omega((v_i,\alpha_i),(v_2,\alpha_2)) = \langle\alpha_1,v_2\rangle - \langle \alpha_2,v_1\rangle
\]

\item If $V$ is any finite-dimensional \underline{complex} inner product space $h:V\times V \to \C$. 

If we take $E = V$, $\omega(v, w) = \Im(h(v, w))$ is symplectic.

\end{enumerate}

Note that these three are all actually the same example. 

\defn

A \underline{symplectomorphism} between symplectic vector spaces $(E_i, \omega_i)$, $(i = 1, 2)$ is a linear isomorphism $A: E_1\to E_2$, such that
\[
\omega_2(Av,Aw) = \omega_1(v,w)
\]
for all $v, w \in E_1$ (I.e $\omega_1 = A^*\omega_2$).

\underline{Remark:} stipulating that it is an isomorphism is a little overkill, because anything satisfying the second condition is injective. 

Symplectomorphisms of $(E,\omega)$ to itself are denoted $\Sp(E,\omega)$, and is called the 

\underline{symplectic group}.

In this sense, it is easy to see that those three examples are all symplectomorphic. 


\section*{Lecture 2 - 9/10/24}

\subsection*{\underline{Subspace of symplectic vector space}}

Let $F \subseteq E$. Define $F^\omega = \{v\in E \mid \omega(v, w) = 0$ for all $w \in F\}$, the ``$w$-orthogonal" space. 

In terms of $\ann(F) = \{\alpha\in E^* \mid \alpha(v) = 0$ for all $v \in F\}$

Note that $\omega^\flat:F^\omega \to \ann(F)$ is an isomorphism.

\prop
\,
\begin{itemize}

\item $\dim F^\omega = \dim E - \dim F$

\item $(F^\omega)^\omega = F$, $(F_1 \cap F_2)^\omega = F_1^\omega + F_2^\omega$, $(F_1 + F_2)^\omega = (F_1)^\omega \cap (F_2)^\omega$

\end{itemize}

\proof
\,
\begin{itemize}

\item $\dim F^\omega = \dim(\ann(F)) = \cdots$

\item Since elements of $F$ are orthogonal to elements of $F^\omega$, we have $F\subseteq (F^\omega)^\omega;$ by dimension count have equality. Etc

\end{itemize}

\defn

A subspace $F \subseteq E$ is called
\begin{itemize}

\item \underline{isotropic} if $F \subseteq F^\omega$

\item \underline{coisotropic} if $F^\omega \subseteq F$

\item \underline{Lagrangian} if $F^\omega = F$. 

\end{itemize}

Note $F$ is isotropic if and only if $\omega|_{F\times F} = 0$

Note: 

If $F$ is isotropic, then $\dim F \leq \frac12\dim E$

If $F$ is coisotropic, then $\dim F \geq \frac12\dim E$

In both cases, if equality holds, $F$ is Lagrangian. 

So if $F$ is Lagrangian, then $\dim F = \frac12 \dim E$.

\defn The set of Lagrangian subspaces is denoted $\Lag(E,\omega)$, called the ``\underline{Lagrangian Grassmannian}". 

\prop
\,
\begin{enumerate}[label=(\alph*)]

\item $\Lag(E, \omega) \neq \varnothing$

\item For every $M \in \Lag(E, \omega)$, there exists $L \in \Lag(E,\omega)$ with $L \cap M = \{0\}$ (i.e. $E = L \oplus M$).

\end{enumerate}

\proof
\,
\begin{enumerate}[label=(\alph*)]

\item By induction: Suppose $F \subseteq E$ is isotropic. If $F$ is Lagrangian, we're done. Otherwise, $F \subset F^\omega$ is a proper subspace. Pick $v \in F^\omega \setminus F$. Then $F' = F + \operatorname{Span}\{v\}$ is again isotropic. The process ends when it becomes Lagrangian.

\item By induction: suppose $F \subseteq E$ is isotropic, with $F\cap M = \{0\}$. If $F$ is Lagrangian, we are done. Otherwise, $F + (F^\omega \cap M) \subseteq F^\omega$ is an isotropic subspace, hence is a proper subspace of $F^\omega$. Pick $v \in F^\omega\setminus (F + (F^\omega \cap M))$; $F' = F + span\{v\}$

\claim $F' \cap M = \{0\}.$

\proof

Indeed: if $y \in F'\cap M$, write $y = x + tv$, $x \in F, t \in \R$. Then
\[
tv = y - x \in (F + M) \cap F^\omega = F + (F^\omega \cap F)
\]

\qed

\end{enumerate}

\qed

\underline{Exercise:} Given $L \in \Lag(E, \omega)$. Let $F \subseteq E$ be any complement, i.e. $E = L \oplus F$. 
\begin{enumerate}[label=(\roman*)]

\item Show that there exists a unique linear map $A:F\to L$ such that $F^\omega = \{v + Av\mid v \in F\}$

\item Show that all $F_t = \{v + tAv \mid v \in F\}$ is a complement to $L$.

\item $F_{\frac12}$ is Lagrangian

\end{enumerate}

Given $(E, \omega)$, we can choose a Lagrangian splitting $E = L \oplus M$. 

\prop The choice of splitting identifies $M\cong L^*$ and determines a symplectomorphism 
\[
E\to L\oplus L^*
\]

\proof

Every $w \in M$ defines a linear functional $\alpha_\omega \in L^*$, $\alpha_\omega(v) = \omega(v, w)$

The map $M \to L^*$, $w \mapsto \alpha_w$ is an isomorphism, using the non-degeneracy of the symplectic structure. The resulting map $E \cong L \oplus M \to L \oplus L^*$ is a symplectomorphism (by formula for sympletic structure on $L \oplus L^*$).

\qed

\prop

For every symplectic $(E, \omega)$, there exists a symplectomorphism $E \to \R^{2n}$, where $\R^{2n}$ has the standard symplectic structure. 

\proof

Choose a Lagrangian splitting $E = L \oplus L^*$. Now pick basis of $L, $ dual basis of $L^*$ to identify $E \cong \R^{2n}$

\underline{Remark:}

$\Lag(\R^2) = \RP(1) \cong S^1$

$\Lag(\R^4) = ?$

\underline{Exercise:}

Given $L \in \Lag(E,\omega)$, show that the set of all 
\begin{itemize}

\item complement to $L$ is an affine space with corresponding linear space $\Hom(E/L,L)$ (note if $L$ is lagrangian, then $E/L$ is naturally identified with $L^*$).

\item Lagrangian complements to $L$ is an affine space with corresponding linear space the self-adjoint maps $L^* \to L$.

\end{itemize}

In general, $\dim \Lag(\R^{2n}) = \frac{n(n + 1)}{2}$

\subsection*{\underline{Linear Reduction:}}

Let $(E, \omega)$ be symplectic. $F \subseteq E$ is \underline{symplectic} if $\omega|_{F\times F}$ is nondegenerate. Equivalently, $\underbrace{F\cap F^\omega}_{=\ker \omega|_{F\times F}} = \{0\}$

Note that $F$ is symplectic if and only if $F^\omega$, and $E = F \oplus F^\omega$.

In general, if $F$ is not symplectic, we can make it symplectic by quotienting by $\ker(\omega|_{F\times F}) = F\cap F^\omega$. 

\prop

For any subspace $F$, the quotient $E_F = F/(F\cap F^\omega)$ inherits a symplectic structure: 
\[
\omega_F(\pi(v), \pi(w)) = \omega(v, w)
\]
where $\pi$ is the quotient map $\pi:F\to F/(F\cap F^\omega)$

\proof

It's well defined: E.g, if $\pi(v) = 0$, then $v \in F \cap F^\omega$, so $\omega(v, w) = 0$ for all $w\in F$. 

It's non-degenerate: If $\pi(v)\in \ker(\omega_F)$, then $\omega(v, w) = 0$ for all $w \in F$, so $v \in F \cap F^\omega$, so $\pi(v) = 0$. 

Note: For $F$ coisotropic, $E_F = F/F^\omega$

\prop

For $F$ coisotropic, $L \subseteq E$ Lagrangian, the subspace $\pi(L\cap F) = L_F$ is again Lagrangian.

\proof

Clearly, $L_F$ is isotropic. To show $L_F$ is Lagrangian, count dimension: $(L_F) = (L \cap F)/(L \cap F^\omega)$.
\begin{align*}
\dim (L \cap F^\omega) & = \dim E - \dim (L \cap F^\omega)^\omega \\
& = \dim E - \dim (L + F) \\
& = \underbrace{\dim E}_{2\dim L} - \dim L - \dim F + \dim(L\cap F)\\
& = \dim L - \dim F + \dim (L \cap F)
\end{align*}

So 
\begin{align*}
\dim (L_F) & = \dim(L\cap F) - \dim(L\cap F^\omega) \\
& = \dim F - \dim L \\
\dim E_f & = \dim F - \dim F^\omega \\& = 2\dim F - \dim E \\ & = 2(\dim F - \dim L) 
\end{align*}

\qed

So, we have constructed a map $\Lag(E, \omega) \to \Lag(E_F, \omega_F), L \mapsto L_F$

\underline{Warning:} This map is not continuous!

It is discontinuous at the set of $L$'s where $L, F$ are not transverse. Away from this set, it's smooth. 

\underline{Exercise:}

Let $E = \R^4$ with standard symplectic basis. Take 
\[
F = \operatorname{span}\{e_1, e_2, f_1\}
\]
Then $F^\omega = \operatorname{span}\{e_2\}$. 

$F/F^\omega \cong \R^2 = \operatorname{span}\{e_1, f_1\}$. 

Let $L_t = \operatorname{span}\{e_1 + tf_2, e_2 + tf_1\}$. 
\begin{enumerate}[label=(\alph*)]

\item Check $L_t$ are Lagrangian

\item Compute $(L_t)_F \subseteq \R^2$ and find it's discontinuous at $t = 0$. 

\end{enumerate}

\subsection*{\underline{Compatible complex structures}}

\underline{Recall:} 

Given a complex vector space $V$, we can always regard it as a real vector space of twice the dimension. ``Multiplication by $\sqrt{-1}"$ becomes a real linear transformation $\mathcal{J} \in \Hom_{\R}(V,V)$, $\mathcal{J}^2 = -I$. 

Conversely, a real vector space with such a $\mathcal{J}$ is called a \underline{complex structure}, and we can imbue it with complex multiplication by defining
\[
(a + ib)v = av + b(\mathcal{J}v)
\]

\defn

Let $(E, \omega)$ be a symplectic vector space. A complex structure $\mathcal{J}$ (meaning $\mathcal{J}^2 = -I)$ is \underline{$\omega$-compatible} if 
\[
g(v, w) \eqdef \omega(v, \mathcal{J}w)
\]
defines an inner product. Denote by $\mathcal{J}(E, \omega) \eqdef \{\omega-$compatible complex structure $\}$.

Given $j \in \mathcal{J}(E, \omega)$, we get a complex inner product by 
\[
h(v, w) \eqdef g(v, w) + \sqrt{-1}\omega(v, w)
\]

\underline{Remark:} $\mathcal{J} \in \mathcal{J}(E,\omega)$ is a symplectomorphism: 
\begin{align*}
\omega(\mathcal{J}v, \mathcal{J}w) & = g(\mathcal{J}v, w) \\
& = g(w, \mc{J}v) \\
& = \omega(w, \mc{J}^2v) \\
& = -\omega(w, v)\\
& = \omega(v, w) \\
\end{align*}

\underline{Remark:} For $\R^{2n}$, there is a standard complex structure given by $\mc{J}(e_i) = f_i, \mc{J}(f_i) = -e_i$.

This identifies $\R^{2n} \cong \C^n$. We can come up with more complex structures by picking an $A \in \Sp(E, \omega)$ and considering $\mc{J} \mapsto A\mc{J}A^{-1}$. 

We have a map $\mc{J}(E,\omega) \to \operatorname{Riem}(E)$ (real inner products) given by $\mc{J} \mapsto g$, where $g$ is as above. 

There is a canonical left inverse $\varphi:\operatorname{Riem}(E) \to \mc{J}(E,\omega)$ as follows: 

\prop There is a canonical retraction (in the sense of topology) from $\operatorname{Riem}(E) \to \mc{J}(E, \omega)$.

\proof

Given $k \in \Riem(E)$, define $A \in \GL(E)$ by $k(v, w) = \omega(v, Aw)$. 

$A$ is not a complex structure in general, but it's skew-symmetric with respect to $h$: 
\[
A^T = -A
\]
Define $|A| = (A^TA)^\frac12 = (-A^2)^\frac12$. This commutes with $A$ by functional calculus, and define $\mc{J} = A|A|^{-1}$. This will do the job. 

\qed

\section*{Lecture 3 - 9/12/24}

Let $(E, \omega)$ be a symplectic vector space. Let $\mc{J} \in \Hom(E, E), \mc{J}^2 = -I$. $Y$ is \underline{$\omega$-compatible} if 
\[
g(v, w) \eqdef \omega(v, \mc{J}w)
\]
is a (real) inner product. 

\exm 

Let $E = \R^{2n} = \Span\{e_1, \dots, e_n, f_1, \dots, f_n\}$. Let $\mc{J}e_i = f_i, \mc{J}f_i = -e_i$. Then $g$ is the standard inner product. 

\underline{Remark:} 
\begin{itemize}

\item Note: in the definition of $\omega$-compatible, any two of $\omega, \mc{J}, g$ deetermines the third. 

\item $\mc{J} \in \Sp(E,\omega) \cap O(E, g)$

\item $h(v, w) = g(v, w) + i\omega(v, w)$ is a \underline{complex} inner product, with corresponding unitary group $U(n) = \underbrace{U(E, h)}_{\text{preserves }h} = \underbrace{\Sp(E, \omega)}_{\text{preserves }\omega} \cap \underbrace{O(E, g)}_{\text{preserves }g}$. Recall that $U(n)$ is compact and connected, and has $\pi_1 = \Z$

\end{itemize}

Let $\mc{J}(E, \omega) = \{\mc{J} \mid  \omega$-compatible $\}$

We have a map $\psi:\mc{J}(E,\omega) \to \Riem(E) \subseteq \Sym^2(E)$, which is contractible.

\thm There is a canonical retraction 
\[
\phi:\Riem(E) \to \mc{J}(E, \omega)
\]
such that $\phi \circ \psi = \Id$.

\cor $\mc{J}(E, \omega)$ is contractible
\proof

Send 
\[
\mc{J} \mapsto \varphi\left((1 - t)\psi(\mc{J}) + tg_0\right)
\] 

\qed

\proof

Let $k \in \Riem(E)$ be given. Define $A \in \Hom(E, E)$ by $k(v, w) = \omega(v, Aw)$. 

Then $A = -A^T$ (skew-adjoint with respect to $k$). Why? note that 
\begin{align*}
k(v, A^{-1}y) &= \omega(v, y) \\&= -\omega(y, v)\\&=-k(y, A^{-1}v\\&=-k(A^{-1}v, y)
\end{align*}

So $(A^{-1})^T = A^{-1}$, so $A^T = -A$.

Hence, we can define $|A| = \sqrt{A^TA} = \sqrt{-A^2}$. Put $\mc{J} = A|A|^{-1}$. Then $\mc{J}^2 = A|A|^{-1}A|A|^{-1} = A^2|A|^{-2} = -I$. 

Now we check that it defines an inner product: 
\begin{align*}
g(v, w) & = \omega(v, \mc{J}w) \\
& = \omega(v, A|A|^{-1}\omega) \\
& = k(v, |A|^{-1}w) \\
& = k(|A|^{-\frac12}v, |A|^{-\frac12}w)
\end{align*}

is an inner product. 

\qed

If $k$ was instead $g$ with a compatible complex structure, then $A$ must be that complex structure on the nose. 

Now, $\Sp(E, \omega)$ acts on $\mc{J}(E, \omega)$ by 
\[
A \cdot \mc{J} = A\mc{J}A^{-1}
\]

\prop The action of $\Sp(E, \omega)$ on $\mc{J}(E, \omega)$ is transitive. That is, for any $\mc{J}_1, \mc{J}_2 \in \mc{J}(E, \omega)$, there is an $A \in \Sp(E,\omega)$ with $\mc{J}_1 = A\mc{J}_2A^{-1}$. It has stabilizers at $\mc{J}\in\mc{J}(E,\omega)$ the unitary group $U(E)$, with respect to $\mc{J}$. I.e.:
\[
\mc{J}(E, \omega) = \Sp(E,\omega)/U(E)
\]

\cor

$\Sp(E,\omega)$ is connected. 

\proof

$\mc{J}(E,\omega)$ is connected, and $U(E)$ is connected, so $\spew$ is connected. 

\qed

\proof

Given $\mc{J}, \mc{J}' \in \jew$, let $e_1, \dots, e_n$ be an orthonormal basis for $E$ a complex inner product space (with respect to $\mc{J}$).

Then $e_1, \dots, e_n, f_1 = \mc{J}e_1, \dots, f_n = \mc{J}e_n$ is a symplectic basis. Similarly, define $e_1', \dots, e_n', f_1', \dots, f_n'$ by $Ae_i = e_i', Af_i = f_i'$. This defines a symplectic transformation $A \in \spew$, with $A\J A^{-1} = \J'$.

\qed

\subsection*{More on $\spew$}

\prop

$\spew$ is a connected Lie group of dimension $2n^2 + n$, where $\dim E = 2n$

\proof

By Cartan's theorem, every closed (in the sense of topology) subgroup of a Lie group is a Lie group. 

This applies to $\spew \subseteq \GL(E)$ (invertible transformations). For connected, see above. 

To get dimension, consider action of $\GL(E)\supseteq \mc{U} = \{$symplectic forms$\}$ on $\bigwedge^2 E^*$ the space of skew-symmetric bilinear forms.  This action is transitive, with stabilizer at $\omega \in \mc{U}$ given by $\spew$. 

Hence $\mc{U} = \GL(E)/\spew$, and using this we can count dimensions:
\[
\underbrace{\dim \mc{U}}_{\dim = \dim\bigvee^2 E^* = {2n\choose 2}} = \underbrace{\dim \GL(E)}_{\dim = (2n)^2} - \dim \spew
\]

So $\dim \spew = (2n)^2 - \frac{2n(2n-1)}{2} = 2n^2 + n$ 

\section*{Lecture 4 - 9/17/24}

\subsection*{\underline{Geometry of $\spew$ and $\Lag(E,\omega)$}}

So far, we know $\spew$ is a Lie group. It is also connected, with dimension $2n^2 + n$, where $\dim E = 2n$. 

For any $\mc{J}\in\jew$, we have a real inner product $g(v, w) = \omega(v, \mc{J}w)$, and a complex one given by $h = g + \sqrt{-1}\omega$. 

$U(E) \subseteq \spew$, $\jew=\spew/U(E)$

(Enough to consider $E = \R^{2n}$, $\mc{J}e_i = f_i, \mc{J}f_i = -e_i$). 

Let $()^T$ be the transpose with respect to the metric $g$. In other words, $g(Av, w) = g(v, A^Tw)$. 

\prop

$A \in \GL(E)$ is \underline{symplectic} if and only if $A^T = \mc{J}A^{-1}\mc{J}^{-1}$. 

\proof

$A\in\spew$ if and only if for all $v, w, \omega(Av,Aw) = \omega(v, w)$. Note that $g(\mc{J}v,w) = \omega(v, w)$. So for all $v, w$, 
\begin{align*}
g(\mc{J}Av, Aw) & = g(\mc{J}v, w) \\
& = g(A^T\mc{J} Av, w) 
\end{align*}
so $A^T\mc{J}A = \mc{J}$ .

\qed

Consequence: if $A\in\spew$, then $\det(A) = 1$. This follows from 
\begin{align*}
\det(A) & = \det(A^T) \\
& = \det(\mc{J}A^{-1}\mc{J}^{-1}) \\
& = \det(A)^{-1} 
\end{align*}
So $\det(A)^2 = 1$, and since $\spew$ is connected, $\det(A) = 1$. 

Now, in $\R^{2n}$, if we let $A = \pmat{a}{b}{c}{d}$, $\mc{J} = \pmat{0}{I}{-I}{0}$.
\[
A \in \spew \iff \begin{cases} a^Tc = c^Ta \\ b^Td = d^Tb \\ a^Td - b^Tc = I \\ \end{cases}
\]

Note: $\spew(\R^2, \omega) = \SL(2,\R)$, the 2x2 matrices of determinant 1. 

Another consequence is: $A$ is symplectic implies that $A^T$ is symplectic. This means we can use \underline{polar decomposition}. 

Recall: $A \in \GL(m, \R)$ has a polar decomposition $A = U|A|$, where $|A|$ is positive definite, $|A| = \sqrt{A^TA}$, and $U \in O(n)$. 

Can write $|A| = \exp(\xi)$, with $\xi^T = \xi$. 

Thus, $\GL(n,\R) = O(n)\times\{\xi \mid \xi^T = \xi\}$ as a manifold. 

For any $G \subseteq \GL(m,\R)$, which is invariant under $A \mapsto A^T$, we get a polar decomposition 
\[
G = K\times p
\]
where $K = G\cap O(m)$, $p = \mk{g} \cap \{\xi \mid \xi^T = \xi \}$
where $\mk{g} = \{\xi \mid \exp(t\xi) \in G \text{ for all }t\}$

In particular, $G = \spew \cong \Sp(2n, \omega)$, we get $K = \Sp(2n,\omega) \cap O(2n) = U(n)$ and $p = \{\xi \in \Sp(2n, \omega) \mid \xi = \xi^T\}$. 

So we get that $\spew = U(e)\times p$. 

Upshot: $\spew$ deformation retracts onto its maximal compact subgroup $U(E)$. 

\cor

There is a canonical isomorphism 
\[
\mu: \pi_1(\spew)\to\Z
\]

\proof

$\pi_1(\spew) \cong \pi_1(U(E))$. Now, $\det:U(E) \to \pi_1(U(1))$, and this map is an isomorphism. 

This is the most primitive version of a ``Maslov Index." It is a ``Maslov index of loop of symplectomorphisms. 

\underline{Exercise:} If $A, B:S^1 \to \spew$, then $\mu(AB) = \mu(A) + \mu(B)$. 

\underline{Remark:} We discussed the noncompact group $S_p(E, \omega) = \Sp(2n,\R)$. There is a compact ``symplectic group" denoted $\Sp(n)$. Both are ``real forms" of the complex symplectic group $\Sp(2n, \C)$. 
\[
\Sp(2n, \R) \subseteq \Sp(2n,\C) \supseteq \Sp(n)
\]
but $\Sp(2n,\R)\neq\Sp(n)$. We have a similar situation for $\SL(n,\C)$: 
\[
\SL(n,\R) \subseteq \SL(n,\C) \supseteq \SU(n)
\]
These two on the left and right have the same complexification.
\[
\R^* \subseteq \C^* \supseteq U(1)
\]

Recall: The Lagrangian Grassmannian, $\Lag(E) = \{L \subseteq E \mid L^\omega = L \}$. $\Lag(E) \subseteq GR_n(E) = \{n$-dimensional subspaces of $E\}$, so it is a topological space in this way. 

Recall that $Gr_k(E)$ can be seen as a manifold in 2 ways:
\begin{enumerate}[label=(\roman*)]

\item View it as a homogeneous space

\item Construct charts

\end{enumerate}

\begin{enumerate}[label=(\roman*)]

\item Pick any $G \subseteq \GL(E)$ such that $G$ acts transitively on $Gr_k(E)$ (e.g. $G = \GL(E), G = O(E)$ for some inner product, $G = \SO(E)$). 

Let $H \subseteq G$ be a stabilizer of some \underline{fixed} $k$-dim subspace. So $Gr_k(E) = G/H$

\item For any subspace $M \subseteq E$ of codimension $k$, the set  $\{L \in Gr_k(E) \mid E = L \oplus M \}$ is canonically an affine space. It is isomorphic to $\{j:E/M\to E \mid \pi\circ j = \Id\}$, $\pi: E\to E/M$, an affine space under $\Hom(E/M,E)$

The punchline is that for any fixed $L$, we get a vector space, and we use this as a chart. 


\end{enumerate}



Now, we want to do the same thing with $\Lag(E)$. 

\prop

The group $U(E)$ acts transitively on $\Lag(E)$ with stabilizers at given $L \in \Lag(E)$ equal to $O(L)$. $U(E)$ is a Lie group, so $\Lag(E) = U(E)/O(L)$ is a manifold of dimension $\frac{n(n+1)}{2}$

\proof

Let $h(v, w) = g(v, w) + \sqrt{-1}\omega(v,w)$. 

Note: On any $L \in \Lag(E)$, get $h|_{L\times L} = g|_{L\times L}$. A $g$-orthonormal basis $e_1, \dot, e_n$ of $L$ is an $h$-orthonormal basis of $E$, given symplectic basis $e_1, \dots, e_n, f_i, \dots, f_n$, where $f_i = \mc{J}e_i$. 

Given another $L'$, choose $g$-orthonormal basis $e_1',\dots,e_n'$ of $L'$. It's $h$-orthonormal basis of $E$. 

The transformation taking $e_1, \dots, e_n$ to $e_1', \dots, e_n'$ is in $U(E) \subseteq \spew$ taking $L$ to $L'$. The stabilizers of $L$ are transformations for which $L = L'$, so they're in $O(L)$. 
Then
\begin{align*}
\dim\Lag(E) & = \dim U(n) - \dim O(n) \\
& = n^2 - \frac{n(n-1)}{2} \\
&= \frac{n(n+1)}{2}
\end{align*}

\qed

Alternatively, pick $\mc{J}\in\jew$, then $\Lag(E) = \Sp(E) / \Sp(E)_L$ (?)

We get again a version of the Maslov index by Arnold. 

The map $\det^2:U(n) \to U(1), A \mapsto (\det(A))^2$ descends to a map $\Lag(\R^{2n}) \to U(1)$, hence gives a map on fundamental groups
\[
\mu:\pi_1(\Lag(E)) \to \pi_1(U(1)) = \Z
\]

\prop(Arnold)

This map is again an isomorphism

\proof

\qed

This is the maslov index of loop of Lagrangian subspaces.

Special Case $n = 1$

$\Lag(\R^2) = U(1)/O(1)$. $O(1) = \{\pm1\}$, so this is a circle under polar identifications, so we get $\RP^1$, which is again $S^1$. 

Given $M \in \Lag(E)$, let $\Lag(E; M) = \{L\in\Lag(E) \mid E = L\oplus M \}$

\prop $\Lag(E;M)$ is canonically an affine space, with corresponding linear spaces $\Sym^2(M) = \{$ symmetric bilinear forms on $M^*\} \cong$ self adjoint maps $M^*\to M$. 

\proof

Let $\pi:E\to M^*$ where $\pi(E) = $ restriction of $\omega^\flat(v) \in E^*$ to $M$. 

$\pi(v)(w) = \omega(v,w)$ for $w\in M$. 

This projection map has kernel $M\subseteq E$ since $M$ is Lagrangian, so gives isomorphisms $E/M\to M^*$. 

$\Lag(E;M) = \{L \in \Lag(E) \mid L \oplus M = E\}\cong \{j:E/M \to E \mid j(M^*)$ is isotropic $, \pi\circ j= \Id\}$. 

Given any such $j$, any other splitting $j'$ is of the form $j'(m) = j(m) + \psi(m)$ for some $\psi:E/M = M^*\to M$.

For all $\mu_1,\mu_2\in M^*$, 
\begin{align*}
0 & = \omega(j'(\mu_1),j'(\mu_2)) \\
& = \omega(j(\mu_1) + \psi(\mu_1), j(\mu_2) + \psi(mu_2) \\
& = \underbrace{\omega(j(\mu_1), j(\mu_2))}_{=0\text{ since $j$ isotropic }} + \underbrace{\omega(\psi(\mu_1),\psi(\mu_2))}_{=0\text{ since $M$ isotropic}} \\
& + \omega(j(\mu_1), \psi(\mu_2)) + \omega(\psi(\mu_1),j(\mu_2)) \\
& = \langle \mu_1, \psi(\mu_2)\rangle - \langle \mu_2, \psi(\mu_1) \rangle \\
\end{align*}

So $\psi$ is self-adjoint, $\beta(\mu_1,\mu_2) = \langle \mu, \psi \mu \rangle$. 

Again we see $\dim = \frac{n(n + 1)}{2}$.

Down-to-earth version

Let $E = \R^{2n}$, $M = 0\oplus\R^n = \Span\{f_1, \dots, f_n\}$. 

$\R^{2n} = L\oplus M$ means $L$ is graph of linear map $S: \R^n\to\R^n$.

$L$ has basis 
\[
g_i = e_i + \sum_{j=1}^n S_{ij}f_j
\]
is Lagrangian if and only if for all $i, k$, 
\begin{align*}
0 = \omega(g_i, g_k) & = \omega(e_i + \sum S_{ij}f_j, e_k + S_{kl}f_l \\
& = \cdots \\
& = S_{ki} - S_{ik}
\end{align*}

\section*{Lecture 5 - 9/19/24}

\subsection*{\underline{Maslov Indices}}

Let $(E,\omega)$ be a symplectic vector space, $\dim E = 2n$. We consider $\Lag(E)$, the Lagrangian Grassmannian, the set of all Lagrangian subspaces. We know this is homeomorphic to $U(n)/O(n)$, once you choose a symplectic basis. It is a manifold, of dimension $frac{n(n+1)}{2}$. 

We have $\pi_1(\Lag(E)) \cong \pi_1(U(n)/O(n))$. The function $\det^2$ descends to a function on this space, and gives a morphism from $\pi_1(U(n)/O(n))$ to $\pi_1(U(1)) \cong \Z$. 

So we have a canonical $\mu:\pi_1(\Lag(E)) \to \Z$ is called the \underline{Maslov Index}. It is somewhat akin to winding number. 

We want to generalize to \underline{paths} of Lagrangians. Fix a Lagrangian subspace $M\in\Lag(E)$, and define
\[
\Lag(E,M) = \{L\in \Lag(E) \mid L \cap M = 0 \}
\]
This is canonically an affine space, and so is contractible. Let $\sum_M = \Lag(E) \setminus \Lag(E,M) = \{L \mid L \cap M \neq 0\}$. This is some kind of singular space. 

Consider a path $L:[a, b] \to \Lag(E), t \mapsto L(t)$ with $L(a), L(b) \not\in\sum_M$

Define the Maslov Index $[L:M]\eqdef$ Maslov index of \underline{loop} obtained by concatenating $L(t)$ with a path in $\Lag(E,M)$ to make a loop. The contractibility of this space means the choice of path doesn't matter. 

This is Maslov's ``original" index as intersection number with the sincular cycle $\sum_M$. 

More generally, we want to find $[L_1:L_2]$ for two arbitrary Lagrangian paths, a kind of signed number of nonzero intersections $L_1(t) \cap L_2(t)$ (remember that $L_i(t)$ is a vector space!). 

Let $L_1, L_2 \in \Lag(E,M)$ related by some $\beta_{12} \in \Sym^2(M)$ (symmetric bilinear form on $M^* \cong L_1$. Recall from last time we have $\beta_{12} \cdot L_1 = L_2$, $\beta_{21} \cdot L_2 = L_1$, $\beta_{12} = \beta_{21}$. To this setting we can attach an invariant. 

Does there exist a symplectomorphism $A$ such that $(L_1, L_2, M) \mapsto_A (L_1, L_2', M)$, with $L_2'$ a Lagrangian transverse to all the others. 

\subsection*{\underline{Signature of symmetric bilinear form}}

For a symmetric matrix $B$, we say the Signature, $\Sig(B)$, is the number of positive eigenvalues minus the number of negative eigenvalues. 

For a symmetric bilinear form $\beta$, $\Sig(\beta) = \Sig(B)$, for $B$ the matrix of $\beta$ in terms of a basis (which does not affect the eigenvalues). 

The number $\Sig(\beta_{12})$ depends on $M$. 

\prop

If $L_1,L_2,L_3 \in \Lag(E;M)$. Then the number $s(L_1,L_2,L_3) = \Sig(\beta_{21}) + \Sig(\beta_{32}) + \Sig(\beta_{13})$ is actually independent of $M$. This is also called a Maslov index. 

\proof

\prop
\,

\begin{enumerate}

\item $s(L_1,L_2,L_3) = s(L_2,L_3,L_1)$

\item $S(L_1,L_2,L_3) = -S(L_2,L_1,L_3)$

\item Cocycle identity: for all Lagrangians $L_1,L_2,L_3,L_4$, 
\[
S(L_2,L_3,L_4) - s(L_1,L_3,L_4) + s(L_1,L_2,L_4) - s(L_1,L_2,L_3) = 0
\]

\item If $M(t)$ is always transverse to $L_1,L_2$, then $s(L_1,L_2, M)$ doesn't depend on $t$. 

\item Up to symplectomorphism, $L_1,L_2,L_3$ is uniquely determined by $\dim(L_1\cap L_2), \dim(L_2\cap L_3), \dim(L_1\cap L_3), \dim(L_1\cap L_2\cap L_3)$, s$(L_1,L_2,L_3)$

\end{enumerate}

\prop

Suppose $[a,b]\to\Lag(E), t \mapsto L_i(t)$, $i = 1, 2$ are paths, and that there exists some $M\in\Lag(E)$ such that $L_i(t) \cap M = 0$ for all $i = 1, 2, t \in [a,b]$. Then
\[
[L_1;L_2] = \frac12(s(L_1(a),L_2(a),M) - s(L_1(b),L_2(b),M))
\]

\proof

Suppose $M'$ is another choice. First term changes by 

\begin{align*}s(L_1(a),L_2(a),M') - s(L_1(a),L_2(a),M')  &= s(L_1(a),M,M') - s(L_2(a),M,M') \\
& = s(L_1(b),M,M') - s(L_2(b),M,M') \\
\end{align*}
This is the change in the second term, so they cancel out.

General definition: 

Consider a partition $a = t_0 < t_1 < \dots < t_k = b$ such that for all $[t_{j-1},t_j] \in M_j\in\Lag(E)$ with $L_i(t) \cap M_j = 0$ for all $t \in [t_{j-1},t_j]$

Then 
\[
[L_1;L_2] = \frac12\sum_{j=1}^k\left(s(L_1(t_{j-1}),L_2(t_{j-1}),M_j) - s(L_1(t_j),L_2(t_j),M_j)\right)
\]

For $A\in\spew$, we have $\operatorname{Graph}(A) = \{(Av,v)\}\subseteq E\times \bar{E}$ (where $\bar{E}$ has the same symplectic form but with an opposite sign) is Lagrangian. 

Define, for any path $A(t)$, $\mu(A) = [\operatorname{Graph}(A), \bigtriangleup]$

\section*{Lecture 6, 9/24/24}

\section*{\underline{Part 2: Symplectic Manifolds}}

Recall that the Lie derivative of a vector field, $\ms{L}_X$, is given by $\dd{}{t}|_{t=0}(F_{-t})^*$, where $F_t$ is a flow along $X$. Note $X(f) = \ms{L}_Xf = \dd{}{t}|_{t=0}(F_{-t})^*f$

Differential of a map $F:M_1\to M_2$ is a map $TF:TM_1\to TM_2$.

For $f:M\to R$, $Tf:TM\to T\R$, while $df\in\Omega^1(M)$.

We will introduce symplectic manifolds by analogy to a complex manifold. 

\subsection*{Complex manifolds}

A complex manifold comes with a family of linear transformations $\mc{J}_m:T_mM\to T_mM$, with $\mc{J}_m^2 = -\Id_{T_mM}$, which depends smoothly on $m\in M$. 

This is typically called an ``almost complex structure".

A complex manifold is the same as a real manifold, but charts go to $\C^n$, and we want transition functions to be holomorphic. Any complex manifold has an almost complex structure on its tangent spaces, but a manifold with an almost complex structure is not necessarily a complex manifold. 

It is some kind of \underline{integrability condition} on $\mc{J} = \{\mc{J}_m\}$, and if this condition vanishes, then the almost complex structure comes from an honest complex manifold. 

\subsection*{Symplectic manifolds}

A symplectic manifold is equipped with a family of functions $\omega_m:T_mM\times T_mM \to \R$ which is symplectic, depending smoothly on $m$. This is called an ``almost symplectic structure." There is again an integrability condition we can impose. We want to stipulate that $\omega_m$ arises from an $\omega = \{\omega_m\}\in \Omega^2M$. The integrability condition is that $d\omega = 0$. 

\defn

A \underline{symplectic structure} on a manifold $M$ is a non-degenerate 2-form $\omega\in\Omega^2(M)$ with $d\omega = 0$.

Non-degenerate just means that each $\omega|_{T_mM\times T_mM}$ is non-degenerate. 

\prop

Any symplectic 2-form $\Omega$, for $\dim M = 2n$, is non-degenerate if and only if $\underbrace{\omega^n}_{=\underbrace{\omega\wedge\cdots\wedge\omega}_{n\text{ times}}}\neq0$ everywhere. 

\proof

Check at $m \in M$. 

In one direction,  suppose $(\omega_m)^n\neq0$. We want to show $\ker\omega_m = 0$. Since $(\omega_m)^n \neq0$, we have $\iota_V(\omega_m^n)$, where we define $\iota_v:\Omega^k(M)\to\Omega^{k-1}(M)$ by $\iota_V(\alpha) = \alpha(V, \cdots)$. 

Anyways, $\iota_V(\omega^n)$ is nonzero for all $v \in T_m$. 

But $\iota_V(\underbrace{\omega_m\wedge\cdots\wedge\omega_m}_{n}) = n(\iota_v\omega_m)\omega_m^{n-1}$, so $\iota_V\omega_m\neq0$. 

In the other direction, suppose $\ker(\omega_m) = 0$, so $\omega_m$ is symplectic. Let $e_1, \dots, e_n, f_1, \dots, f_n$ be a symplectic basis for $T_mM$ with respect to $\omega_m$.

Consider 
\begin{align*}
\iota_{e_n}\iota_{e_{n-1}}\cdots\iota_{e_1}(\omega_m^n) & = n(\iota_{e_n}\cdots\iota_{e_2})((\iota_{e_1}\omega_m)\wedge \omega_m^{n-1}) \\
& = n(n-1)(\iota_{e_n}\cdots\iota_{e_3})((\iota_{e_2}\omega_m)(\iota_{e_1}\omega_m)\omega_m^{n-2} \\
& \vdots \\
& = n!(\iota_{e_n}\omega_m)\wedge\cdots\wedge(\iota_{e_1}\omega_m)
\end{align*}

So $\iota_{f_n}\cdots\iota_{e_1}(\omega_m^n) = \pm n! \neq 0$

\defn

Let $(M,\omega)$ be an (almost) symplectic manifold. The volume form 
\[
\bigwedge = \frac{\omega^n}{n!} = (\exp(\omega))_{\dim M}
\]
is called the \underline{Liousville volume form} on $M$. 

\defn

Let $(M,\omega)$ be a symplectic manifold.
\begin{enumerate}[label=(\alph*)]

\item A \underline{symplectomorphism} is a diffeomorphism $F \in \operatorname{Diff}(M)$ preserving $\omega$, i.e. $F^*\omega = \omega$. The group of symplectomorphisms of $(M,\omega)$ is denoted $\Diff(M,\omega)$

\item A \underline{symplectic vector field} on $M$ is a vector field $X \in \ms{X}(M)$ preserving $\omega$, i.e. $\ms{L}_X\omega = 0$.

The Lie algebra of symplectic vector fields is denoted $\ms{X}(M,\omega)$, i.e. the Local flow is symplectic. 

\end{enumerate}

\defn

Let $(M,\omega)$ be a symplectic manifold, let $H\in C^\oo(M)$. 

The \underline{Hamiltonian vector field} $X_H \in \ms{X}(M)$ is the unique vector field such that 
\[
\iota(X_H)\omega = -dH
\]
The space of Hamiltonian vector fields is denoted $\ms{X}_{Ham}(M,\omega)$

\prop Indeed, $\ms{X}_{Ham}(M,\omega) \subseteq \ms{X}(M,\omega)$.

\proof

Let $X = X_H$ be Hamiltonian. We check 
\begin{align*}
\ms{L}_X\omega & = (d\iota_X + \iota_Xd)\omega \\
& = d\iota_X\omega \\ 
& = -ddH \\
& = 0\\
\end{align*}
by the Cartan Formula

\qed

It turns out that $\omega$ is symplectic if and only if $\iota_X\omega$ is closed. 

$X$ is Hamiltonian if and only if $\iota_X$ is exact. 

\end{document}
