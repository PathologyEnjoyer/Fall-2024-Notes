
\documentclass[x11names,reqno,14pt]{extarticle}
% Choomno Moos
% Portland State University
% Choom@pdx.edu


%% stupid experiment %%
%%%%%%%%%%%%% PACKAGES %%%%%%%%%%%%%

%%%% SYMBOLS AND MATH %%%%
\let\oldvec\vec
\usepackage{authblk}	% author block customization
\usepackage{microtype}	% makes stuff look real nice
\usepackage{amssymb} 	% math symbols
\usepackage{siunitx} 	% for SI units, and the degree symbol
\usepackage{mathrsfs}	% provides script fonts like mathscr
\usepackage{mathtools}	% extension to amsmath, also loads amsmath
\usepackage{esint}		% extended set of integrals
\mathtoolsset{showonlyrefs} % equation numbers only shown when referenced
\usepackage{amsthm}		% theorem environments
\usepackage{relsize}	%font size commands
\usepackage{bm}			% provides bold math
\usepackage{bbm}		% for blackboard bold 1

%%%% FIGURES %%%%
\usepackage{graphicx} % for including pictures
\usepackage{float} % allows [H] option on figures, so that they appear where they are typed in code
\usepackage{caption}
\usepackage{hyperref}
%\usepackage{titling}
\usepackage{tikz} % for drawing
\usetikzlibrary{shapes,arrows,chains,positioning,cd,decorations.pathreplacing,decorations.markings,hobby,knots,braids}
\usepackage{subcaption}	% subfigure environment in figures

%%%% MISC %%%%
\usepackage{enumitem} % for lists and itemizations
\setlist[enumerate]{leftmargin=*,label=\bf \arabic*.}

\usepackage{multicol}
\usepackage{multirow}
\usepackage{url}
\usepackage[symbol]{footmisc}
\renewcommand{\thefootnote}{\fnsymbol{footnote}}
\usepackage{lastpage} % provides the total number of pages for the "X of LastPage" page numbering
\usepackage{fancyhdr}
\usepackage{manfnt}
\usepackage{nicefrac}
%\usepackage{fontspec}
%\usepackage{polyglossia}
%\setmainlanguage{english}
%\setotherlanguages{khmer}
%\newfontfamily\khmerfont[Script=Khmer]{Khmer Busra}

%%% Khmer script commands for math %%%
%\newcommand{\ka}{\text{\textkhmer{ក}}}
%\newcommand{\ko}{\text{\textkhmer{ត}}}
%\newcommand{\kha}{\text{\textkhmer{ខ}}}

%\usepackage[
%backend=biber,
% numeric
%style=numeric,
% APA
%bibstyle=apa,
%citestyle=authoryear,
%]{biblatex}

\usepackage[explicit]{titlesec}
%%%%%%%% SOME CODE FOR REDECLARING %%%%%%%%%%

\makeatletter
\newcommand\RedeclareMathOperator{%
	\@ifstar{\def\rmo@s{m}\rmo@redeclare}{\def\rmo@s{o}\rmo@redeclare}%
}
% this is taken from \renew@command
\newcommand\rmo@redeclare[2]{%
	\begingroup \escapechar\m@ne\xdef\@gtempa{{\string#1}}\endgroup
	\expandafter\@ifundefined\@gtempa
	{\@latex@error{\noexpand#1undefined}\@ehc}%
	\relax
	\expandafter\rmo@declmathop\rmo@s{#1}{#2}}
% This is just \@declmathop without \@ifdefinable
\newcommand\rmo@declmathop[3]{%
	\DeclareRobustCommand{#2}{\qopname\newmcodes@#1{#3}}%
}
\@onlypreamble\RedeclareMathOperator
\makeatother

\makeatletter
\newcommand*{\relrelbarsep}{.386ex}
\newcommand*{\relrelbar}{%
	\mathrel{%
		\mathpalette\@relrelbar\relrelbarsep
	}%
}
\newcommand*{\@relrelbar}[2]{%
	\raise#2\hbox to 0pt{$\m@th#1\relbar$\hss}%
	\lower#2\hbox{$\m@th#1\relbar$}%
}
\providecommand*{\rightrightarrowsfill@}{%
	\arrowfill@\relrelbar\relrelbar\rightrightarrows
}
\providecommand*{\leftleftarrowsfill@}{%
	\arrowfill@\leftleftarrows\relrelbar\relrelbar
}
\providecommand*{\xrightrightarrows}[2][]{%
	\ext@arrow 0359\rightrightarrowsfill@{#1}{#2}%
}
\providecommand*{\xleftleftarrows}[2][]{%
	\ext@arrow 3095\leftleftarrowsfill@{#1}{#2}%
}
\makeatother

%%%%%%%% NEW COMMANDS %%%%%%%%%%

% settings
\newcommand{\N}{\mathbb{N}}                     	% Natural numbers
\newcommand{\Z}{\mathbb{Z}}                     	% Integers
\newcommand{\Q}{\mathbb{Q}}                     	% Rationals
\newcommand{\R}{\mathbb{R}}                     	% Reals
\newcommand{\C}{\mathbb{C}}                     	% Complex numbers
\newcommand{\K}{\mathbb{K}}							% Scalars
\newcommand{\F}{\mathbb{F}}                     	% Arbitrary Field
\newcommand{\E}{\mathbb{E}}                     	% Euclidean topological space
\renewcommand{\H}{{\mathbb{H}}}                   	% Quaternions / Half space
\newcommand{\RP}{{\mathbb{RP}}}                       % Real projective space
\newcommand{\CP}{{\mathbb{CP}}}                       % Complex projective space
\newcommand{\Mat}{{\mathrm{Mat}}}						% Matrix ring
\newcommand{\M}{\mathcal{M}}
\newcommand{\GL}{{\mathrm{GL}}}
\newcommand{\SL}{{\mathrm{SL}}}

\newcommand{\tgl}{\mathfrak{gl}}
\newcommand{\tsl}{\mathfrak{sl}}                  % Lie algebras; i.e., tangent space of SO/SL/SU
\newcommand{\tso}{\mathfrak{so}}
\newcommand{\tsu}{\mathfrak{sl}}


% typography
\newcommand{\noi}{\noindent}						% Removes indent
\newcommand{\tbf}[1]{\textbf{#1}}					% Boldface
\newcommand{\mc}[1]{\mathcal{#1}}               	% Calligraphic
\newcommand{\ms}[1]{\mathscr{#1}}               	% Script
\newcommand{\mbb}[1]{\mathbb{#1}}               	% Blackboard bold


% (in)equalities
\newcommand{\eqdef}{\overset{\mathrm{def}}{=}}		% Definition equals
\newcommand{\sub}{\subseteq}						% Changes default symbol from proper to improper
\newcommand{\psub}{\subset}						% Preferred proper subset symbol

% Categories
\newcommand{\catname}[1]{{\text{\sffamily {#1}}}}

\newcommand{\Cat}{{\catname{C}}}
\newcommand{\cat}[1]{{\catname{\ifblank{#1}{C}{#1}}}}
\newcommand{\CAT}{{\catname{Cat}}}
\newcommand{\Set}{{\catname{Set}}}

\newcommand{\Top}{{\catname{Top}}}
\newcommand{\Met}{{\catname{Met}}}
\newcommand{\PL}{{\catname{PL}}}
\newcommand{\Man}{{\catname{Man}}}
\newcommand{\Diff}{{\catname{Diff}}}

\newcommand{\Grp}{{\catname{Grp}}}
\newcommand{\Grpd}{{\catname{Grpd}}}
\newcommand{\Ab}{{\catname{Ab}}}
\newcommand{\Ring}{{\catname{Ring}}}
\newcommand{\CRing}{{\catname{CRing}}}
\newcommand{\Mod}{{\mhyphen\catname{Mod}}}
\newcommand{\Alg}{{\mhyphen\catname{Alg}}}
\newcommand{\Field}{{\catname{Field}}}
\newcommand{\Vect}{{\catname{Vect}}}
\newcommand{\Hilb}{{\catname{Hilb}}}
\newcommand{\Ch}{{\catname{Ch}}}

\newcommand{\Hom}{{\mathrm{Hom}}}
\newcommand{\End}{{\mathrm{End}}}
\newcommand{\Aut}{{\mathrm{Aut}}}
\newcommand{\Obj}{{\mathrm{Obj}}}
\newcommand{\op}{{\mathrm{op}}}

% Norms, inner products
\delimitershortfall=-1sp
\newcommand{\widecdot}{\, \cdot \,}
\newcommand\emptyarg{{}\cdot{}}
\DeclarePairedDelimiterX{\norm}[1]{\Vert}{\Vert}{\ifblank{#1}{\emptyarg}{#1}}
\DeclarePairedDelimiterX{\abs}[1]\vert\vert{\ifblank{#1}{\emptyarg}{#1}}
\DeclarePairedDelimiterX\inn[1]\langle\rangle{\ifblank{#1}{\emptyarg,\emptyarg}{#1}}
\DeclarePairedDelimiterX\cur[1]\{\}{\ifblank{#1}{\emptyarg,\emptyarg}{#1}}
\DeclarePairedDelimiterX\pa[1](){\ifblank{#1}{\emptyarg}{#1}}
\DeclarePairedDelimiterX\brak[1][]{\ifblank{#1}{\emptyarg}{#1}}
\DeclarePairedDelimiterX{\an}[1]\langle\rangle{\ifblank{#1}{\emptyarg}{#1}}
\DeclarePairedDelimiterX{\bra}[1]\langle\vert{\ifblank{#1}{\emptyarg}{#1}}
\DeclarePairedDelimiterX{\ket}[1]\vert\rangle{\ifblank{#1}{\emptyarg}{#1}}

% mathmode text operators
\RedeclareMathOperator{\Re}{\operatorname{Re}}		% Real part
\RedeclareMathOperator{\Im}{\operatorname{Im}}		% Imaginary part
\DeclareMathOperator{\Stab}{\mathrm{Stab}}
\DeclareMathOperator{\Orb}{\mathrm{Orb}}
\DeclareMathOperator{\Id}{\mathrm{Id}}
\DeclareMathOperator{\vspan}{\mathrm{span}}			% Vector span
\DeclareMathOperator{\tr}{\mathrm{tr}}
\DeclareMathOperator{\adj}{\mathrm{adj}}
\DeclareMathOperator{\diag}{\mathrm{diag}}
\DeclareMathOperator{\eq}{\mathrm{eq}}
\DeclareMathOperator{\coeq}{\mathrm{coeq}}
\DeclareMathOperator{\coker}{\mathrm{coker}}
\DeclareMathOperator{\dom}{\mathrm{dom}}
\DeclareMathOperator{\cod}{\mathrm{codom}}
\DeclareMathOperator{\im}{\mathrm{im}}
\DeclareMathOperator{\Dim}{\mathrm{dim}}
\DeclareMathOperator{\codim}{\mathrm{codim}}
\DeclareMathOperator{\Sym}{\mathrm{Sym}}
\DeclareMathOperator{\lcm}{\mathrm{lcm}}
\DeclareMathOperator{\Inn}{\mathrm{Inn}}
\DeclareMathOperator{\sgn}{sgn}						% sgn operator
\DeclareMathOperator{\intr}{\text{int}}             % Interior
\DeclareMathOperator{\co}{\mathrm{co}}				% dual/convex Hull
\DeclareMathOperator{\Ann}{\mathrm{Ann}}
\DeclareMathOperator{\Tor}{\mathrm{Tor}}


% misc symbols
\newcommand{\divides}{\big\lvert}
\newcommand{\grad}{\nabla}
\newcommand{\veps}{\varepsilon}						% Preferred epsilon
\newcommand{\vphi}{\varphi}
\newcommand{\del}{\partial}							% Differential/Boundary
\renewcommand{\emptyset}{\text{\O}}					% Traditional emptyset symbol
\newcommand{\tril}{\triangleleft}					% Quandle operation
\newcommand{\nabt}{\widetilde{\nabla}}				% Contravariant derivative
\newcommand{\later}{$\textcolor{red}{\blacksquare}$}% Laziness indicator

% misc
\mathchardef\mhyphen="2D							% mathomode hyphen
\renewcommand{\mod}[1]{\ (\mathrm{mod}\ #1)}
\renewcommand{\bar}[1]{\overline{#1}}				% Closure/conjugate
\renewcommand\qedsymbol{$\blacksquare$} 			% Changes default qed in proof environment
%%%%% raised chi
\DeclareRobustCommand{\rchi}{{\mathpalette\irchi\relax}}
\newcommand{\irchi}[2]{\raisebox{\depth}{$#1\chi$}}
\newcommand\concat{+\kern-1.3ex+\kern0.8ex}

% Arrows
\newcommand{\weak}{\rightharpoonup}					% Weak convergence
\newcommand{\weakstar}{\overset{*}{\rightharpoonup}}% Weak-star convergence
\newcommand{\inclusion}{\hookrightarrow}			% Inclusion/injective map
\renewcommand{\natural}{\twoheadrightarrow}				% Natural map

% Environments
\theoremstyle{plain}
\newtheorem{thm}{Theorem}[section]
%\newtheorem{lem}[thm]{Lemma}
\newtheorem{lem}{Lemma}
\newtheorem*{lems}{Lemma}
\newtheorem{cor}[thm]{Corollary}
\newtheorem{prop}{Proposition}
\newtheorem*{claim}{Claim}
\newtheorem*{cors}{Corollary}
\newtheorem*{props}{Proposition}
\newtheorem*{conj}{Conjecture}

\theoremstyle{definition}
\newtheorem{defn}{Definition}[section]
\newtheorem*{defns}{Definition}
\newtheorem{exm}{Example}[section]
\newtheorem{exer}{Exercise}[section]

\theoremstyle{remark}
\newtheorem*{rem}{Remark}

\newtheorem*{solnx}{Solution}
\newenvironment{soln}
    {\pushQED{\qed}\renewcommand{\qedsymbol}{$\Diamond$}\solnx}
    {\popQED\endsolnx}%

% Macros
\newcommand{\restr}[1]{_{\mkern 1mu \vrule height 2ex\mkern2mu #1}}
\newcommand{\Upushout}[5]{
    \begin{tikzcd}[ampersand replacement = \&]
    \&#2\ar[rd,"\iota_{#2}"]\ar[rrd,bend left,"f"]\&\&\\
    #1\ar[ur,"#4"]\ar[dr,"#5"]\&\&#2\oplus_{#1} #3\ar[r,dashed,"\vphi"]\&Z\\
    \&#3\ar[ur,"\iota_{#3}"']\ar[rru,bend right,"g"']\&\&
    \end{tikzcd}
}
\newcommand{\exactshort}[5]{
		\begin{tikzcd}[ampersand replacement = \&]
			0\ar[r]\&#1\ar[r,"#2"]\& #3 \ar[r,"#4"]\& #5 \ar[r]\&0
		\end{tikzcd}
}
\newcommand{\product}[6]{
		\begin{tikzcd}[ampersand replacement = \&]
			#1 \& #2 \ar[l,"#4"'] \\
			#3 \ar[u,"#5"] \ar[ur,"#6"']
		\end{tikzcd}
}
\newcommand{\coproduct}[6]{
		\begin{tikzcd}[ampersand replacement = \&]
			#1 \ar[r,"#4"] \ar[d,"#5"'] \& #2 \ar[dl,"#6"] \\
			#3
		\end{tikzcd}
}
%%%%%%%%%%%% PAGE FORMATTING %%%%%%%%%

\usepackage{geometry}
    \geometry{
		left=15mm,
		right=15mm,
		top=15mm,
		bottom=15mm	
		}

\usepackage{color} % to do: change to xcolor
\usepackage{listings}
\lstset{
    basicstyle=\ttfamily,columns=fullflexible,keepspaces=true
}
\usepackage{setspace}
\usepackage{setspace}
\usepackage{mdframed}
\usepackage{booktabs}
\usepackage[document]{ragged2e}
\usepackage{epsfig}
\usepackage{dynkin-diagrams}

\pagestyle{fancy}{
	\fancyhead[L]{Fall 2024}
	\fancyhead[C]{MAT1344F}
	\fancyhead[R]{John White}
  
  \fancyfoot[R]{\footnotesize Page \thepage \ of \pageref{LastPage}}
	\fancyfoot[C]{}
	}
\fancypagestyle{firststyle}{
     \fancyhead[L]{}
     \fancyhead[R]{}
     \fancyhead[C]{}
     \renewcommand{\headrulewidth}{0pt}
	\fancyfoot[R]{\footnotesize Page \thepage \ of \pageref{LastPage}}
}
\newcommand{\pmat}[4]{\begin{pmatrix} #1 & #2 \\ #3 & #4 \end{pmatrix}}
\newcommand{\A}{\mathbb{A}}
\newcommand{\B}{\mathbb{B}}
\newcommand{\fin}{``\in"}
\newcommand{\mk}[1]{\mathfrak{#1}}
\newcommand{\g}{\mk{g}}
\newcommand{\h}{\mk{h}}
\newcommand{\J}{\mc{J}}
\newcommand{\tphi}{\tilde{\phi}}
\newcommand{\pois}[2]{\{#1,#2\}}
\newcommand{\fibrate}[3]{\begin{tikzcd} #1 \ar[d, "#2"] \\ #3 \end{tikzcd}}
\DeclareMathOperator{\Perm}{Perm}
\DeclareMathOperator{\pdim}{pdim}
\DeclareMathOperator{\gldim}{gldim}
\DeclareMathOperator{\lgldim}{lgldim}
\DeclareMathOperator{\rgldim}{rgldim}
\DeclareMathOperator{\idim}{idim}
\DeclareMathOperator{\SU}{SU}
\DeclareMathOperator{\SO}{SO}
\DeclareMathOperator{\Ad}{Ad}
\DeclareMathOperator{\ad}{ad}
\DeclareMathOperator{\gr}{gr}
\DeclareMathOperator{\Sig}{Sig}
\newcommand{\Rmod}{R-\text{mod}}
\newcommand{\RMod}{R-\text{Mod}}
\newcommand{\onto}{\twoheadrightarrow}
\newcommand{\into}{\hookrightarrow}
\newcommand{\barf}{\bar{f}}
\newcommand{\dd}[2]{\frac{d#1}{d#2}}
\newcommand{\pp}[2]{\frac{\partial #1}{\partial #2}}
\newcommand{\gl}{\mk{g}\mk{l}}
\newcommand{\spew}{\Sp(E,\omega)}
\newcommand{\jew}{\mc{J}(E,\omega)}
\renewcommand{\P}{\mathbb{P}}
\renewcommand{\E}{\mathbb{E}}
\DeclareMathOperator{\Ext}{Ext}
\DeclareMathOperator{\Rank}{Rank}
\DeclareMathOperator{\Sp}{Sp}
\DeclareMathOperator{\ann}{ann}
\DeclareMathOperator{\Lag}{Lag}
\DeclareMathOperator{\Riem}{Riem}
\DeclareMathOperator{\Span}{span}
\newcommand{\exactlon}[5]{
		\begin{tikzcd}
			0\ar[r]&#1\ar[r,"#2"]& #3 \ar[r,"#4"]& #5 \ar[r]&0
		\end{tikzcd}
}

\title{MAT 1344}
\author{John White}
\date{Fall 2024}


\begin{document}

\section*{Lecture 1 - 3/5/24}

A good starting point is Newton's equation $V(q_1, \dots, q_n)$ for a particle: 
\[
m\ddot{q_i} = -\pp{V}{q_i}
\]

The first observation is that if energy is 
\[
E = \frac{m}{2}\dot{q}^2 + V(q)
\]
then $E$ is constant along solution curves (take the $t$ derivative). 

A classic physics trick is to reduce $n$th order to first order by letting higher derivatives be introduced as new variables. Introduce $p_i = m\dot{q}_i$. We have the equations
\begin{align*}
\dot{q_i} & = \frac{1}{m}p_i  & \dot{p_i} = -\pp{V}{q_i}
\end{align*}
The energy becomes ``Hamiltonian." 
\[
H(q, p) = \frac{1}{2m}\sum_{i=1}^n p_i^2 + V(q)
\]

We can write these equations from earlier quite nicely in terms of the Hamiltonian (if you know the potential you know the Hamiltonian, and vice-versa) as 
\begin{align*}
\dot{q_i} & = \pp{H}{p_i},  & \dot{p_i} = -\pp{H}{q_i}
\end{align*}

Hamilton's equation. This looks similar to $\dot{X_i} = -\pp{V}{x_i}$, the equation of a gradient flow. 

One advantage of these Hamiltonian equations is that we have \underline{\underline{lots}} of symmetry, i.e. for coordinate changes 
\begin{align*}
\tilde{q_i} = f_i(q, p), \,\,\,& \tilde{p_i} = g_i(q, p); & \tilde{H}(\tilde{q},\tilde{p}) = H(q, p)
\end{align*}
then in new coordinates, $\dot{\tilde{q_i}} = \pp{\tilde{H}}{\tilde{p_i}}, \dot{\tilde{p_i}} = -\pp{\tilde{H}}{\tilde{q_i}}$

\exm

$\tilde{p_i} = -q_i, \tilde{q_i} = p_i$

\exm

$\tilde{q_i} = q_i, \tilde{p_i} = p_i + \varphi_i(q_1, \dots, q_n)$

We think of this Hamiltonian as having a very large infintie dimensional symmetry group, in contrast to the earlier graident flow, which has a very small symmetry group. 

A \underline{Hamiltonian vector field} 
\[
X_H = \sum_{i=1}^n \left(\pp{H}{q_i}\pp{}{p_i} - \pp{H}{p_i}\pp{}{q_i}\right)
\]

If we take the exterior derivative, 
\[
dH = \sum_{i=1}^n\left(\pp{H}{q_i}dq_i + \pp{H}{p_i}dp_i\right)
\]
We can write Hamilton's equation as $\iota(X_H)\omega = - dH$, with $\omega = \sum_{i=1}^ndq_i\wedge dp_i$ where $\iota$ means contraction. 

Often we take this equation as the definition of the Hamiltonian vector field, which defines a differential equation, which defines a flow, et cetera. 

\underline{Remark:} The word ``Symplectic" was introduced by Hermann Weyl in the theory of Lie groups. 

Symplectic manifolds were introduced by Charles Ehrsmann and Paulette Libermann around 1948. 

In the 60s and 70s, Souriau, Kostant (SP?), others did more work such as trying to phrase classical mechanics in this language. Many others, such as Arnold, Thurston, who showed that there symplectic and complex manifolds are not the same. Arnold initiated a program of symplectic topology in 74(?). Weinstein, Steinberg, Guillemain...

\section*{\underline{Part 1: Symplectic Linear Algebra}}

\defn

A \underline{symplectic structure} on a finite dimensional (real for now) vector space $E$ is a bilinear form
\[
\omega:E\times E \to \R
\]
which is
\begin{enumerate}[label=(\roman*)]

\item Skew-symmetric, meaning $\omega(v, w) = \omega(w, v)$

\item Nondegenerate, meaning $\ker \omega \eqdef \{v \in E \mid \omega(v, w) = 0$ for all $w \in E\}$ is trivial. (Every vector has a friend).

\end{enumerate}

In terms of $\omega^\flat:E\to E^*$, $v \mapsto \omega(v, \cdot)$. 

Skew-symmetry means $(\omega^\flat)* = -\omega^\flat$. 

Non-degeneracy means $\ker(\omega^\flat) = 0$

\exm
\,

\begin{enumerate}

\item ``Standard symplectic structure" For $E = \R^{2n}$ with basis $e_1, \dots, e_n, f_1, \dots, f_n$, if we set 
\begin{align*}
\omega(e_i,e_j) = 0,\omega(f_i,f_j)=0,\,&\omega(e_i,f_j)=\delta_{ij}
\end{align*}

\item For $V$ any finite dimensional vector space, setting $E = V \oplus V^*$, and 
\[
\omega((v_i,\alpha_i),(v_2,\alpha_2)) = \langle\alpha_1,v_2\rangle - \langle \alpha_2,v_1\rangle
\]

\item If $V$ is any finite-dimensional \underline{complex} inner product space $h:V\times V \to \C$. 

If we take $E = V$, $\omega(v, w) = \Im(h(v, w))$ is symplectic.

\end{enumerate}

Note that these three are all actually the same example. 

\defn

A \underline{symplectomorphism} between symplectic vector spaces $(E_i, \omega_i)$, $(i = 1, 2)$ is a linear isomorphism $A: E_1\to E_2$, such that
\[
\omega_2(Av,Aw) = \omega_1(v,w)
\]
for all $v, w \in E_1$ (I.e $\omega_1 = A^*\omega_2$).

\underline{Remark:} stipulating that it is an isomorphism is a little overkill, because anything satisfying the second condition is injective. 

Symplectomorphisms of $(E,\omega)$ to itself are denoted $\Sp(E,\omega)$, and is called the 

\underline{symplectic group}.

In this sense, it is easy to see that those three examples are all symplectomorphic. 


\section*{Lecture 2 - 9/10/24}

\subsection*{\underline{Subspace of symplectic vector space}}

Let $F \subseteq E$. Define $F^\omega = \{v\in E \mid \omega(v, w) = 0$ for all $w \in F\}$, the ``$w$-orthogonal" space. 

In terms of $\ann(F) = \{\alpha\in E^* \mid \alpha(v) = 0$ for all $v \in F\}$

Note that $\omega^\flat:F^\omega \to \ann(F)$ is an isomorphism.

\prop
\,
\begin{itemize}

\item $\dim F^\omega = \dim E - \dim F$

\item $(F^\omega)^\omega = F$, $(F_1 \cap F_2)^\omega = F_1^\omega + F_2^\omega$, $(F_1 + F_2)^\omega = (F_1)^\omega \cap (F_2)^\omega$

\end{itemize}

\proof
\,
\begin{itemize}

\item $\dim F^\omega = \dim(\ann(F)) = \cdots$

\item Since elements of $F$ are orthogonal to elements of $F^\omega$, we have $F\subseteq (F^\omega)^\omega;$ by dimension count have equality. Etc

\end{itemize}

\defn

A subspace $F \subseteq E$ is called
\begin{itemize}

\item \underline{isotropic} if $F \subseteq F^\omega$

\item \underline{coisotropic} if $F^\omega \subseteq F$

\item \underline{Lagrangian} if $F^\omega = F$. 

\end{itemize}

Note $F$ is isotropic if and only if $\omega|_{F\times F} = 0$

Note: 

If $F$ is isotropic, then $\dim F \leq \frac12\dim E$

If $F$ is coisotropic, then $\dim F \geq \frac12\dim E$

In both cases, if equality holds, $F$ is Lagrangian. 

So if $F$ is Lagrangian, then $\dim F = \frac12 \dim E$.

\defn The set of Lagrangian subspaces is denoted $\Lag(E,\omega)$, called the ``\underline{Lagrangian Grassmannian}". 

\prop
\,
\begin{enumerate}[label=(\alph*)]

\item $\Lag(E, \omega) \neq \varnothing$

\item For every $M \in \Lag(E, \omega)$, there exists $L \in \Lag(E,\omega)$ with $L \cap M = \{0\}$ (i.e. $E = L \oplus M$).

\end{enumerate}

\proof
\,
\begin{enumerate}[label=(\alph*)]

\item By induction: Suppose $F \subseteq E$ is isotropic. If $F$ is Lagrangian, we're done. Otherwise, $F \subset F^\omega$ is a proper subspace. Pick $v \in F^\omega \setminus F$. Then $F' = F + \operatorname{Span}\{v\}$ is again isotropic. The process ends when it becomes Lagrangian.

\item By induction: suppose $F \subseteq E$ is isotropic, with $F\cap M = \{0\}$. If $F$ is Lagrangian, we are done. Otherwise, $F + (F^\omega \cap M) \subseteq F^\omega$ is an isotropic subspace, hence is a proper subspace of $F^\omega$. Pick $v \in F^\omega\setminus (F + (F^\omega \cap M))$; $F' = F + span\{v\}$

\claim $F' \cap M = \{0\}.$

\proof

Indeed: if $y \in F'\cap M$, write $y = x + tv$, $x \in F, t \in \R$. Then
\[
tv = y - x \in (F + M) \cap F^\omega = F + (F^\omega \cap F)
\]

\qed

\end{enumerate}

\qed

\underline{Exercise:} Given $L \in \Lag(E, \omega)$. Let $F \subseteq E$ be any complement, i.e. $E = L \oplus F$. 
\begin{enumerate}[label=(\roman*)]

\item Show that there exists a unique linear map $A:F\to L$ such that $F^\omega = \{v + Av\mid v \in F\}$

\item Show that all $F_t = \{v + tAv \mid v \in F\}$ is a complement to $L$.

\item $F_{\frac12}$ is Lagrangian

\end{enumerate}

Given $(E, \omega)$, we can choose a Lagrangian splitting $E = L \oplus M$. 

\prop The choice of splitting identifies $M\cong L^*$ and determines a symplectomorphism 
\[
E\to L\oplus L^*
\]

\proof

Every $w \in M$ defines a linear functional $\alpha_\omega \in L^*$, $\alpha_\omega(v) = \omega(v, w)$

The map $M \to L^*$, $w \mapsto \alpha_w$ is an isomorphism, using the non-degeneracy of the symplectic structure. The resulting map $E \cong L \oplus M \to L \oplus L^*$ is a symplectomorphism (by formula for sympletic structure on $L \oplus L^*$).

\qed

\prop

For every symplectic $(E, \omega)$, there exists a symplectomorphism $E \to \R^{2n}$, where $\R^{2n}$ has the standard symplectic structure. 

\proof

Choose a Lagrangian splitting $E = L \oplus L^*$. Now pick basis of $L, $ dual basis of $L^*$ to identify $E \cong \R^{2n}$

\underline{Remark:}

$\Lag(\R^2) = \RP(1) \cong S^1$

$\Lag(\R^4) = ?$

\underline{Exercise:}

Given $L \in \Lag(E,\omega)$, show that the set of all 
\begin{itemize}

\item complement to $L$ is an affine space with corresponding linear space $\Hom(E/L,L)$ (note if $L$ is lagrangian, then $E/L$ is naturally identified with $L^*$).

\item Lagrangian complements to $L$ is an affine space with corresponding linear space the self-adjoint maps $L^* \to L$.

\end{itemize}

In general, $\dim \Lag(\R^{2n}) = \frac{n(n + 1)}{2}$

\subsection*{\underline{Linear Reduction:}}

Let $(E, \omega)$ be symplectic. $F \subseteq E$ is \underline{symplectic} if $\omega|_{F\times F}$ is nondegenerate. Equivalently, $\underbrace{F\cap F^\omega}_{=\ker \omega|_{F\times F}} = \{0\}$

Note that $F$ is symplectic if and only if $F^\omega$, and $E = F \oplus F^\omega$.

In general, if $F$ is not symplectic, we can make it symplectic by quotienting by $\ker(\omega|_{F\times F}) = F\cap F^\omega$. 

\prop

For any subspace $F$, the quotient $E_F = F/(F\cap F^\omega)$ inherits a symplectic structure: 
\[
\omega_F(\pi(v), \pi(w)) = \omega(v, w)
\]
where $\pi$ is the quotient map $\pi:F\to F/(F\cap F^\omega)$

\proof

It's well defined: E.g, if $\pi(v) = 0$, then $v \in F \cap F^\omega$, so $\omega(v, w) = 0$ for all $w\in F$. 

It's non-degenerate: If $\pi(v)\in \ker(\omega_F)$, then $\omega(v, w) = 0$ for all $w \in F$, so $v \in F \cap F^\omega$, so $\pi(v) = 0$. 

Note: For $F$ coisotropic, $E_F = F/F^\omega$

\prop

For $F$ coisotropic, $L \subseteq E$ Lagrangian, the subspace $\pi(L\cap F) = L_F$ is again Lagrangian.

\proof

Clearly, $L_F$ is isotropic. To show $L_F$ is Lagrangian, count dimension: $(L_F) = (L \cap F)/(L \cap F^\omega)$.
\begin{align*}
\dim (L \cap F^\omega) & = \dim E - \dim (L \cap F^\omega)^\omega \\
& = \dim E - \dim (L + F) \\
& = \underbrace{\dim E}_{2\dim L} - \dim L - \dim F + \dim(L\cap F)\\
& = \dim L - \dim F + \dim (L \cap F)
\end{align*}

So 
\begin{align*}
\dim (L_F) & = \dim(L\cap F) - \dim(L\cap F^\omega) \\
& = \dim F - \dim L \\
\dim E_f & = \dim F - \dim F^\omega \\& = 2\dim F - \dim E \\ & = 2(\dim F - \dim L) 
\end{align*}

\qed

So, we have constructed a map $\Lag(E, \omega) \to \Lag(E_F, \omega_F), L \mapsto L_F$

\underline{Warning:} This map is not continuous!

It is discontinuous at the set of $L$'s where $L, F$ are not transverse. Away from this set, it's smooth. 

\underline{Exercise:}

Let $E = \R^4$ with standard symplectic basis. Take 
\[
F = \operatorname{span}\{e_1, e_2, f_1\}
\]
Then $F^\omega = \operatorname{span}\{e_2\}$. 

$F/F^\omega \cong \R^2 = \operatorname{span}\{e_1, f_1\}$. 

Let $L_t = \operatorname{span}\{e_1 + tf_2, e_2 + tf_1\}$. 
\begin{enumerate}[label=(\alph*)]

\item Check $L_t$ are Lagrangian

\item Compute $(L_t)_F \subseteq \R^2$ and find it's discontinuous at $t = 0$. 

\end{enumerate}

\subsection*{\underline{Compatible complex structures}}

\underline{Recall:} 

Given a complex vector space $V$, we can always regard it as a real vector space of twice the dimension. ``Multiplication by $\sqrt{-1}"$ becomes a real linear transformation $\mathcal{J} \in \Hom_{\R}(V,V)$, $\mathcal{J}^2 = -I$. 

Conversely, a real vector space with such a $\mathcal{J}$ is called a \underline{complex structure}, and we can imbue it with complex multiplication by defining
\[
(a + ib)v = av + b(\mathcal{J}v)
\]

\defn

Let $(E, \omega)$ be a symplectic vector space. A complex structure $\mathcal{J}$ (meaning $\mathcal{J}^2 = -I)$ is \underline{$\omega$-compatible} if 
\[
g(v, w) \eqdef \omega(v, \mathcal{J}w)
\]
defines an inner product. Denote by $\mathcal{J}(E, \omega) \eqdef \{\omega-$compatible complex structure $\}$.

Given $j \in \mathcal{J}(E, \omega)$, we get a complex inner product by 
\[
h(v, w) \eqdef g(v, w) + \sqrt{-1}\omega(v, w)
\]

\underline{Remark:} $\mathcal{J} \in \mathcal{J}(E,\omega)$ is a symplectomorphism: 
\begin{align*}
\omega(\mathcal{J}v, \mathcal{J}w) & = g(\mathcal{J}v, w) \\
& = g(w, \mc{J}v) \\
& = \omega(w, \mc{J}^2v) \\
& = -\omega(w, v)\\
& = \omega(v, w) \\
\end{align*}

\underline{Remark:} For $\R^{2n}$, there is a standard complex structure given by $\mc{J}(e_i) = f_i, \mc{J}(f_i) = -e_i$.

This identifies $\R^{2n} \cong \C^n$. We can come up with more complex structures by picking an $A \in \Sp(E, \omega)$ and considering $\mc{J} \mapsto A\mc{J}A^{-1}$. 

We have a map $\mc{J}(E,\omega) \to \operatorname{Riem}(E)$ (real inner products) given by $\mc{J} \mapsto g$, where $g$ is as above. 

There is a canonical left inverse $\varphi:\operatorname{Riem}(E) \to \mc{J}(E,\omega)$ as follows: 

\prop There is a canonical retraction (in the sense of topology) from $\operatorname{Riem}(E) \to \mc{J}(E, \omega)$.

\proof

Given $k \in \Riem(E)$, define $A \in \GL(E)$ by $k(v, w) = \omega(v, Aw)$. 

$A$ is not a complex structure in general, but it's skew-symmetric with respect to $h$: 
\[
A^T = -A
\]
Define $|A| = (A^TA)^\frac12 = (-A^2)^\frac12$. This commutes with $A$ by functional calculus, and define $\mc{J} = A|A|^{-1}$. This will do the job. 

\qed

\section*{Lecture 3 - 9/12/24}

Let $(E, \omega)$ be a symplectic vector space. Let $\mc{J} \in \Hom(E, E), \mc{J}^2 = -I$. $Y$ is \underline{$\omega$-compatible} if 
\[
g(v, w) \eqdef \omega(v, \mc{J}w)
\]
is a (real) inner product. 

\exm 

Let $E = \R^{2n} = \Span\{e_1, \dots, e_n, f_1, \dots, f_n\}$. Let $\mc{J}e_i = f_i, \mc{J}f_i = -e_i$. Then $g$ is the standard inner product. 

\underline{Remark:} 
\begin{itemize}

\item Note: in the definition of $\omega$-compatible, any two of $\omega, \mc{J}, g$ deetermines the third. 

\item $\mc{J} \in \Sp(E,\omega) \cap O(E, g)$

\item $h(v, w) = g(v, w) + i\omega(v, w)$ is a \underline{complex} inner product, with corresponding unitary group $U(n) = \underbrace{U(E, h)}_{\text{preserves }h} = \underbrace{\Sp(E, \omega)}_{\text{preserves }\omega} \cap \underbrace{O(E, g)}_{\text{preserves }g}$. Recall that $U(n)$ is compact and connected, and has $\pi_1 = \Z$

\end{itemize}

Let $\mc{J}(E, \omega) = \{\mc{J} \mid  \omega$-compatible $\}$

We have a map $\psi:\mc{J}(E,\omega) \to \Riem(E) \subseteq \Sym^2(E)$, which is contractible.

\thm There is a canonical retraction 
\[
\phi:\Riem(E) \to \mc{J}(E, \omega)
\]
such that $\phi \circ \psi = \Id$.

\cor $\mc{J}(E, \omega)$ is contractible
\proof

Send 
\[
\mc{J} \mapsto \varphi\left((1 - t)\psi(\mc{J}) + tg_0\right)
\] 

\qed

\proof

Let $k \in \Riem(E)$ be given. Define $A \in \Hom(E, E)$ by $k(v, w) = \omega(v, Aw)$. 

Then $A = -A^T$ (skew-adjoint with respect to $k$). Why? note that 
\begin{align*}
k(v, A^{-1}y) &= \omega(v, y) \\&= -\omega(y, v)\\&=-k(y, A^{-1}v\\&=-k(A^{-1}v, y)
\end{align*}

So $(A^{-1})^T = A^{-1}$, so $A^T = -A$.

Hence, we can define $|A| = \sqrt{A^TA} = \sqrt{-A^2}$. Put $\mc{J} = A|A|^{-1}$. Then $\mc{J}^2 = A|A|^{-1}A|A|^{-1} = A^2|A|^{-2} = -I$. 

Now we check that it defines an inner product: 
\begin{align*}
g(v, w) & = \omega(v, \mc{J}w) \\
& = \omega(v, A|A|^{-1}\omega) \\
& = k(v, |A|^{-1}w) \\
& = k(|A|^{-\frac12}v, |A|^{-\frac12}w)
\end{align*}

is an inner product. 

\qed

If $k$ was instead $g$ with a compatible complex structure, then $A$ must be that complex structure on the nose. 

Now, $\Sp(E, \omega)$ acts on $\mc{J}(E, \omega)$ by 
\[
A \cdot \mc{J} = A\mc{J}A^{-1}
\]

\prop The action of $\Sp(E, \omega)$ on $\mc{J}(E, \omega)$ is transitive. That is, for any $\mc{J}_1, \mc{J}_2 \in \mc{J}(E, \omega)$, there is an $A \in \Sp(E,\omega)$ with $\mc{J}_1 = A\mc{J}_2A^{-1}$. It has stabilizers at $\mc{J}\in\mc{J}(E,\omega)$ the unitary group $U(E)$, with respect to $\mc{J}$. I.e.:
\[
\mc{J}(E, \omega) = \Sp(E,\omega)/U(E)
\]

\cor

$\Sp(E,\omega)$ is connected. 

\proof

$\mc{J}(E,\omega)$ is connected, and $U(E)$ is connected, so $\spew$ is connected. 

\qed

\proof

Given $\mc{J}, \mc{J}' \in \jew$, let $e_1, \dots, e_n$ be an orthonormal basis for $E$ a complex inner product space (with respect to $\mc{J}$).

Then $e_1, \dots, e_n, f_1 = \mc{J}e_1, \dots, f_n = \mc{J}e_n$ is a symplectic basis. Similarly, define $e_1', \dots, e_n', f_1', \dots, f_n'$ by $Ae_i = e_i', Af_i = f_i'$. This defines a symplectic transformation $A \in \spew$, with $A\J A^{-1} = \J'$.

\qed

\subsection*{More on $\spew$}

\prop

$\spew$ is a connected Lie group of dimension $2n^2 + n$, where $\dim E = 2n$

\proof

By Cartan's theorem, every closed (in the sense of topology) subgroup of a Lie group is a Lie group. 

This applies to $\spew \subseteq \GL(E)$ (invertible transformations). For connected, see above. 

To get dimension, consider action of $\GL(E)\supseteq \mc{U} = \{$symplectic forms$\}$ on $\bigwedge^2 E^*$ the space of skew-symmetric bilinear forms.  This action is transitive, with stabilizer at $\omega \in \mc{U}$ given by $\spew$. 

Hence $\mc{U} = \GL(E)/\spew$, and using this we can count dimensions:
\[
\underbrace{\dim \mc{U}}_{\dim = \dim\bigvee^2 E^* = {2n\choose 2}} = \underbrace{\dim \GL(E)}_{\dim = (2n)^2} - \dim \spew
\]

So $\dim \spew = (2n)^2 - \frac{2n(2n-1)}{2} = 2n^2 + n$ 

\section*{Lecture 4 - 9/17/24}

\subsection*{\underline{Geometry of $\spew$ and $\Lag(E,\omega)$}}

So far, we know $\spew$ is a Lie group. It is also connected, with dimension $2n^2 + n$, where $\dim E = 2n$. 

For any $\mc{J}\in\jew$, we have a real inner product $g(v, w) = \omega(v, \mc{J}w)$, and a complex one given by $h = g + \sqrt{-1}\omega$. 

$U(E) \subseteq \spew$, $\jew=\spew/U(E)$

(Enough to consider $E = \R^{2n}$, $\mc{J}e_i = f_i, \mc{J}f_i = -e_i$). 

Let $()^T$ be the transpose with respect to the metric $g$. In other words, $g(Av, w) = g(v, A^Tw)$. 

\prop

$A \in \GL(E)$ is \underline{symplectic} if and only if $A^T = \mc{J}A^{-1}\mc{J}^{-1}$. 

\proof

$A\in\spew$ if and only if for all $v, w, \omega(Av,Aw) = \omega(v, w)$. Note that $g(\mc{J}v,w) = \omega(v, w)$. So for all $v, w$, 
\begin{align*}
g(\mc{J}Av, Aw) & = g(\mc{J}v, w) \\
& = g(A^T\mc{J} Av, w) 
\end{align*}
so $A^T\mc{J}A = \mc{J}$ .

\qed

Consequence: if $A\in\spew$, then $\det(A) = 1$. This follows from 
\begin{align*}
\det(A) & = \det(A^T) \\
& = \det(\mc{J}A^{-1}\mc{J}^{-1}) \\
& = \det(A)^{-1} 
\end{align*}
So $\det(A)^2 = 1$, and since $\spew$ is connected, $\det(A) = 1$. 

Now, in $\R^{2n}$, if we let $A = \pmat{a}{b}{c}{d}$, $\mc{J} = \pmat{0}{I}{-I}{0}$.
\[
A \in \spew \iff \begin{cases} a^Tc = c^Ta \\ b^Td = d^Tb \\ a^Td - b^Tc = I \\ \end{cases}
\]

Note: $\spew(\R^2, \omega) = \SL(2,\R)$, the 2x2 matrices of determinant 1. 

Another consequence is: $A$ is symplectic implies that $A^T$ is symplectic. This means we can use \underline{polar decomposition}. 

Recall: $A \in \GL(m, \R)$ has a polar decomposition $A = U|A|$, where $|A|$ is positive definite, $|A| = \sqrt{A^TA}$, and $U \in O(n)$. 

Can write $|A| = \exp(\xi)$, with $\xi^T = \xi$. 

Thus, $\GL(n,\R) = O(n)\times\{\xi \mid \xi^T = \xi\}$ as a manifold. 

For any $G \subseteq \GL(m,\R)$, which is invariant under $A \mapsto A^T$, we get a polar decomposition 
\[
G = K\times p
\]
where $K = G\cap O(m)$, $p = \mk{g} \cap \{\xi \mid \xi^T = \xi \}$
where $\mk{g} = \{\xi \mid \exp(t\xi) \in G \text{ for all }t\}$

In particular, $G = \spew \cong \Sp(2n, \omega)$, we get $K = \Sp(2n,\omega) \cap O(2n) = U(n)$ and $p = \{\xi \in \Sp(2n, \omega) \mid \xi = \xi^T\}$. 

So we get that $\spew = U(e)\times p$. 

Upshot: $\spew$ deformation retracts onto its maximal compact subgroup $U(E)$. 

\cor

There is a canonical isomorphism 
\[
\mu: \pi_1(\spew)\to\Z
\]

\proof

$\pi_1(\spew) \cong \pi_1(U(E))$. Now, $\det:U(E) \to \pi_1(U(1))$, and this map is an isomorphism. 

This is the most primitive version of a ``Maslov Index." It is a ``Maslov index of loop of symplectomorphisms. 

\underline{Exercise:} If $A, B:S^1 \to \spew$, then $\mu(AB) = \mu(A) + \mu(B)$. 

\underline{Remark:} We discussed the noncompact group $S_p(E, \omega) = \Sp(2n,\R)$. There is a compact ``symplectic group" denoted $\Sp(n)$. Both are ``real forms" of the complex symplectic group $\Sp(2n, \C)$. 
\[
\Sp(2n, \R) \subseteq \Sp(2n,\C) \supseteq \Sp(n)
\]
but $\Sp(2n,\R)\neq\Sp(n)$. We have a similar situation for $\SL(n,\C)$: 
\[
\SL(n,\R) \subseteq \SL(n,\C) \supseteq \SU(n)
\]
These two on the left and right have the same complexification.
\[
\R^* \subseteq \C^* \supseteq U(1)
\]

Recall: The Lagrangian Grassmannian, $\Lag(E) = \{L \subseteq E \mid L^\omega = L \}$. $\Lag(E) \subseteq GR_n(E) = \{n$-dimensional subspaces of $E\}$, so it is a topological space in this way. 

Recall that $Gr_k(E)$ can be seen as a manifold in 2 ways:
\begin{enumerate}[label=(\roman*)]

\item View it as a homogeneous space

\item Construct charts

\end{enumerate}

\begin{enumerate}[label=(\roman*)]

\item Pick any $G \subseteq \GL(E)$ such that $G$ acts transitively on $Gr_k(E)$ (e.g. $G = \GL(E), G = O(E)$ for some inner product, $G = \SO(E)$). 

Let $H \subseteq G$ be a stabilizer of some \underline{fixed} $k$-dim subspace. So $Gr_k(E) = G/H$

\item For any subspace $M \subseteq E$ of codimension $k$, the set  $\{L \in Gr_k(E) \mid E = L \oplus M \}$ is canonically an affine space. It is isomorphic to $\{j:E/M\to E \mid \pi\circ j = \Id\}$, $\pi: E\to E/M$, an affine space under $\Hom(E/M,E)$

The punchline is that for any fixed $L$, we get a vector space, and we use this as a chart. 


\end{enumerate}



Now, we want to do the same thing with $\Lag(E)$. 

\prop

The group $U(E)$ acts transitively on $\Lag(E)$ with stabilizers at given $L \in \Lag(E)$ equal to $O(L)$. $U(E)$ is a Lie group, so $\Lag(E) = U(E)/O(L)$ is a manifold of dimension $\frac{n(n+1)}{2}$

\proof

Let $h(v, w) = g(v, w) + \sqrt{-1}\omega(v,w)$. 

Note: On any $L \in \Lag(E)$, get $h|_{L\times L} = g|_{L\times L}$. A $g$-orthonormal basis $e_1, \dot, e_n$ of $L$ is an $h$-orthonormal basis of $E$, given symplectic basis $e_1, \dots, e_n, f_i, \dots, f_n$, where $f_i = \mc{J}e_i$. 

Given another $L'$, choose $g$-orthonormal basis $e_1',\dots,e_n'$ of $L'$. It's $h$-orthonormal basis of $E$. 

The transformation taking $e_1, \dots, e_n$ to $e_1', \dots, e_n'$ is in $U(E) \subseteq \spew$ taking $L$ to $L'$. The stabilizers of $L$ are transformations for which $L = L'$, so they're in $O(L)$. 
Then
\begin{align*}
\dim\Lag(E) & = \dim U(n) - \dim O(n) \\
& = n^2 - \frac{n(n-1)}{2} \\
&= \frac{n(n+1)}{2}
\end{align*}

\qed

Alternatively, pick $\mc{J}\in\jew$, then $\Lag(E) = \Sp(E) / \Sp(E)_L$ (?)

We get again a version of the Maslov index by Arnold. 

The map $\det^2:U(n) \to U(1), A \mapsto (\det(A))^2$ descends to a map $\Lag(\R^{2n}) \to U(1)$, hence gives a map on fundamental groups
\[
\mu:\pi_1(\Lag(E)) \to \pi_1(U(1)) = \Z
\]

\prop(Arnold)

This map is again an isomorphism

\proof

\qed

This is the maslov index of loop of Lagrangian subspaces.

Special Case $n = 1$

$\Lag(\R^2) = U(1)/O(1)$. $O(1) = \{\pm1\}$, so this is a circle under polar identifications, so we get $\RP^1$, which is again $S^1$. 

Given $M \in \Lag(E)$, let $\Lag(E; M) = \{L\in\Lag(E) \mid E = L\oplus M \}$

\prop $\Lag(E;M)$ is canonically an affine space, with corresponding linear spaces $\Sym^2(M) = \{$ symmetric bilinear forms on $M^*\} \cong$ self adjoint maps $M^*\to M$. 

\proof

Let $\pi:E\to M^*$ where $\pi(E) = $ restriction of $\omega^\flat(v) \in E^*$ to $M$. 

$\pi(v)(w) = \omega(v,w)$ for $w\in M$. 

This projection map has kernel $M\subseteq E$ since $M$ is Lagrangian, so gives isomorphisms $E/M\to M^*$. 

$\Lag(E;M) = \{L \in \Lag(E) \mid L \oplus M = E\}\cong \{j:E/M \to E \mid j(M^*)$ is isotropic $, \pi\circ j= \Id\}$. 

Given any such $j$, any other splitting $j'$ is of the form $j'(m) = j(m) + \psi(m)$ for some $\psi:E/M = M^*\to M$.

For all $\mu_1,\mu_2\in M^*$, 
\begin{align*}
0 & = \omega(j'(\mu_1),j'(\mu_2)) \\
& = \omega(j(\mu_1) + \psi(\mu_1), j(\mu_2) + \psi(mu_2) \\
& = \underbrace{\omega(j(\mu_1), j(\mu_2))}_{=0\text{ since $j$ isotropic }} + \underbrace{\omega(\psi(\mu_1),\psi(\mu_2))}_{=0\text{ since $M$ isotropic}} \\
& + \omega(j(\mu_1), \psi(\mu_2)) + \omega(\psi(\mu_1),j(\mu_2)) \\
& = \langle \mu_1, \psi(\mu_2)\rangle - \langle \mu_2, \psi(\mu_1) \rangle \\
\end{align*}

So $\psi$ is self-adjoint, $\beta(\mu_1,\mu_2) = \langle \mu, \psi \mu \rangle$. 

Again we see $\dim = \frac{n(n + 1)}{2}$.

Down-to-earth version

Let $E = \R^{2n}$, $M = 0\oplus\R^n = \Span\{f_1, \dots, f_n\}$. 

$\R^{2n} = L\oplus M$ means $L$ is graph of linear map $S: \R^n\to\R^n$.

$L$ has basis 
\[
g_i = e_i + \sum_{j=1}^n S_{ij}f_j
\]
is Lagrangian if and only if for all $i, k$, 
\begin{align*}
0 = \omega(g_i, g_k) & = \omega(e_i + \sum S_{ij}f_j, e_k + S_{kl}f_l \\
& = \cdots \\
& = S_{ki} - S_{ik}
\end{align*}

\section*{Lecture 5 - 9/19/24}

\subsection*{\underline{Maslov Indices}}

Let $(E,\omega)$ be a symplectic vector space, $\dim E = 2n$. We consider $\Lag(E)$, the Lagrangian Grassmannian, the set of all Lagrangian subspaces. We know this is homeomorphic to $U(n)/O(n)$, once you choose a symplectic basis. It is a manifold, of dimension $frac{n(n+1)}{2}$. 

We have $\pi_1(\Lag(E)) \cong \pi_1(U(n)/O(n))$. The function $\det^2$ descends to a function on this space, and gives a morphism from $\pi_1(U(n)/O(n))$ to $\pi_1(U(1)) \cong \Z$. 

So we have a canonical $\mu:\pi_1(\Lag(E)) \to \Z$ is called the \underline{Maslov Index}. It is somewhat akin to winding number. 

We want to generalize to \underline{paths} of Lagrangians. Fix a Lagrangian subspace $M\in\Lag(E)$, and define
\[
\Lag(E,M) = \{L\in \Lag(E) \mid L \cap M = 0 \}
\]
This is canonically an affine space, and so is contractible. Let $\sum_M = \Lag(E) \setminus \Lag(E,M) = \{L \mid L \cap M \neq 0\}$. This is some kind of singular space. 

Consider a path $L:[a, b] \to \Lag(E), t \mapsto L(t)$ with $L(a), L(b) \not\in\sum_M$

Define the Maslov Index $[L:M]\eqdef$ Maslov index of \underline{loop} obtained by concatenating $L(t)$ with a path in $\Lag(E,M)$ to make a loop. The contractibility of this space means the choice of path doesn't matter. 

This is Maslov's ``original" index as intersection number with the sincular cycle $\sum_M$. 

More generally, we want to find $[L_1:L_2]$ for two arbitrary Lagrangian paths, a kind of signed number of nonzero intersections $L_1(t) \cap L_2(t)$ (remember that $L_i(t)$ is a vector space!). 

Let $L_1, L_2 \in \Lag(E,M)$ related by some $\beta_{12} \in \Sym^2(M)$ (symmetric bilinear form on $M^* \cong L_1$. Recall from last time we have $\beta_{12} \cdot L_1 = L_2$, $\beta_{21} \cdot L_2 = L_1$, $\beta_{12} = \beta_{21}$. To this setting we can attach an invariant. 

Does there exist a symplectomorphism $A$ such that $(L_1, L_2, M) \mapsto_A (L_1, L_2', M)$, with $L_2'$ a Lagrangian transverse to all the others. 

\subsection*{\underline{Signature of symmetric bilinear form}}

For a symmetric matrix $B$, we say the Signature, $\Sig(B)$, is the number of positive eigenvalues minus the number of negative eigenvalues. 

For a symmetric bilinear form $\beta$, $\Sig(\beta) = \Sig(B)$, for $B$ the matrix of $\beta$ in terms of a basis (which does not affect the eigenvalues). 

The number $\Sig(\beta_{12})$ depends on $M$. 

\prop

If $L_1,L_2,L_3 \in \Lag(E;M)$. Then the number $s(L_1,L_2,L_3) = \Sig(\beta_{21}) + \Sig(\beta_{32}) + \Sig(\beta_{13})$ is actually independent of $M$. This is also called a Maslov index. 

\proof

\prop
\,

\begin{enumerate}

\item $s(L_1,L_2,L_3) = s(L_2,L_3,L_1)$

\item $S(L_1,L_2,L_3) = -S(L_2,L_1,L_3)$

\item Cocycle identity: for all Lagrangians $L_1,L_2,L_3,L_4$, 
\[
S(L_2,L_3,L_4) - s(L_1,L_3,L_4) + s(L_1,L_2,L_4) - s(L_1,L_2,L_3) = 0
\]

\item If $M(t)$ is always transverse to $L_1,L_2$, then $s(L_1,L_2, M)$ doesn't depend on $t$. 

\item Up to symplectomorphism, $L_1,L_2,L_3$ is uniquely determined by $\dim(L_1\cap L_2), \dim(L_2\cap L_3), \dim(L_1\cap L_3), \dim(L_1\cap L_2\cap L_3)$, s$(L_1,L_2,L_3)$

\end{enumerate}

\prop

Suppose $[a,b]\to\Lag(E), t \mapsto L_i(t)$, $i = 1, 2$ are paths, and that there exists some $M\in\Lag(E)$ such that $L_i(t) \cap M = 0$ for all $i = 1, 2, t \in [a,b]$. Then
\[
[L_1;L_2] = \frac12(s(L_1(a),L_2(a),M) - s(L_1(b),L_2(b),M))
\]

\proof

Suppose $M'$ is another choice. First term changes by 

\begin{align*}s(L_1(a),L_2(a),M') - s(L_1(a),L_2(a),M')  &= s(L_1(a),M,M') - s(L_2(a),M,M') \\
& = s(L_1(b),M,M') - s(L_2(b),M,M') \\
\end{align*}
This is the change in the second term, so they cancel out.

General definition: 

Consider a partition $a = t_0 < t_1 < \dots < t_k = b$ such that for all $[t_{j-1},t_j] \in M_j\in\Lag(E)$ with $L_i(t) \cap M_j = 0$ for all $t \in [t_{j-1},t_j]$

Then 
\[
[L_1;L_2] = \frac12\sum_{j=1}^k\left(s(L_1(t_{j-1}),L_2(t_{j-1}),M_j) - s(L_1(t_j),L_2(t_j),M_j)\right)
\]

For $A\in\spew$, we have $\operatorname{Graph}(A) = \{(Av,v)\}\subseteq E\times \bar{E}$ (where $\bar{E}$ has the same symplectic form but with an opposite sign) is Lagrangian. 

Define, for any path $A(t)$, $\mu(A) = [\operatorname{Graph}(A), \bigtriangleup]$

\section*{Lecture 6, 9/24/24}

\section*{\underline{Part 2: Symplectic Manifolds}}

Recall that the Lie derivative of a vector field, $\ms{L}_X$, is given by $\dd{}{t}|_{t=0}(F_{-t})^*$, where $F_t$ is a flow along $X$. Note $X(f) = \ms{L}_Xf = \dd{}{t}|_{t=0}(F_{-t})^*f$

Differential of a map $F:M_1\to M_2$ is a map $TF:TM_1\to TM_2$.

For $f:M\to R$, $Tf:TM\to T\R$, while $df\in\Omega^1(M)$.

We will introduce symplectic manifolds by analogy to a complex manifold. 

\subsection*{Complex manifolds}

A complex manifold comes with a family of linear transformations $\mc{J}_m:T_mM\to T_mM$, with $\mc{J}_m^2 = -\Id_{T_mM}$, which depends smoothly on $m\in M$. 

This is typically called an ``almost complex structure".

A complex manifold is the same as a real manifold, but charts go to $\C^n$, and we want transition functions to be holomorphic. Any complex manifold has an almost complex structure on its tangent spaces, but a manifold with an almost complex structure is not necessarily a complex manifold. 

It is some kind of \underline{integrability condition} on $\mc{J} = \{\mc{J}_m\}$, and if this condition vanishes, then the almost complex structure comes from an honest complex manifold. 

\subsection*{Symplectic manifolds}

A symplectic manifold is equipped with a family of functions $\omega_m:T_mM\times T_mM \to \R$ which is symplectic, depending smoothly on $m$. This is called an ``almost symplectic structure." There is again an integrability condition we can impose. We want to stipulate that $\omega_m$ arises from an $\omega = \{\omega_m\}\in \Omega^2M$. The integrability condition is that $d\omega = 0$. 

\defn

A \underline{symplectic structure} on a manifold $M$ is a non-degenerate 2-form $\omega\in\Omega^2(M)$ with $d\omega = 0$.

Non-degenerate just means that each $\omega|_{T_mM\times T_mM}$ is non-degenerate. 

\prop

Any symplectic 2-form $\Omega$, for $\dim M = 2n$, is non-degenerate if and only if $\underbrace{\omega^n}_{=\underbrace{\omega\wedge\cdots\wedge\omega}_{n\text{ times}}}\neq0$ everywhere. 

\proof

Check at $m \in M$. 

In one direction,  suppose $(\omega_m)^n\neq0$. We want to show $\ker\omega_m = 0$. Since $(\omega_m)^n \neq0$, we have $\iota_V(\omega_m^n)$, where we define $\iota_v:\Omega^k(M)\to\Omega^{k-1}(M)$ by $\iota_V(\alpha) = \alpha(V, \cdots)$. 

Anyways, $\iota_V(\omega^n)$ is nonzero for all $v \in T_m$. 

But $\iota_V(\underbrace{\omega_m\wedge\cdots\wedge\omega_m}_{n}) = n(\iota_v\omega_m)\omega_m^{n-1}$, so $\iota_V\omega_m\neq0$. 

In the other direction, suppose $\ker(\omega_m) = 0$, so $\omega_m$ is symplectic. Let $e_1, \dots, e_n, f_1, \dots, f_n$ be a symplectic basis for $T_mM$ with respect to $\omega_m$.

Consider 
\begin{align*}
\iota_{e_n}\iota_{e_{n-1}}\cdots\iota_{e_1}(\omega_m^n) & = n(\iota_{e_n}\cdots\iota_{e_2})((\iota_{e_1}\omega_m)\wedge \omega_m^{n-1}) \\
& = n(n-1)(\iota_{e_n}\cdots\iota_{e_3})((\iota_{e_2}\omega_m)(\iota_{e_1}\omega_m)\omega_m^{n-2} \\
& \vdots \\
& = n!(\iota_{e_n}\omega_m)\wedge\cdots\wedge(\iota_{e_1}\omega_m)
\end{align*}

So $\iota_{f_n}\cdots\iota_{e_1}(\omega_m^n) = \pm n! \neq 0$

\defn

Let $(M,\omega)$ be an (almost) symplectic manifold. The volume form 
\[
\bigwedge = \frac{\omega^n}{n!} = (\exp(\omega))_{\dim M}
\]
is called the \underline{Liousville volume form} on $M$. 

\defn

Let $(M,\omega)$ be a symplectic manifold.
\begin{enumerate}[label=(\alph*)]

\item A \underline{symplectomorphism} is a diffeomorphism $F \in \operatorname{Diff}(M)$ preserving $\omega$, i.e. $F^*\omega = \omega$. The group of symplectomorphisms of $(M,\omega)$ is denoted $\Diff(M,\omega)$

\item A \underline{symplectic vector field} on $M$ is a vector field $X \in \ms{X}(M)$ preserving $\omega$, i.e. $\ms{L}_X\omega = 0$.

The Lie algebra of symplectic vector fields is denoted $\ms{X}(M,\omega)$, i.e. the Local flow is symplectic. 

\end{enumerate}

\defn

Let $(M,\omega)$ be a symplectic manifold, let $H\in C^\oo(M)$. 

The \underline{Hamiltonian vector field} $X_H \in \ms{X}(M)$ is the unique vector field such that 
\[
\iota(X_H)\omega = -dH
\]
The space of Hamiltonian vector fields is denoted $\ms{X}_{Ham}(M,\omega)$

\prop Indeed, $\ms{X}_{Ham}(M,\omega) \subseteq \ms{X}(M,\omega)$.

\proof

Let $X = X_H$ be Hamiltonian. We check 
\begin{align*}
\ms{L}_X\omega & = (d\iota_X + \iota_Xd)\omega \\
& = d\iota_X\omega \\ 
& = -ddH \\
& = 0\\
\end{align*}
by the Cartan Formula

\qed

It turns out that $\omega$ is symplectic if and only if $\iota_X\omega$ is closed. 

$X$ is Hamiltonian if and only if $\iota_X$ is exact. 

\underline{Basic examples:} 

Open subsets $U \subseteq \R^n$. Let $e_1, \dots, e_n, f_1, \dots, f_n$ be the standard symplectic basis. 

Let $q_1, p_1, q_2, p_2, \dots$ be corresponding coordinates. Take \[
\omega = \sum_{j=1}^ndq_i\wedge dp_i
\]
Liousville Volume form: $\bigwedge= \frac{1}{n!}\omega\wedge\cdots\wedge\omega$, 
\begin{align*}
\bigwedge & =\frac{1}{n!}(dq_1\wedge dp_1 + \cdots)\wedge(dq_1 + dp_1 + \cdots)\wedge \cdots \\
& = dq_1 \wedge dp_1 \wedge dq_2 \wedge dp_2 \wedge\cdots\wedge dq_n \wedge dp_n
\end{align*}
which is the standard volume form. 

Given $H \in C^\oo(U)$, $dH = \sum(\pp{H}{q_i}dq_i + \pp{H}{p_i}dp_i)$. 

$\iota(X_H)\omega = -dH$ implies $X_H = \sum(\pp{H}{q_j}\pp{}{p_j} - \pp{H}{p_j}\pp{}{q_j})$, with corresponding ODE $\dot{q_j} = \pp{H}{p_j}, \dot{p_j} = -\pp{H}{q_j}$, which are Hamilton's equations. 

\underline{Example: Cotangent bundles}

Let $M = T^*Q$ (dual of $TQ$), with $Q$ any manifold. Let $\pi:T^*Q\to Q$ be the projection. 

There's a distinguished 1-form $\theta \in \Omega^1(T^*Q)$

\defn

The \underline{canonical $1$-form} $\theta \in \Omega^1(T^*Q)$ is defined in terms of its contractions with $v \in T_{\mu}(T^*Q)$, $\mu \in T^*Q$, where
\[
\langle \theta_\mu, v\rangle = \langle \underbrace{\mu}_{\in T_{\pi(\mu)}^*Q}, \underbrace{T\pi(v)}_{\in T_{\pi(\mu)}Q} \rangle
\]

Another perspective: 

$T\pi:T_\mu(T^*Q) \to T_{\pi(\mu)}Q$ has a dual map $(T_\mu\pi)^*:\underbrace{T_{\pi(\mu)}^*Q}_{\ni\mu}\to T_{\mu}^*(T^*Q)$. Then $(T_\mu\pi)^*(\mu) = \theta_|mu$

Another perspective:

For $\alpha\in\Omega^1(Q)$, let $\sigma_\alpha:Q\to T^*Q$ be the corresponding section. 

\prop

$\theta\in\Omega^1(T^*Q)$ is the unique $1$-form such that for all $\alpha\in\Omega^1(Q)$, $\sigma_\alpha^*\theta = \alpha$. 

\proof

Let $\omega\in T_qQ$, $q \in Q$. Let $\mu = \sigma_\alpha(q) = \alpha|_q$. Then
\begin{align*}
\langle\sigma_\alpha^*\theta|_q, \omega \rangle & = \langle \theta_\mu, (T_q\sigma_\alpha)(\omega)\rangle \\
& = \langle \mu, (T_\mu\pi)(T_q\sigma_\alpha)(\omega)\rangle \\
& = \langle \mu, \omega \rangle \\
& = \langle \alpha_q, \omega \rangle
\end{align*}

\qed

In coordinates: Let $q_1, \dots, q_n$ be local coordinates on $U \subseteq Q$. Then $dq_1, \dots, dq_n \in \Gamma(T^*Q|_\mu)$ are (pointwise) basis of $1$-forms on $U$. 

This gives a basis of $T_q^*Q$, all $q$. Let $p_1, \dots, p_n$ be fiber coordinates, with cotangent coordinates $q_1, \dots, q_n, p_1, \dots, p_n$. 

\lem

In cotangent coordinates, 
\[
\theta =\sum_{i=1}^np_idq_i
\]

\proof

Let $\alpha = \sum_{i=1}^n \alpha_i(q_1,\dots,q_n)dq_i \in \Omega^1(Q)$. 

Then 
\[
\sigma_\alpha:U\to T^*Q|_U, (q_1, \dots, q_n) \mapsto (\alpha_1(q_1, \dots, q_n), \alpha_2(q_1,\dots,q_n),\cdots)
\]
i.e. $\sigma_\alpha^*p_i = \alpha_i, \sigma_\alpha^*q_i = q_i$.
\[
\sigma_\alpha^*(\sum p_idq_i) = \sum \alpha_idq_i = \alpha
\]
So $\theta = \sum p_idq_i$

\qed

\thm

The 2-form
\[
\omega = -d\theta \in \Omega^2(T^*Q)
\]
is symplectic. 

This is the \underline{canonical symplectic form} on $T^*Q$. 

\proof

In local coordinates, 
\begin{align*}
-d\theta & = -d\left(\sum p_idq_i\right) \\
& = \sum dq_i\wedge dp_i
\end{align*}

\qed

We'll describe symplectomorphisms etc. of $T^*Q$, taking into account $\pi:T^*Q\to Q$ is a fibration.

Terminology: For any surjective submersion $\pi:P\to Q$, we say $F\in\Diff(P)$ ``\underline{lifts}" $f \in \Diff(Q)$ if $\pi\circ F = f\circ \pi$. Denote by $\Diff(P,\pi)$ the ``fibration-preserving" diffeomorphisms.

We'll find all of $\Diff(T^*Q,\omega) \cap \Diff(T^*Q,\omega)$

\underline{Example:} If $P = TQ$, $f \in \Diff(Q)$, then $f_T = Tf \in \Diff(TQ)$ is a lift.

If $P = T^*Q$, $f \in \Diff(Q)$, $f_{T^*} = (Tf^{-1})^*\in\Diff(T^*Q,\pi)$

The upshot is that all $f_{T^*}$ are symplectic. 

Other lifts: given $\alpha\in \Omega^1(Q)$ we get $G_\alpha\in\Diff(T^*Q,\pi)$ by adding $\alpha$ fiberwise. 

We'll see that $G_\alpha$ is symplectic if and only if $d\alpha = 0$. 

\section*{Lecture 7, 9/26/24}

Let $\pi:T^*Q\to Q$ be the projection of the cotangent bundled, and let $\theta \in \Omega^1(T^*Q)$ be the canonical 1-form, characterized by $\sigma_\alpha^*\theta = \alpha$ (that is, if we view $\sigma$ as a map from $Q$ to $T^*Q$, then the pullback of the form $\theta$ along this map is $\alpha$. 

In coordinates, $\theta = \sum p_jdq_j$, $\omega = -2\theta$ is symplectic.

Let $\pi:P\to Q$ be a surjective submersion. Recall that $\Diff(P,\pi)$ is the set of diffeomorphisms of $P$ which preserve the fibration, i.e. $F:P\to P$ such that $F\circ\pi= \pi\circ f$ for some $f \in \Diff(Q)$

\lem

Suppose that $\pi$ has connected fibers. Then $\beta\in\Omega^*(P)$ is of the form $\beta = \pi^*(\alpha)$ if and only if for all vertical vector fields $Z \in \ms{X}(p)$, $\iota_Z(\beta) = 0$, and $\ms{L}_Z\beta=0$.

\proof

Exercise. Use coordinates adapted to submersion. 

E.g. for $\beta\in\Omega^1(P)$, $\beta = \sum f_i(q, p)dq_i + \sum g_i(q, p)dp_i$

Given $f \in \Diff(Q)$, we have the cotangent lift $f_{T^*}\in \Diff(T^*Q)$, $f_{T^*} = (Tf^{-1})^*$. 

\prop

We have $(f_{T^*})^*\theta = \theta$. In particular, $f_{T^*}\in\Diff(T^*Q,\omega)\cap \Diff(T^*Q,\pi)$. Conversely, all $F \in \Diff(T^*Q,\pi)$ with $F^*\theta = \theta$ are of this form. 

\proof

Note this diagram commutes:

\[
\begin{tikzcd}
T^*Q\ar[r, "f_{T^*}"] & T^*Q \\
Q \ar[u, "\sigma_\alpha"]\ar[r, "f"] & Q\ar[u, "\sigma_{(f^{-1})^*\alpha}"']
\end{tikzcd}
\]

For any $\alpha\in\Omega^1(Q)$, 
\begin{align*}
\sigma_\alpha^*(f_T)^*\theta &  = (f_T\circ\sigma_\alpha)^*\theta \\
& = (\sigma_{(f^{-1})^*\alpha}\circ f)^*\theta\\
& = f^*\circ ((\sigma_{(f^{-1})^*\alpha})^*\theta\\
& = f^*(f^{-1})^*\alpha \\
& = \alpha \\
\end{align*}

Now for uniqueness?

The corresponding cotangent coordinates are $\theta = \sum p_jdq_j$, and we have $F^*q_j = q_j$, since $f = \Id_Q$. Hence $F^*\theta = \theta$ gives $F^*p_j = p_j$. 

So $F = \Id_{T^*Q}$.

\qed?

Given $\alpha\in\Omega^1(Q)$, let $G_\alpha \in \Diff(T^*Q,\pi)$, $G_\alpha = \mu + \alpha|_{\pi(\mu)}$, i.e. $G_\alpha\circ\sigma_\beta = \sigma_{\alpha + \beta}$.

\prop

For $\alpha\in\Omega^1(Q)$, 
\[
G_\alpha^*\theta = \theta + \pi^*\alpha
\]

So $G_\alpha \in \Diff(T^*Q,\omega)$ if and only if $\alpha\in\Omega^1_{Cl}(Q)$, where $\Omega^1_{Cl}$ denotes the closed forms. 

Every $F \in \Diff(T^*Q,\omega)\cap\Diff(T^*Q,\pi)$ with trivial base map, then it is of the form $F = Q_\alpha$. 

\proof

For $\beta\in\Omega^1(Q)$, we have 
\begin{align*}
\sigma_\beta^*(G_\alpha^*\theta-\pi^*\alpha) & = (G_\alpha\circ\sigma_\beta)^*\theta - (\pi\circ\sigma_\beta)^*\alpha \\
& = \sigma_{\alpha+\beta}^*\theta \\
& = \alpha + \beta - \alpha \\
& = \beta
\end{align*}

For uniqueness part, suppose we have $F \in \Diff(T^*Q,\omega)\cap\Diff(T^*Q,\pi)$ induces $f = \Id_Q$. Let $\tilde{\alpha} = F^*\theta - \theta$. 

For $Z \in \ms{X}(T^*Q)$ \underline{vertical} (i.e. at every point it is in the kernel of $d\pi$), we get 
\[
\iota_Z\tilde{\alpha}-\underbrace{\iota_Z\theta}_{=0} = f^*\iota_{(F_*Z)}\theta = 0
\]

\[
\ms{L}_Z\tilde{\alpha} = \iota_Zd\tilde{\alpha} + \underbrace{d\iota_Z\tilde{\alpha}}_{=0} = \iota_Z\underbrace{(f^*d\theta - d\theta)}_{=0} = 0
\]
Hence $\tilde{\alpha} = \pi^*\alpha$. 

$F^*\theta - \theta = \pi^*\alpha$. 

So $F = G_\alpha$.

Recall for a representation of a group $G$ on a vector space $V$, $V\rtimes G$ is the group on the set $V\times G$, with 
\[
(v, g) (v', g') = (v + g'\cdot v', gg')
\]

In our case, $G = \Diff(Q)$, which acts on $\Omega^1_\alpha(Q)$. 

\thm
\[
\Diff(T^*Q,\omega)\cap\Diff(T^*Q,\pi) = \underbrace{\Omega^1_{Cl}(Q)}_{G_\alpha} \rtimes \underbrace{\Diff(Q)}_{f_{T^*}}
\]

\proof

\qed

There is an infinitesimal counterpart 
\[
\ms{X}(T^*Q,\omega) \cap \ms{X}(T^*Q,\pi) = \Omega^1_{Cl}(Q)\rtimes\ms{X}(Q)
\]
where $\ms{X}(Q)$ acts on $\Omega_{Cl}^1(Q)$ by the Lie derivative. 

In particular, for $Y \in \ms{X}(Q)$, the cotangent lift $Y_{T^*} \in \ms{X}(T^*Q,\omega)$

\prop

We have $Y_{T^*}\in\ms{X}_{Ham}(T^*Q,\omega)$ with the Hamiltonian   
\[
H = -\iota(Y_{T^*})\theta = - \langle\theta,Y_{T^*}\rangle
\]

\proof

Let $X = Y_{T^*}$. Then $dH = -d\iota_X\theta = -\ms{L}_X\theta + \iota_Xd\theta = -\iota_X\omega$

$(f_{T^*})^*\theta = \theta$, so $\ms{L}_{Y_{T^*}}\theta = 0$

So $X = X_H$

\qed

In coordinates $q_1, \dots, q_n, p_1, \dots, p_n$, $Y = \sum Y_j\pp{}{q_j}$

Note: for lifts $X$ of $Y$, $\iota_X\theta$ doesn't depend on choice of lift !

So in coordinates, we can take $X = \sum Y_j\pp{}{q_j}$

\[
\iota_X(\theta) = \sum Y_j\iota(\pp{}{q_j} + \sum p_k dq_k = \sum Y_j(q)P_j
\]
\[
H = -\sum Y_j(g)P_j
\]
\[
dH = \sum\pp{Y_j}{q_k}p_jdq_k + \sum Y_jdp_j
\]
So with $\omega = \sum dq_i \wedge dp$, 
\[
X_H = Y_{T^*} = \sum Y_j\pp{}{q_j} - \sum p_j\pp{Y_j}{q_k}\pp{}{p_k}
\]

\section*{Lecture 8, 10/1/24}

\subsection*{\underline{K\"ahler Manifolds}}

An almost complex structure on a manifold $M$ is a collection $\{\mc{J}_m\}$, $\mc{J}_m \in \End(T_mM), \mc{J}_m^2 = -I$ depending smoothly on $m$, i.e. the map $M \mapsto \End(TM) = \coprod \End(T_mM)$, $m \mapsto \mc{J}_m$, has to be smooth. 

A complex manifold has $\C^n \cong \R^{2n}$-valued charts with holomorphic transition functions. If we have a complex manifold, we have a complex structure on the tangent spaces. 

\thm (Newland-Nirenberg)

$\mc{J}$ comes from a complex structure if and only if $(N_{ij})_{\mc{J}} = 0$
\[
(N_{ij})_{\mc{J}} = [\mc{J}X,\mc{J}Y] - [X, Y] - \mc{J}[\mc{J}X, Y] + \mc{J}[\mc{J}Y, X]
\]

In this case, the complex structure is \underline{unique}

\proof

\qed

\cor If $M$ is a complex manifold, $N \subseteq M$ a real submanifold, sutch that $\mc{J}(TN) \subseteq TN$, then $N$ is a complex submanifold. 

\proof

\qed


Let $(M,\omega)$ be a symplectic manifold. $\mc{J}$ is \underline{$\omega$-compatible} if for all $m \in M$, $\mc{J}_m$ is $\omega_m$-compatible. 

Equivalently, for all vector fields $g(X, Y) = \omega(X,\mc{J}Y)$, $X, Y \in \ms{X}(M)$ is a Riemannian metric. 

Let $\mc{J}(M,\omega) =$ the set of all $\omega$-compatible almost complex structures. 

From discussion for symplectic vector spaces, the map $\mc{J}(M,\omega) \to \Riem(M)$, $\omega \mapsto g$, has a left inverse $\Riem(M) \to \mc{J}(M,\omega)$. In particular, $\mc{J}(M,\omega)\neq\varnothing$. 

Given $\mc{J}_0,\mc{J}_1\in\mc{J}(M,\omega)$, there exists a smooth homotopy $\mc{J}_t\in\mc{J}(M,\omega)$

\defn

A K\"ahler manifold is a triple $(M,\omega,\mc{J})$ where $M$ is a manifold, $\omega\in\Omega^2(M)$ symplectic, and $\mc{J}\in\mc{J}(M,\omega)$ is a \underline{complex} (i.e integrable) structure. 

\exm

$M = \C^n$, standard complex and symplectic structure, and open subsets thereof. 

\exm 

$M = \Sigma$ a 2-dimensional manifold, $\omega\in\Omega^2(\Sigma)$ the volume form (so $d\omega = 0$ automatically), any $\mc{J}\in\mc{J}(M,\omega)$ is automatically integrable. 

\prop

If $(M,\omega,\mc{J})$ is K\"ahler, and $\iota:N \into M$ is a complex submanifold (i.e. $\mc{J}(TN) \subseteq TN$), then $(N,\omega_N, \mc{J}_M)$, with $\omega_n = \iota^*\omega$, is K\"ahler. 

\proof

All we have to show $\omega_N$ is non-degenerate, i.e. that every nonzero vector $v\in TN$ has a friend $\omega\in TN$ with $\omega(v,w)\neq0$. Consider $w = \mc{J}v$. We have
\[
\omega(v,w) = \omega(v,\mc{J}v) = g(v,v) > 0
\]
\qed

A consequence is that any complex submanifold of $\C^n$ is K\"ahler and in particular symplectic. 

Consider $\CP(n) = \frac{\C^{n+1}\setminus\{0\}}{\C\setminus\{0\}}= S^{2n+1}/U(1)$

There is a complex structure such that $q:\C^{n+1}\setminus\{0\} \to \CP(n)$ is holomorphic. 

Let $S^{2n + 1} \subset \C^n$ be the unit sphere. 

Then 
\[
\begin{tikzcd}
\ar[d, "\pi"] S^{2n + 1} \ar[r, "\iota", hook] & \C^{n+1}\setminus\{0\} \\
\CP(n) & 
\end{tikzcd}
\]

\thm $\CP(n)$ has a unique symplectic structure, usually denoted $\omega_{FS}$ (for Fubini-Study), such that $\pi^*\omega_{FS} = \iota^*\omega$. With this symplectic structure it's a K\"ahler manifold. 

\proof

To show that $\iota^*\omega \in\Omega^2(S^2)$ descends under $\pi$, we have to show 
\[
\iota_z(\iota^*\omega) = 0, \ms{L}_Z(\iota^*\omega) = 0
\]
for all vertical vector fields. By Cartan, since $\iota^*\omega$ is closed, it is enough to check the 1st condition. 

Since the fibers of $\pi$ are 1-dimensional, it is enough to show $\ker(T_z\pi) \subseteq \ker(\iota^*\omega)$

For $z \in S^{2n+1}$, $T_z(\C^{n+1}\setminus\{0\}) = \C^{n+1}$. $\ker(T_zq) = \Span_{\C}(z) = \C\cdot z$.

This contains $\Span_{\R}(z) = \R\cdot z = (T_zS^{2n+1})^\perp$, hence also $\mc{J}(\R z)\subseteq T_zS^{2n+1}$

We get
\[
T_z(\C^{n+1}\setminus\{0\}) = \R z\oplus \underbrace{ \mc{J}(\R z) \oplus\overbrace{\left(T_zS^{2n+1} \cap \mc{J}(T_zS^{2n+1}\right)}^{\text{complex structure(?)}}}_{T_z(S^{2n+1})}
\]
and $\mc{J}(\R z) = \ker(T_z\pi)$, $\R z\oplus \mc{J}(\R z)$ is complex. 

...

\qed

\subsection*{\underline{Basics of symplectic manifolds}}

Let $\mk{X}(M,\omega) = \{X \mid \ms{L}_X\omega = 0\}$, and $\mk{X}_{\Hom}(M,\omega) = \{X \mid \exists H, \iota_X\omega = -dH\}$

For $H \in C^\oo(M)$, we call $X_H$ the vector field such that $\iota(X_H)\omega = -dH$.

\prop

There is an exact sequence
\[
\exactshort{\mk{X}_{Ham}(M,\omega)}{}{\mk{X}(M,\omega)}{}{H^1(M)}
\]
where $H^1(M)$ is the de Rham cohomology.

\proof

The map $\omega^\flat:TM\to T^*M$ given by $v \mapsto \iota_v\omega = \omega(v,\cdot)$, gives an isomorphism $\omega^\flat:\mk{X}(M)\to\Omega^1(M)$. 

We have $L_X\omega=0 \iff d\iota_X\omega = 0 \iff \iota_X\omega \in \Omega^1_{Cl}(M)$

We see $\omega^\flat:\mk{X}(M,\omega)\to \Omega^1_{Cl}(M), \omega^\flat:\mk{X}_{Ham}(M,\omega) \to \Omega^1_{ex}(M)$

\qed

\prop

We have $[\mk{X}(M,\omega),\mk{X}(M,\omega)]\subseteq\mk{X}_{Ham}(M,\omega)$

\proof

Exercise

\qed

\prop


In fact, for $Y_1,Y_2 \in \mk{X}(M,\omega)$
\[
[Y_1,Y_2] = X_{\omega(Y_1,Y_2)}
\]
\proof
This is because 
\begin{align*}
d\omega(Y_1,Y_2) & = d\iota_{Y_2}\iota_{Y_1}\omega \\
& = \ms{L}_{Y_2}(\iota_{Y_1}\omega) - \iota_{Y_2}d\iota_{Y_1}\omega \\
& = \iota(\ms{L}_{Y_2}Y_1)\omega + \iota_{Y_1}\ms{L}_{Y_2}\omega - \iota{Y_2}\ms{L}_{Y_1}\omega + \iota_{Y_1}\iota_{Y_2}\omega \\
& = -\iota([Y_1,Y_2])\omega
\end{align*}
Various things in the third line dissapear because of various things being symplectic.
\qed

Consider next the map $C^\oo(M) \to \mk{X}_{Ham}(M,\omega)$, $H \mapsto X_H$.

We have an exact sequence
\[
\exactshort{H^0(M)}{}{C^\oo(M)}{}{\mk{X}_{Ham}(M,\omega)}
\]
$H^0(M)$ is the locally constant functions. We want to define a Lie algebra on these things to make this a short exact sequence of Lie algebras. 

\defn

The \underline{Poisson bracket} of $F, G \in C^\oo(M)$ is defined by 
\[
\{F, G\} = \omega(X_F,X_G)
\]

It can be shown that $\{F,G\} = -\{G,F\}$.

\prop

$\{\cdot,\cdot\}$ is a Lie bracket on $C^\oo(M)$. That is, 
\begin{enumerate}

\item $\{F,G\}=-\{G,F\}$

\item $\pois{F}{\pois{G}{H}} + \pois{G}{\pois{H,F}} + \pois{H}{\pois{F,G}} = 0$
\end{enumerate}

The map $C^\oo(M) \to \mk{X}_{Ham}(M,\omega), F \mapsto X_F$, is a Lie algebra homomorphism. 

\proof\,

We start by proving (2). 

We want to show that $[X_F,X_G] = X_{\pois{F}{G}} = X_{\omega(X_F,X_G)}$, which follows from the previous proposition, $(*)$.

For (1), note first that 
\[
\pois{F}{G} = \omega(X_F,X_G) = \iota(X_G)\underbrace{\omega(X_f,\cdot)}_{-dF} = -\ms{L}_{X_G}F = \ms{L}_{X_F}G
\]

Now 
\begin{align*}
\pois{F}{\pois{G}{H}} & = \ms{L}_{X_F}\pois{G}{H} \\
& = \ms{L}_{X_F}\omega(X_G,X_F) \\
& = \underbrace{(\ms{L}_{X_F}\omega)}_{=0}(X_G,X_H) + \omega(\ms{L}_{X_F}X_G,X_H) + \omega(X_G,\ms{L}_{X_F}X_H \\
& = \omega([X_F,X_G],X_4) + \omega(X_G,[X_F,X_H]) \\
& = \omega(X_{\pois{F}{G}},X_H) + \omega(X_G, X_{\pois{F}{G}}) \\
& = \pois{\pois{F}{G}}{H} + 
\end{align*}
He erased it :(

Note: we also have 
\[
\pois{F}{G\cdot H} = \pois{F}{G}H + G\pois{F}{H}\hfill\hfill(*)
\]
since $\pois{F}{\cdot} = \ms{L}_{X_F}$. 

\underline{Remark:} A Poisson structure on a manifold $M$ is a Lie bracket $\pois{\cdot}{\cdot}$ on $C^\oo(M)$, satisfying the above equation.

(word I can't read) For classical mechanics: note for algebra $\ms{A}$, the commutator $[a,b] = ab - ba$ has a property similar to $(*)$: $[a,bc] = [a,b]c + b[a,c]$

\prop

If $\pois{F}{G} = 0$, then 
\begin{enumerate}[label=(\alph*)]

\item $G$ is constant along the integral curves of $X_F$

\item The flows of Hamiltonian vector fields $X_F, X_G$, commute. 

\end{enumerate}

\proof
\,
\begin{enumerate}[label=(\alph*)]

\item From $\ms{L}_{X_F}G = \pois{F}{G} = 0$. 

\item $[X_F,X_G] = X_{\pois{F}{G}} = 0$

\end{enumerate}

\qed

\prop

For $(M,\omega)$ compact connected, the Lie algebra map $C^\oo(M) \to \mk{X}_{Ham}(M,\omega)$ has a canonical splitting, i.e. a right inverse. 

\proof

Given $X \in \mk{X}_{Ham}(M,\omega)$, let $H\in C^\oo(M)$ be the unique function such that $X = X_H$, and $\int_MH\frac{\omega^n}{n!} = 0$

If $F, G$ are normalized in this way, we get
\[
\int_M\pois{F}{G}\frac{\omega^n}{n!} = \int_M\left(\ms{L}_{X_F}G\right)\frac{\omega^n}{n!} = \int_M\ms{L}_{X_F}\left(G\frac{\omega^n}{n!}\right) = 0
\]

In coordinates, $q_1, p_1, \dots, q_n, p_n$, 
\[
\omega = \sum dq_i \wedge dp_i
\]
\[
X_F = \pm \sum\left(\pp{F}{p_i}\pp{}{q_i} - \pp{F}{q_i}\pp{}{p_i}\right)
\]
and
\[
\pois{F}{G} = \pm\sum\left(\pp{F}{p_i}\pp{G}{q_i} - \pp{F}{q_i}\pp{G}{p_i}\right)
\]

\section*{Lecture 9, 10/3/24}

\subsection*{\underline{Lagrangian Submanifolds}}

Let $(M,\omega)$ be a symplectic manifold. 

\defn 

A submanifold $N\subseteq M$ is 
\begin{itemize}

\item \underline{Isotropic}

\item \underline{Coisotropic}

\item \underline{Lagrangian}

\end{itemize}

If, for all $m \in N$, the tangent space $T_mN\subseteq T_mM$ has that property. 

\underline{Remark:} Letting $\iota:N\into M$, we have that $N$ is isotropic if and only if $\iota^*\omega = 0$ and Lagrangian if also $\dim N = \frac12\dim M$.

\underline{Remark:} Every 1-dimensional submanifold is isotropic, and every codimension-1 submanifold is coisotropic. 

Weinstein: ``Everything is a Lagrangian submanifold"

\exm\,
\begin{enumerate}

\item Let $\pi:M=T^*Q\to Q$, and $N = \pi^{-1}(q)$ are Lagrangian for any $q \in Q$. This comes from $\omega = \sum dq_i \wedge dp_i$, looking at a fiber is setting $dq_i = 0$. 

\item $M = T^*Q, N = Q$(as zero-section). More generally, if $\alpha\in\Omega^1(Q)$, the range of $\sigma_\alpha:Q\to T^*Q$ is Lagrangian if and only if $d\alpha = 0$. 

proof 
\[
\sigma_\alpha^*\omega = -\sigma_\alpha^*d\theta = -d\sigma_\alpha^*\theta = -d\alpha
\]

\item $\R^n \subseteq \C^n \cong \R^{2n}$, or likewise $\RP^n\subseteq \CP^n$ are Lagrangian submanifolds.

More generally, if $(M,\omega,\J)$ is a K\"ahler manifold with a ``complex conjugation" $F \in \Diff(M)$ (i.e. $F\circ F = \Id_M$, $F^*\circ \J = -\J$, $F^*g = g$) then the fixed set of $F$ is a Lagrangian submanifold. 

\item More generally, if $(M,\omega)$ is a symplectic manifold with a symplectic involution $F\in\Diff(M)$ (i.e. $F\circ F = \Id_M, F^*\omega = -\omega$), then the fixed point set is a Lagrangian submanifold. 

proof: 

Fixed point set of any involution is a submanifold, $N\subseteq M$. 

This reduces the problem to tangent spaces $T_mN\subseteq T_mM$ (fixed point set of $T_mF$). 

It is enough to look at symplectic vector space $(V,\omega)$ with $A \in \operatorname{Symp}(V,\omega)$, $A^2 =I, A^*\omega = -\omega$. Write $V = V_{+1} \oplus V_{-1}$, the eigenspaces of $A$.

\underline{Claim:}

$\omega$ restricts to 0 on both $V_{\pm}$. 

proof: 

For $v, w \in V_{-1}$, 
\[
\omega(v, w) = -\omega(Av,Aw) = -\omega(-v,-w) = -\omega(v, w)
\]
since $A^*\omega=-\omega$. 

\item For $(M,\omega)$ symplectic, denote by $\bar{M}$ the same manifold but with the symplectic form negated. Then $\bigtriangleup_M \subseteq M \times \bar{M}$ is a Lagrangian submanifold. 

More generally, if $F \in \Diff(M)$, then $gr(F) = \{(F(m),m) \mid m \in M \}\subseteq M \times \bar{M}$ is a Lagrangian submanifold if and only if $F \in \Diff(M,\omega)$.

\item Let $F:Q_1\to Q_2$ be a smooth map. Then $gr(T^*F) \subseteq T^*Q_2\times\bar{T^*Q_1}$ is Lagrangian, where $gr(T^*F) = \{(\mu_2,\mu_1) \mid \mu_1 = (T_q^*(F))(\mu_2), q = \pi(\mu_1)\}$

proof

Exercise (understand the linear case)

\item Suppose $S\subseteq Q$ is a submanifold. The conormal bundle $\nu^*(Q,S)$ which we define as follows. First, let
\[
\nu(Q,S) \eqdef TQ|_S / TS
\]
$(V/W)^* = \ann(W) \subseteq V^*$, i.e. the elements of $V^*$ which kill $W$. So we define
\[
\nu^*(Q,S) = \ann(TS) \subseteq T^*Q|_S
\]
is a Lagrangian submanifold. 

proof

Near any given point of $S$, choose coordinates $q_1,\dots,q_n$ such that $S$ is given by $q_{k + 1} = \cdots = q_n = 0$. 

Let $q_1,p_1, \dots, q_n,p_n$ be the corresponding coordinates on the cotangent bundle. 

Then $\nu^*(Q,S)$ given by $q_{k+1} = \cdots = q_n = 0$

$TS = \Span\{\pp{}{q_1},\cdots\pp{}{q_n}\}$

\item

Prop: If $j:S \into Q$ is a submanifold, and $\alpha\in\Omega^1(S)$ is closed, then $N = \{\mu\in T^*Q|_S, j^*\mu = \alpha|_{\pi(\mu)}\}\subseteq T^*Q$ is again a lagrangian submanifold. 

Rem: In particular, every function $f \in C^\oo(S)$ determines a Lagrangian submanifold of $T^*Q:$ take $\alpha = df$.

Proof: Again use coordinates. This time, if $\alpha = \sum_{i=1}^k \alpha_idq_i$, equations for $N$ are $q_{k+1} = \cdots = q_n = 0$, $p_1 = \alpha_1, \dots, p_k = \alpha_k$
\[
\sum_{i=1}^n dq_i\wedge dp_i = \sum_{i,j =1}^k \pp{\alpha_i}{q_j}dq_i \wedge dq_j = d\alpha
\]
so this will be zero if and only if $\alpha$ is closed. 
\end{enumerate}

\section*{Lecture 10, 10/8/24}

\subsection*{\underline{Lagrangian submanifolds (continued)}}

\underline{Recall:} Let $(M,\omega)$ be a symplectic manifold, $F \in \Diff(M_1,M_2)$ is symplectic if and only if $gr(F)$, the graph of $F$, is a Lagrangian submanifold of $M_2\times\bar{M}_1$.

More generally: 

\defn

A \underline{Lagrangian relation} from $M_1$ to $M_2$ is a Lagrangian submanifold of $M_2\times \bar{M_1}$. 

Write $R:M_1\dashrightarrow M_2$.

\exm
\,
\begin{itemize}

\item $R = gr(F)$, $F \in \Diff(M,\omega)$

\item Any Lagrangian submanifold $N \subseteq M$ is a Lagrangian relation $pt\dashrightarrow M$, or $M\dashrightarrow pt$ 

\item For $f \in C^\oo(Q_1,Q_2)$, the cotangent relation $T^*f: T^*Q_1\dashrightarrow T^*Q_2$, that is $\underbrace{\mu_1}_{\in T_xQ_1}\sim\underbrace{\mu_2}_{\in T_{f(x)}Q_2} \iff \mu_1 = (T_xf)^*\mu_2$

\item Given another Lagrangian relation $R'\subseteq M_3\times\bar{M_2}$, i.e. $R':M_2\dashrightarrow M_3$, we can compose 
\[
R'\circ R\eqdef \{(m_3,m_1) \mid \exists m_2\in M_2: (m_3,m_2)\in R', (m_2,m_1) \in R\} \subseteq M_3\times\bar{M_1}
\]

Under suitable transversality conditions (e.g. $(R'\times R)$ needs to be transverse to $M_3\times\bigtriangleup_{M_2}\times M_1$), $R'\circ R$ is a submanifold, and is again Lagrangian.

Weinstein calls this the Symplectic ``category". 

\item Let $j:S\into Q$ be a submanifold, $\alpha\in\Omega^1_{Cl}(S)$. From $j$ we get the cotangent relation 
\[
T^*j: T^*S\dashrightarrow T^*Q
\]
From $\alpha$, get Lagrangian submanifold $N_\alpha \subseteq T^*S$, which we think of as a relation $pt\dashrightarrow T^*S$. 

By composition, we get $pt\dashrightarrow T^*Q$,i.e. a Lagrangian submanifold of $T^*Q$. 

\end{itemize}

\underline{Applications:}

\thm(Tulczyjew)

Let $E\to B$ be a vector bundle. Then there is a canonical symplectic isomorphism; 
\[
T^*E\to T^*E^*
\]

\proof

Note that $T^*E$ is not just a vector bundle over $E$, but over $E^*$, in some kind of ``double bundle."
\[
\begin{tikzcd}
T^*E \ar[r] \ar[d] & E^* \ar[d] \\
E \ar[r] & B
\end{tikzcd}
\]
we also have
\[
\begin{tikzcd}
TE \ar[r] \ar[d] & TB \ar[d] \\
E \ar[r] & B
\end{tikzcd}
\]
We consider the pullback

\[
\begin{tikzcd}
\pi^*E \ar[r] & TE \ar[r] \ar[d] & TB \ar[d] \\
& E \ar[r] & B
\end{tikzcd}
\]
dualizing, we have
\[
\begin{tikzcd}
T^*E^* \ar[r] \ar[d] & E \ar[d] \\
E^* \ar[r] & B
\end{tikzcd}
\]

So this is an isomorphism of double bundles, not just of vector bundles. 

For the proof, we'll describe it in terms of its graph. Consider $\underbrace{E^*\oplus E}_{S}$ (meaning pairs $(u,w)$ where $u$ is a tangent vector and $w$ is a cotangent vector on the same base point) as a submanifold of $\underbrace{E^*\times E}_{Q}$

We have a function $f:E^*\oplus E\to \R$ given by pairing. Take $\alpha = df$. 

Then we get a Lagrangian submanifold of $T^*Q = T^*E^*\times T^*E$

After sign change in last factor, we get Lagrangian submanifold in $T^*E^*\times\bar{T^*E}$, i.e. a Lagrangian relation $T^*E\dashrightarrow T^*E^*$ one checks that this is the graph of a symplectomorphism. 

One checks this by assuming $E = B\times V$, then $T^*E = T^*B\times T^*V, T^*E^* = T^*B\times T^*V^*, E^* = B\times V^*$. One finds: the map 
\[
T^*V=V\oplus V^* \to T^*V^* = V^*\oplus V
\]
is $(\nu,\mu)\mapsto(-\mu,\nu)$

\qed

Special case: $T^*(TQ) \cong T^*(T^*Q)$, the Legendre transform. 

\section*{\underline{Coisotropic submanifolds}}

Recall that if $(V,\omega)$ is a symplectic vector space, then $F^\omega = \{v\in V \mid \omega(v, w) = 0 $for all $w\in F\}$

Recall $\omega^\flat:V\to V^*$: we have $\omega^\flat:F^\omega\to \ann(F)$

Recall: $N\subseteq M$ is \underline{coisotropic} if $(TN)^\omega \subseteq TN$. 

In general, consider for $N \subseteq M$, 
\[
I_N = C^\oo(M)_N = \{f\in C^\oo(M) \mid f|_N = 0\}
\]
the vanishing ideal. 

\begin{itemize}

\item $X \in \ms{X}(M)$ is tangent to $N \iff X$ preserves $C^\oo(M)_N$

\item $v\in TM$ lies in $TN \iff v(f) = 0$ for all $f \in C^\oo(M)_N$.

\item $T^*M|_N \supseteq \ann(TN) = \Span\{df|_N\mid f \in C^\oo(M)_N\}$

\end{itemize}

\lem

For any submanifold $N$ of $(M,\omega)$, the bundle $TN^\omega$ is spanned by $X_f$ with $f \in C^\oo(M)_N$

\proof

$\omega^\flat$ restricts to an isomorphism $TN^\omega \to \ann(TN)$, under this isomorphism $X_f|_N \mapsto df|_N$, and $\ann(TN)$ is spanned by all $df|_N, f\in C^\oo(M)_N$.

\qed

\thm

The following are equivalent:
\begin{enumerate}

\item For all $f \in C^\oo(M)_N$, $X_f$ is tangent to $N$

\item $C^\oo(M)_N$ is closed under $\{\cdot,\cdot\}$

\item $N\subseteq M$ is a coisotropic submanifold

\end{enumerate}

\proof

For the first, it follows because $TN^\omega$ is spanned by $X_f, f \in C^\oo(M)_N$.

For the second, for all $f \in C^\oo(M)_N$, $X_f$ is tangent to $N$ means that for all $g\in C^\oo(M)_N$, $\underbrace{X_f(g)}_{=\pois{f}{g}} \in C^\oo(M)_N$

\qed

If we want to check the second condition (that is, that $C^\oo(M)_N$ is closed under $\pois{\cdot}{\cdot}$), it's enough to check $\pois{f_i}{f_j} = 0$ for any collection $f_1, \dots, f_k \in C^\oo(M)_N$ such that $df_i|_N$ span $\ann(TN)$

\prop

Let $(M,\omega)$ be symplectic, and $\pi:M\to Q$ a submersion. Then the fibers of $\pi$ are all coisotropic $\iff \pi^*C^\oo(Q) \subseteq C^\oo(M)$ 

\proof

In the left direction, suppose all the fibers are coisotropic. Given $q \in Q$, $N = \pi^{-1}(q)$ is coisotropic. For $f \in C^\oo(Q)$, then $\pi^*f - f(q) \in C^\oo(M)_N$. Given $f, g \in C^\oo(Q)$, get $\pois{\pi^*f}{\pi^*g} = \pois{\pi^*f - f(q)}{\pi^*g - g(q)}$ vanishes on $N = \pi^{-1}(q)$. Since $q$ was arbitrary, this bracket vanishes everywhere. 

In the right direction, suppose that $\pi^*C^\oo(Q)$ has zero Poisson bracket. 

Given $q\in Q$, we want to sho $N = \pi^{-1}(q)$ is coisotropic. Choose $f_1,\dots, f_n$ with $f_i(q) = 0$ such that $df_i|_q$ span $T_qQ$. 

Then their pullbacks $\pi^*f_i$ are such that $d(\pi^*f_i)$ span $\ann(\pi^{-1}(q))$, and by assumption $\pois{\pi^*f_i}{ \pi^*f_j} = 0$

\qed

Note for $\dim Q = \frac12\dim M$, we get a ``Lagrangian Fibration."

\section*{\underline{Constant rank submanifolds}}

\defn

A 2-form $\sigma\in\Omega^2(Q)$ has \underline{constant rank} if the rank of $\sigma^\flat|_q:T_qQ\to T_q^*Q$ is constant. 

This is equivalent to saying that the dimension of $\ker(\sigma) = \{v\mid \sigma(v,\cdot) =0\}$ is constant (which is by definition the kernel of $\sigma^\flat)$.

\defn

For $(M,\omega)$ a symplectic manifold with submanifold $j:N\into M$ has \underline{constant rank} if $j^*\omega$ has constant rank.

E.g: isotropic, coisotropic, Lagrangian, symplectic

\prop

If $\sigma \in \Omega_{Cl}^2(Q)$ is a closed 2-form of constant rank, then $\ker(\sigma)\subseteq TQ$ is integrable, i.e. corresponds to some fibration.

\proof

Use Fubini's theorem: $E \subseteq TQ$ is integrable if for all $X, Y \in \Gamma(E) \subseteq \Gamma(TQ) = \ms{X}(Q)$, we have $[X,Y]\in \Gamma(E)$

\underline{Check:} Suppose $X, Y \in \ker(\sigma)$. Then 
\begin{align*}
\iota([X,Y])\sigma & = \iota(L_XY)\sigma \\
& = L_X(\underbrace{\iota(Y)\sigma}_{=0})-\iota(Y)L_X\sigma \\
& = -\iota(Y)(\iota(X)d + d\iota(X))\sigma \\
& = 0 
\end{align*}

So, if $N\subseteq M$ is constant rank, we get a fibration of $N$.

In particular, for coisotropic. 

$\hfill$

For $\sigma\in\Omega^2_{Cl}(Q)$ closed, constant rank, this fibration is called the \underline{null fibration}.

\underline{If} this foliation is ``fibrating," i.e. comes from a submersion $\pi:Q\to B$ (i.e. the space of leaves of foliations is a manifold). 

In that case, $\sigma$ descends to a 2-form $\omega_B\in\Omega^2(B)$, $\pi^*\omega_B = \sigma$. 

Furthermore, $\omega_B$ is symplectic (closed: $\pi^*\omega_B = d\pi^*\omega_B - dt = 0$, nondegenerate...)

So if $N \subseteq M$ $(M,\omega)$ is constant rank submanifold, and if (big if!) the nullfibration of $j^*\omega$ is fibrating, then $B = N/\sim$ (the space of leaves?) inherits a symplectic form. This is the general case of ``symplectic reduction."

\underline{Moser argument}

\defn

The flow of a time-dependent vector field $X_t\in\ms{X}(Q)$, $t \in \R$, is a smooth family of diffeomorphisms $\varphi_t\in\Diff(Q)$ such that $\phi_0 = \Id$, $\phi_t^*(\ms{L}_{X_t}f) = -\dd{}{t}\phi^*_t(f)$

In coordinates, if $X_t = \sum_{i=1}^na_i(x,t)\pp{}{x_i}$, this corresponds to the ODE
\[
\dd{x_i}{t} = a_i(x(t),t)
\]

\thm(Moser stability for volume forms)

Let $Q$ be a compact oriented manifold, $\Lambda_0,\Lambda_1\in\Omega^{\dim Q}(Q)$ are volume forms, with $\int_Q\wedge_0 = \int_Q\wedge_1$. Then there exists a diffeomorphism $\varphi\in\Diff(Q)$ such that $\varphi^*\Lambda^1=\Lambda^0$. 

\proof

Let $\Lambda_t = t\Lambda_1 + (1-t)\Lambda_0$. These are all volume forms, same volume. 

We want a family of diffeomorphisms $\varphi_t$ such that $\varphi_t^*\Lambda_t = \Lambda_0$.

Let $X_t$ corresponding time-dependent vector field having $\varphi_t$ as its flow. 
Then 
$\varphi_t^*\Lambda_t = \Lambda_0 \iff \dd{}{t}(\varphi_t^*\Lambda_t) = 0$. We calculate
\begin{align*}
\dd{}{t}(\varphi_t^*\Lambda_t) & = -\varphi_t^*(-\ms{L}_{X_t}\Lambda_t + \Lambda_1-\Lambda_0) \\
& = \varphi_t^*(-d\iota(X_t)\Lambda_t + d\beta) \\
& = d\varphi_t^*(-\iota(X_t)\Lambda_t + \beta) \\
\end{align*}
Now define $X_t$ by $\iota(X_t)\Lambda_t = \beta$ (where $\beta$ is a primitive of $\Lambda_1-\Lambda_0$ (because they are by definition cohomologous)) (this definition also uniquely determines $X_t$). 

Then we are done.

\qed

\section*{Lecture 11, 10/10/24}

\subsection*{\underline{Moser-Weinstein theorems}}

\underline{Background: Homotopy operators}

\defn

Given smooth maps $F_0,F_1\in C^\oo(Q_1,Q_2)$, a \underline{smooth homotopy} between them is a map $F \in C^\oo([0,1]\times Q_1,Q_2), (t, q) \mapsto F_t(q)$, such that the values at $t = 0$ and $t = 1$ are the given ones.

In this case, $F_0^* = F_1^*$ in de Rham cohomology. But how can we see this? 

Define a homotopy operator
\[
\begin{tikzcd}
\Omega^k(Q_2) \ar[rr, bend right = 30, "h"]\ar[r, "F^*"] & \Omega^k([0,1]\times Q_1) \ar[r, "\int_{[0,1]}"] & \Omega^{k-1}(Q_1)
\end{tikzcd}
\]
Here 
\[
\int_{[0,1]}(ds\wedge\underbrace{\beta_s}_{\in \Omega^{k-1}(Q_1)} + \underbrace{\gamma_s}_{\in\Omega^k(Q_1)}) = \int_0^1\beta_s|ds|
\]

\underline{Exercise:}

For $\alpha\in\Omega^k([0,1]\times Q_2)$, $\int_{[0,1]}d\alpha + d\left(\int_0^1\alpha\right) = \iota_1^*\alpha - \iota_2^*\alpha$

One finds: $d\circ h + h\circ d = F_1^* - F_0^*$ as maps $\Omega^k(Q_2) \to \Omega^k(Q_1)$. 

So $h$ is a chain homotopy between $F_1^*,F_0^*$, so by basic homological algebra, $F_1^* = F_0^*$.

\exm

Consider $\R^n$, let $F_1 = \Id, F_0(x) = 0$. Get homotopy by $F_t = tF_1$ for $t \in [0,1]$. So we get a homotopy operator $h:\Omega^k(\R^n)\to\Omega^{k-1}(\R^n)$.

The point is these homotopy operators give canonical primitives to closed forms. 

\thm(Moser stability theorem for symplectic structures)

Let $M$ be a compact manifold, $\omega_t\in\Omega^2(M)$ a family of symplectic forms with 
\[
\dd{\omega_t}{t} = d\beta_t
\]
for some family of 1-forms $\beta_t$. Then there exists a family $\varphi_t \in \Diff(M)$, with $\varphi_0=\Id_M$, such that
\[
\varphi_t^*\omega_t = \omega_0
\]
\proof

Recall: $X_t \in \ms{X}(M)$ time dependent vector field, has flow $\varphi_t$ ($\varphi_0=\Id_M$)
\[
\dd{}{t}\varphi_t^*\alpha = -\varphi_t^*\left(\ms{L}_{X_t}\alpha\right)
\]
for $\alpha \in \Omega^k(M)$.

We want $0 = -\dd{}{t}(\varphi_t^*\omega_t)$. We have
\begin{align*}
-\dd{}{t}(\varphi_t^*\omega_t) & = \varphi_t^*\left(\ms{L}_{X_t}\omega_t - \dd{\omega_t}{t}\right) \\
& = \varphi_t^*d(\iota_{X_t}\omega_t -\beta_t) 
\end{align*}
(recall that by hypothesis $\dd{\omega_t}{t} = d\beta_t$, which means $[\dd{\omega_t}{t}] = 0$ in $H^2(M)$, which implies $[\omega_t]$ in $H^2(M)$ doesnt depent on $t$. The converse is also true, but is more difficult).

We use compactness to guarantee the existence of flows (this is also what we use it for in the Moser argument from last time). 

Define $X_t$ by $\iota_{X_t}\omega_t=\beta_t$

\qed

\thm(Darboux' theorem (Libermann, 1948))

Let $(M,\omega)$ be a symplectic manifold, $m\in M$. There exists a local coordinate chart $(U,\varphi)$ around $m$, with coordinates $q_1,p_1,\dots,q_n,p_n$, such that $\omega|_U = \varphi^*(\sum_{i=1}^n dq_i\wedge dp_i)$. Coordinates of this type are called ``Darboux coordinates".

\proof

Start by choosing any coordinates around $m$ to reduce to the case that $U \subseteq \R^{2n}$, with $m$ the origin. 

Let $\omega_0 = \sum_{i=1}^n dq_i \wedge dp_i$. Using linear change of coordinates, we may assume $\omega|_{T_0\R^{2n}\times T_0\R^{2n}}$ is the standard one. 

Let $\omega_t = (1-t)\omega_0 + t\omega_1$, where $\omega_1$ is the $\omega$ we defined above. 

These are all standard on $T_0\R^{2n}$, hence are all symplectic on the origin, hence also on some open neighborhood of 0. By shrinking $U$, we may assume symplectic on all of $U$, and may also assume $U$ is an open ball around 0. Now, by de Rham homotopy operator for $U = \R^{2n}$, 
\[
\dd{\omega_t}{t} = \omega_1 = \omega_0 = d\beta
\]
Define $X_t$ by $\iota(X_t) = \beta$. Let $\varphi_t$ be its flow.  But $X_t$ vanishes at $0\in U$ since $\beta_t|_0 = 0$. 

Hence close to 0, $X_t$ is ``small." Hence, the time-1-flow exists in some neighborhood $U'\subseteq U$.

Then define $\varphi_t:U'\to U$ for $t \in [0,1]$.

Now we use Moser's argument. 

\qed

\underline{Remark:} If a compact Lie group $G$ acts on $(M,\omega)$ by symplectic diffeomorphisms, and if $m\in M$ is $G$-fixed, then we get an equivariant Darboux theorem, i.e. with $U$ $G$-invariant, and $\varphi$ $G$-equivariant for some \underline{linear symplectic} $G$-action on $\R^{2n}$.

(Indeed, can make all choices $G$-invariant by using averaging (because the Lie group is compact))


\section*{Lecture 12, 10/15/24}

\subsection*{\underline{Normal Form Theorems}}

Last time, we did Darboux' theorem. Today, we consider generalizations along submanifolds. 

Given a symplectic manifold $(M,\omega)$ and a submanifold $N \subseteq M$, the symplectic structure is uniquely determined on some neighborhood $U$ of $N$ by $\omega|_N \in \Gamma(\wedge^2T^*M|_N)$. More precisely:

\thm(``Master Theorem")

Let $(M_i,\omega_i)$, $i \in \{0,1\}$ be symplectic manifolds and $N_i \subseteq M_i$ submanifolds. Suppose there is an isomorphism of symplectic vector bundles (i.e. a vector bundle whose fibers have symplectic structures)
\[
\begin{tikzcd}
TM_0|_{N_0} \ar[r, "\hat{\psi}"] \ar[d] & TM_1|_{N_1} \ar[d] \\
N_0 \ar[r, "\psi"] & N_1
\end{tikzcd}
\]
such that $\hat{\psi}$ extends $T\psi:TN_0\to TN_1$. Then $\psi:N_0\to N_1$ extends to a symplectomorphism $\varphi:U_0\to U_1$ of open neighborhoods $U_i$ of $N_i$ such that $\tilde{\psi} = T\varphi|_{N_0}$

\proof

Later

\qed

\underline{Review:} Normal bundles. 

Given a submanifold $N \subseteq M$, then the normal bundle $\nu(M,N)$ is defined by $TM|_N/TN$, a vector bundle over $N$. Then:
\begin{itemize}

\item $\nu$ is functorial: Given $\varphi:M_0\to M_1$ with $\varphi(N_0) \subseteq N_1$, we get $\nu(\varphi): \nu(M_0,N_0) \to \nu(M_1,N_1)$.

\item If $M = E \to N$ is a vector bundle, then $TE|_N = E\oplus TN$, so $\nu(E,N) = E$.

\end{itemize}


\defn Given $(M,N)$, a \underline{tubular neighborhood embedding} $\varphi:U\to M$ of some open neighborhood $U \subseteq \nu(M,N)$ of $N$ is an embedding such that $\varphi(N) \subseteq N$ and $\nu(\varphi) = $identity of $\nu(M,V)$. 

\begin{itemize}

\item Tubular neighborhood embeddings exist. For ex, can take $U$ to be a bundle of open balls ($\varepsilon$-neighborhoods of $N$).

\end{itemize}

Now we prove the theorem. 

\proof

Using tubular neighborhood embeddings, we may assume $M_i = \nu(M_i,N_i)$. The map $\hat{\psi}$ gives an isomorphism $\nu(M_0,N_0) \to \nu(M_1,N_1)$, because it takes the tangent bundle to the tangent bundle, and the tangent of the submanifold to the tangent of the submanifold. 

Using this, we may assume $M_0 = M_1 = M$ is a vector bundle $\pi:M\to N = N_0 = N_1$ equipped with two symplectic forms, $\omega_0,\omega_1$, so that $\omega_0|_N = \omega_1|_N$.

In particular, $\iota^*\omega_0 = \iota^*\omega_1$, hence $[\omega_0] = [\omega_1]$. 

The homotopy operator $h:\Omega^k(M) \to \Omega^{k-1}(M)$ for the vector bundle $\pi:M\to N$, gives $\beta\in\Omega^1(M)$, $\beta = h(\omega_1-\omega_0)$ such that $d\beta = \omega_1-\omega_0$. 

We get a family of cohomologous symplectic forms on smaller neighborhood of $N\subseteq M$, $\omega_t = (1-t)\omega_0 + t\omega_1 = \omega_0 + td\beta$.

Now, use the Moser argument.

\qed

Special case: \underline{Lagrangian submanifolds}

Recall that $N \subseteq M$ is Lagrangian $\iff$ $TN^\omega = TN \iff$ $\iota^*\omega = 0$, $\dim N = \frac12\dim M$

\lem

For $N$ Lagrangian, we have canonical isomorphism of vector bundles $\nu(M,N) = T^*N$

\proof

Recall the isomorphism $\omega^\flat:TM|_N\to T^*M|_N$ restricts to an isomorphism $TN^\omega \to \ann(TN)$

But for a Lagrangian submanifold, $TN^\omega=TN$.

So, $\nu(M,N) = TM/TN \to T^*M|_N/\ann(TN) = T^*N$

\qed

\thm(Weinstein)

Let $(M,\omega)$ be symplectic, and $N\subseteq M$ Lagrangian. Then there exists a tubular neighborhood embedding
\[
\psi:U\to M
\]
with $U \subseteq\nu(M,N) = T^*N$ which is \underline{symplectic} (for the standard symplectic structure on $T^*N$).

\proof

We need to find $\hat{\psi}$ so that 
\[
\begin{tikzcd}
T\nu(M,N)|_N \ar[r,"\hat{\psi}"] \ar[d] & TM|_N \ar[d] \\
N \ar[r, "\cong"] & N
\end{tikzcd}
\]
commutes. Now, $TM|_N$ has a Lagrangian tubbundle $TN$. Choose complex Lagrangian subbundle $L \subseteq TM|_N$ (e.g. Take $L = \J(TN)$ for compatible complex structure $\J$), symplectic structure identifies $L \cong T^*N$ so, $TM|_N = TN \oplus T^N$ with fiberwise symplectic structure given by pairing. 

Do the same for $T(T^*M)|_N = TN \oplus T^*N\cong TM|_N$

\qed

For constant rank submanifolds $N \subseteq M$, define the \underline{symplectic normal bundle} as 
\[
F = TN^\omega / (TN \cap TN^\omega)
\]

Note $F \subseteq TM|_N/TN$

\thm(Constant rank embedding theorem)

Let $(M_i,\omega_i)$, $i \in 2$, $N_i \subseteq M_i$ constant rank submanifolds, with symplectic noraml bundles $F_i = TN_i^\omega /(TN_i \cap TN_i^\omega)$. Suppose
\[
\begin{tikzcd}
F_0 \ar[r, "\hat{\psi}"] \ar[d] & F_1 \ar[d] \\
N_0  \ar[r,"\psi"'] & N_1
\end{tikzcd}
\]
is an isomorphism of symplectic vector bundles, and such that $\psi^*\iota_1^*\omega_1 = \iota_0^*\omega_0$. 

Then $\psi$ extends to symplectomorphism of neighborhoods of $N_0, N_1$, inducing $\hat{\psi}$. 

\proof

We want to reduce to the master theorem. Given constant rank submanifold $N \subseteq M$, define three symplectic vector bundles 
\begin{align*}
E & = TN/(TN\cap TN^\omega) \\
F & = TN^\omega/(TN\cap TN^\omega) \\
G & = (TN \cap TN^\omega)\oplus(TN\cap TN^\omega)^*
\end{align*}
with $E, G$ symplectidc subbundles. 

Choosing splittings for $E, F$, we get 
\begin{align*}
TN & \cong E\oplus(TN\cap (TN)^\omega) \\
TN^\omega & \cong F\oplus(TN\cap(TN)^\omega)
\end{align*}

One has 
\[
TM|_N /(TN + TN^\omega) \cong (TN\cap(TN^\omega))^*
\]
choosing splittings get
\[
TM|_N = (TN + TN^\omega)\oplus(TN\cap TN^\omega)^*
\]

We get 
\[
TM|_N = E\oplus F \oplus (TN\cap TN^\omega)\oplus(TN\cap TN^\omega)^*
\]

So $TM|_N$ is sum of three symplectic vector bundles.

Note $E\oplus (TN\cap TN^\omega) = TN$, $F\oplus(TN\cap TN^\omega) = TN^\omega$, and $(TN\cap TN^\omega)\oplus(TN\cap TN^\omega)^* = G$. 

Do this for both $N_i \subseteq M_i, i \in 2$, 
\begin{align*}
TM_0|_{N_0} & = E_0\oplus F_0\oplus G_0 \\
TM_1|_{N_1} & = E_1\oplus F_1\oplus G_1 \\
\end{align*}
The map $\psi$ gives isomorphisms $E_0\to E_1, G_0\to G_1$, and $\hat{\psi}$ gives an isomorphism $F_0\to F_1$. So by the master theorem we get $TM_0|_{N_0} \to TM_1|_{N_1}$ is an isomorphism. 

\qed

\underline{Special cases:}\,
\begin{enumerate}

\item $N_i$ is coisotropic, i.e. $TN_i^\omega\subseteq TN_i, F_i = 0$. 

Get symplectomorphism of neighborhoods provided that 
\[
\psi^*(\iota_1^*\omega_1) = \iota_0^*\omega_0
\]

\item If $N_i$ is isotropic, i.e. $TN_i^\omega\supseteq TN$, $F_i = TN_i^\omega/TN_i$. Need $F_0 \simeq F_1$. 

\item If $N_i$ is symplectic, we have $F_i = (TN_i)^\omega$. Need $N_0\to N_1$ to be symplectic and an isomorphism of symplectic vector bundles.

\end{enumerate}


\subsection*{\underline{Lagrangian fibrations}}

(Here, ``fibration" is used as a synonym for ``fiber bundle," and does \underline{NOT} refer to Serre fibrations)

Let $\begin{tikzcd} M \ar[d,"\pi"] \\ B \end{tikzcd}$ be a fiber bundle, i.e. surjective submersion such that all $\pi^{-1}(b) \cong F$, with local triviality, meaning that any point admits a neighborhood $U$ with $\pi^{-1}(U) \cong U\times F$. 

\defn

Let $(M,\omega)$ be symplectic. A \underline{Lagrangian fibration} of $M$ is a fibration $\fibrate{M}{\pi}{B}$ such that fibers are Lagrangian submanifolds. 

Recall: These are the fibrations such that $\pois{\pi^*f}{\pi^*g} = 0$ for all $f, g \in C^\oo(B)$ and $\dim B = \frac12\dim M$

\exm\,
\begin{enumerate}

\item $\pi:T^*Q\to Q$ cotangent bundle.

\item Suppose $Q = \R^n/\Z^n = (S^1)^n$. Then $TQ = Q\times\R^n$, $\pi:T^*Q=Q\times(\R^n)^*\to (\R^n)^*$ is also a Lagrangian fibration, with fibers the torus.

\end{enumerate}

We'll see that fibers of Lagrangian fibration are products of vector spaces and tori (I notice these are exactly the homeomorphism types of connected Abelian Lie groups, but I don't know the significance of this). 

\thm

Let $(M,\omega)$ be a symplectic manifold, and $\fibrate{M}{\pi}{B}$ a Lagrangian fibration with compact connected fibers. Then there is a canonical, fiberwise transitive action of the vector bundle $\fibrate{T^*B}{}{B}$ on $\fibrate{M}{}{B}$. 

Then $M$ has structure of an ``affine torus bundle."

\proof

Later

\qed

\underline{Background:}

A $G$-action on manifold $M$ is when a Lie group $G$ acts on a manifold $M$. Recall this is a function $G\times M \to M$, $(g, m) \mapsto g\cdot m$, so that $g\cdot(g'\cdot m) = (gg')\cdot m$, and $e\cdot m = m$. 

The orbits through $m$ are $\{g\cdot m\mid g \in G\} = G\cdot m$.

The stabilizer of $m$ is $G_m = \{g \in G \mid g\cdot m = m\}$. Have $G\circ m = G/G_M$ is a manifold 

If the action is \underline{transitive}, meaning for any $m, m'$, there is a $g$ with $g\cdot m = m'$, then $G\circ m = M$. We say the action is \underline{free} if all $G_m = \{e\}$, i.e. all stabilizers are trivial. 

A vector space is a group. If it has a free and transitive action on manifold $M$, then $M$ is an affine space.

If we have a free and transitive action of $G = \R^n/\Z^n$, then $M$ is an affine torus. 

Now, a vector bundle action of $\fibrate{T^*B}{}{B}$ means each $T^*B|_b$ acts on $\pi^{-1}(b)$.

\proof

Let $VM = \ker(T\pi) \subseteq TM$ the vertical subbundle. This is a Lagrangian subbundle. For any $\alpha\in\Gamma(T^*B) = \Omega^1(B)$, we have $\pi^*\alpha \in \Omega^1(M)$

Let $X_\alpha \in \ms{X}(M)$ defined by $\iota(X_\alpha)\omega = -\pi^*\alpha$

Now, we claim that $X_\alpha \in \Gamma(VM)$, i.e. it is a vertical vector field. For all $Y \in \Gamma(VM)$, we have $\omega(X_\alpha, Y) = \iota(Y)\iota(X_\alpha)\omega = -\iota(Y)\pi^*\alpha=0$. 

So $X$ is vertical. 

Note: Restriction of $X_\alpha$ to $\pi^{-1}(m)$ depends only on $\alpha|_m$. 

This gives a map $\pi^*(T^*B)\to VM$.

Let $F_\alpha^t:M\to M$ be the flow of that vector field. Since $X_\alpha$ is vertical it takes fibers to fibers, and exists for all time because the fibers are compact. Action of $T^*B$ is the trace $1$-form $F_\alpha^1$ (etc.)

\section*{Lecture 13, 10/17/24}

\subsection*{\underline{Lagrangian fibrations and a dim-angle coordinates(?)}}

Recall; $\pois{H}{F} = 0 \iff L_{X_H}F = 0$, which implies that the integral curves of $X_H$ are constrained by level sets of $F$. 

More generally, consider $F_1,\dots,F_k$ such that $\pois{F_i}{F_j} = 0$, $\pois{H}{F_i} = 0$, then we can try to build coordinate systems to ``solve" $X_H$.

Recall: Regular level sets of $F = (F_1,\dots, F_k)$ are coisotropic submanifolds of codimension $k$. 

For $k = n$ these will be Lagrangian submanifolds. 

\defn

A \underline{Lagrangian submersion} $\pi:M\to B$ is a submersion with Lagrangian fibers. 

We'll see: All compact fibers of a Lagrangian submersion of a Lagrangian submersion are tori. 

For any submersion $\pi:M\to B$, we have an exact sequence of vector bundles over $M$
\[
\exactshort{\ker(T\pi)}{}{TM}{q}{\pi^*(TB)}
\]
where $\pi^*(TB)$ is the pullback bundle, where $\pi^*(TB)|_m \subseteq T_{\pi(m)}B$. So, $TM/\ker(T\pi) = \pi^*(B)$.

For a Lagrangian submersion, $\omega$ gives a nondegenerate pairing between $\ker(T\pi)$ and $TM/\ker(T\pi)$

(Recall: For symplectic vector space $V$, and Lagrangian subspace $L\subseteq V$, we have a pairing between $L, V/L$, i.e. we have an identification $L \cong (V/L)^*$)

Thus $\ker(T\pi) \cong (TM/\ker(T\pi))^* = \pi^*(T^*B)$

Thus $\ker(T_m\pi) \cong T_{\pi(M)}^*B$ for all $m$.

So, for any fiber $\pi^{-1}(b)$, $\ker(T\pi)|_{\pi^{-1}(b)} = T(\pi^{-1}(b)) \cong \pi^{-1}(b)\times T^*_bB$

\prop

For a Lagrangian submersion $\pi:M\to B$ there is a canonical isomorphism 
\[
\underbrace{\ker(T\pi)}_{\text{tangent bundle to fibers}} \cong \pi^*(T^*B)
\]

\proof

\qed

Taking $v \in \ker(T_m\pi)$ to $\mu\in T_{\pi(m)}^*B$ defined by $\iota(v)\omega_m = -(T_m\pi)^*\mu$

For $\mu\in T_b^*B$, we get $X_\mu \in \ms{X}(\pi^{-1}(b))$ by this isomorphism.

Equivalently, for $\alpha\in\Omega^1(B)$, we get $X_\alpha \in \Gamma(\ker(T\pi))\subseteq \ms{X}(M)$

\prop

For $\mu_1,\mu_2 \in T_b^*B$, we have $[X_{\mu_1},X_{\mu_2}] = 0$.

\proof

Extend $\mu_i$ to $\alpha_i\in\Omega^1(B)$. Let $X_i = X_{\alpha_i}$. By def, $\iota(X_i)\omega = -\pi\alpha_i$. 
Now, 
\begin{align*}
\iota(\underbrace{[X_1,X_2]}_{L_{X_1}X_2})\omega & = L_{X_1}(\iota_{X_2}\omega) - \iota_{X_2}L_{X_1}\omega \\
& = - L_{X_1}\pi^*\alpha_2 - \iota_{X_2}\iota_{X_1}\underbrace{d\omega}_{=0} - \iota_{X_2}d\underbrace{\iota_{X_1}}_{-\pi^*\alpha_1}\omega \\
& = -L_{X_1}\pi^*\alpha + L_{X_2}\pi^*\alpha_1 - d\iota_{X_2}\pi^*\alpha_1 \\
\end{align*}

Because $X_i$ are vertical vector fields (in the kernel of $T\pi$), all three of these terms above disappear.

\qed

\underline{Remark:} Choose a basis of $T_b^*B, \mu_1,\dots,\mu_n$. Get corrresponding vector fields $X_{\mu_1},\dots,X_{\mu_n}$. We can use these to build coordinate systems. 

For $m \in \pi^{-1}(b), (t_1,\dots,t_n) \mapsto F_{t_1}^{X_{\mu_1}}\circ\cdots\circ F_{t_n}^{F_{X_{\mu_n}}}(m)$ is a coordinate system. In fact, this is canonical up to affine transformation. 

So $\pi^{-1}(b)$ has an affine structure. 

\defn

Call a Lagrangian submersion $\pi:M\to B$ \underline{complete} if for all $X_\alpha, \alpha\in\Omega^1(M)$ are complete, i.e. flow exists for all $t$. 

E.g. if fibers $\pi^{-1}(b)$ are compact, then it is complete. 

Let $F_\alpha(t):M\to M$ be the flow of $X_\alpha$. $(F_\alpha^t\circ F_\beta^s = F_{\beta}^s\circ F_\alpha^t$. Let $F_\alpha = F_\alpha^1$. We have $F_\alpha\circ F_\beta = F_{\alpha + \beta}$ (we are using that $F_\alpha^t \circ F_\beta^t = F_{\alpha+\beta}^t$ when they commute). 

On each $\pi^{-1}(b)$, get action of $T_b^*B$ (viewed as an Abelian group).

Then $\mu\circ m = F_\mu(m)$. Note $(\mu_1 + \mu_2)\circ m = \mu_1\circ(\mu_2\circ m)$, and $0\circ m = m$. 

\prop

For a complete Lagrangian submersion, all fibers $\pi^{-1}(b)$ have locally free, transitive action of $T_b^*B$, meaning that there is only one orbit, and discrete stabilizers. 

\proof

Since $X_\mu, \mu\in T_b^*B$ spans $\ker(T\pi)$, the orbits (which are fibers) are $n$-dimensional, hence are open, hence all of $\pi^{-1}(b). $

So $\pi^{-1}(b) = T_b^*B / \Lambda_b$, where $\Lambda_b$ is the stabilizer of $b$.

\qed

We've seen that $\pi^{-1}(b) = T^*_bB/\Lambda_b$, where $\Lambda_b$ is discrete. Choose a basis of $e_1, \dots, e_n\ell$ of $\Lambda_b$, extend to basis $e_1,\dots,e_n$ of $T_b^*B$

Then $\pi^{-1}(b) = \R^n/\Z^k = (\R/\Z)^k\times\R^{n-k}$. In particular, if this is compact, it must be a torus.




\section*{Lecture 14, 10/22/24}

Missed

\section*{Lecture 15, 10/24/24}

\subsection*{\underline{Hamiltonian group actions}}

\underline{Review of Lie theory:}

\begin{itemize}

\item A Lie group $G$ is a manifold, also a group, such $Mult_G:G\times G \to G$ is $C^\oo$. 

\item Cartan's theorem: every closed subgroup of a Lie group is a Lie group. 

\item A Lie algebra $\g = T_eG$ with bracket from left invariant vector fields, the set of which we denote by $\ms{X}^L(G)$ (i.e. invariant under all $L_a:G\to G, g \mapsto ag$). For matrix Lie groups, the bracket is the commutator.

\item There is a map called the exponential map $\exp:\g\to G$ which sends $\xi \mapsto \exp(\xi) = \gamma_\xi(1)$, where $\gamma_\xi:\R\to G$ is a one-parameter subgroup such that $\dd{}{t}|_{t=0}\gamma_\xi(t) = \xi$. The assignment of $\g = T_eG$ to $G$ is functorial. For matrix Lie groups, it is the matrix exponential

\item Adjoint action: For every $a \in G$, we have the action $Ad_a:G\to G$, $g\mapsto aga^{-1}$. At the tangent space we have the derivative $T_eAd_a$, and the assignment $a \mapsto T_eAd_a$ is a Lie algebra morphism $G \mapsto \Aut(\g) \subseteq \GL(\g)$. 

This induces a Lie algebra morphism $\g \to \Aut(\g)$, which is the Lie algebra of $\GL(\g)$. THe image of $\xi$ under this is called $ad_\xi$. It turns out that
\[
ad_\xi(y)=\dd{}{t}Ad_{\exp(t\xi)}(y)
\]

The assignment of $G$ to $T_eG=\g$ is functorial. That is, if we have $\varphi:G_1\to G_2$, then the a map $T_e\varphi:\g_1\to\g_2$ makes the diagram commute:
\[
\begin{tikzcd}
\g_1 \ar[r, "\exp"']\ar[d, "T_e\varphi"] & G_1 \ar[d, "\varphi"]\\
\g_2 \ar[r, "\exp"] & G_2
\end{tikzcd}
\]

A consequence is that $\exp(ad_\xi) = \Ad(\exp(\xi))$

\end{itemize}


\defn

Let $G$ be a Lie group. A \underline{$G$-action} on a manifold $Q$ is a group homomorphism $\ms{A}:G\to\Diff(Q)$, $g\mapsto \ms{A}_g$, such that the map $\ms{A}:G\times Q \to Q, (g,q) \mapsto \ms{A}_g(q)$ is smooth. 

\defn

Let $\g$ be a finite dimensional Lie algebra. A $\g$-action on $Q$ is a Lie algebra homomorphism $\g\to\ms{X}(Q)$, $\xi\mapsto \xi_Q$ such that the map $\g\times Q\to TQ, (\xi,q) \mapsto \xi_Q(q)$ is smooth. 

\exm
\,

\begin{enumerate}

\item For $Q = G$, we have the $G$-actions

\begin{itemize}

\item Adjoint action: $g\mapsto Ad_g, a \mapsto gag^{-1}$

\item Left multiplication: $g \mapsto L_g, a \mapsto ga$

\item Right multiplication: $g\mapsto R_{g^{-1}}: a\mapsto ag^{-1}$

\end{itemize}

\item Any representation of $G$, i.e. a group homomorphism $G \mapsto \GL(V)$ is a G-action 

\item Given a $G$-action on $Q$, we get actions on $TQ, T^*Q$ induced by $g \mapsto \ms{A}_g$, $G \curvearrowright TQ$, $g\mapsto T\ms{A}_g$, $G \curvearrowright T^*Q, g \mapsto (T\ms{A}_{g^{-1}})^*$

\end{enumerate}


\underline{Notation:} For $G\curvearrowright Q$, instead of $g\mapsto \ms{A}_g$, we write $\ms{A}_g(q) = g\cdot q$. Note 
\[
g_1\cdot(g_2\cdot q) = (g_1\cdot g_2)\cdot q
\]



Given a $G$-action $\ms{A}:G\mapsto\Diff(G)$ $g\mapsto \ms{A}_g$, we get a Lie algebra $\g$-action $\g \mapsto \ms{X}(Q),\xi\mapsto\xi_Q$ as follows.

From $\ms{A}$ we get a $G$-representation on functions: 
\[
(g\cdot f)(q) = f(g^{-1}\cdot q)
\]
Define $\xi_Q$ by derivative: 
\[
(\ms{L}_{\xi_q}(f))(q) = \dd{}{t}|_{t=0}f(\exp(-t\xi)\cdot q)
\]

I.e $\xi_Q$ is the vector field having flow $g \mapsto (\exp t\xi)\cdot q$

\prop

The map $\g\to \ms{X}(Q)$, $\xi\mapsto \xi_Q$ is a Lie algebra action, i.e. 
\[
[\xi_Q,\eta_Q] = [\xi,\eta]_Q
\]
We have 
\[
(Ad_g\xi)_Q = (\ms{A}_g)_*\xi_Q
\]

\proof

\qed


For any $\xi\in G$, let $\xi^L\in\ms{X}^L(G)$ be the corresponding left-invariant vector field, and $\xi^R \in \ms{X}^R(G)$ the right invariant vector field with the property that $\xi^R|_e = \xi^L|_e = \xi$

\exm

Consider left multiplication of $G$ on itself $g\cdot a = ga$. Then $\xi_Q = ?$

Since this commutes with right multiplication, $\xi_Q$ is a right-invariant vector field. 

At $e$, we have $(\ms{L}_{\xi_Q}f)(e) = \dd{}{t}|_{t=0}f(\exp(-t\xi)) = -\underbrace{\xi}_{\in T_eG}(f)$

So $\xi_Q = -\xi^R$. So 
\[
[-\xi^R,-\eta^R] = -[\xi,\eta]^R
\]
similarly for the action $g\cdot a = ag^{-1}$, we get the left invariant vector field $\xi_Q = \xi^L$. So 
\[
[\xi^L,\eta^L] = [\xi,\eta]
\]

For action $g\cdot a = gag^{-1}$, $\xi_Q = \xi^L-\xi^R$

\section*{Lecture 16, 1/12/24}

\subsection*{\underline{Hamiltonian action and moment maps}}

\underline{Recall:} Given a group action $\ms{A}:G\to\Diff(M)$, we define generating vector fields by 
\[
\g \to \ms{X}(M), \xi \mapsto \xi_M
\]
\[
(\ms{L}_{\xi_M}f)(m) = \dd{}{t}|_{t=0}f(\exp(-t\xi)\cdot m)
\]
where $\exp:\g\to G$ is the Lie exponential.

\exm

Let $M = \R^n, G = \GL(n,\R)$, $\g = \gl(n,\R) = \End(R^n)$. What is the formula for the generating vector fields? 

For $A \in \gl(n,\R)$, 
\begin{align*}
(\ms{L}_{A_{\R^n}}f)(x) & = \dd{}{t}|_{t=0}f(\exp(-tA)\cdot x) \\
& = \sum_{i=1}^n\pp{f}{x_i}(x)(Ax)_i \\
& = \underbrace{-\sum_{i,j=1}^n A_{ij}x_j\pp{}{x_i}}_{A_{\R^n}}f
\end{align*}

If we take the commutator, 
\[
[A_{\R^n},B_{\R^n}] = [A,B]_{\R^n}
\]
In general, 
\[
[\xi_M,\zeta_M] = [\xi,\zeta]_M
\]

\exm

Take $M = \R^n, G = \R^n$ as an additive Lie group. Then $\g = \R^n$ (with zero Lie bracket), and $\exp:\g\to G$ is given by the identity: $\exp(b) = b$. For any $b \in \g = \R^n$ gives rise to a 1-parameter subgroup $\gamma_b(t) = tb$, and the derivative of this is again $b$. 

Let $G$ act on $\R^n$ by translations, namely $x \mapsto x - b$ (we use a minus sign so formulas turn out nicer, but can put a plus sign). 

Then the generating vector field is 
\begin{align*}
(\ms{L}_{B_{\R^n}}f)(x) & = \dd{}{t}|_{t=0} f(\exp(-tb)\cdot x) \\
& = \dd{}{t}|_{t=0}f(x + tb) \\
& = \sum_i\pp{f}{x_i}b_i\\
\end{align*}
so $b_{\R^n} = \sum b_i\pp{}{x_i}$

For a real vector space $V$, let $G = \GL(V)$, $T_vV = V$. Then 
\[
\xi_V|_v = -\xi\cdot v, \g = \gl(V) = \End(V)
\]

Let $(M,\omega)$ be a (connected) symplectic manifold, $\ms{A}:G\to\Diff(M)$, which has generating vector field $\g\to\ms{X}(M,\omega) = \{X \mid \ms{L}_X\omega = 0\}$

Recall we have $C^\oo(M) \to \ms{X}(M)$ given by $f \mapsto X_f$, where $X_f$ is determined by. $\iota(X_f)\omega = -df$. Such a vector field is called Hamiltonian.

\defn

The $G$-action is called \underline{weakly Hamiltonian} if all generating vector fields are Hamiltonian vector fields, i.e. $\xi_M \in \ms{X}_{Ham}(M,\omega)$. 

We have the short exact sequence 
\[
\exactshort{\R}{}{C^\oo(M)}{}{\ms{X}_{Ham}(M,\omega)}
\]
where the kernel of $C^\oo\to \ms{X}_{Ham}(M,\omega)$ is the constant functions. But because the action is weakly hamiltonian, we have a map from $\g\to\ms{X}_{Ham}(M,\omega)$, which we can lift: 
\[
\begin{tikzcd}
0 \ar[r] & \R \ar[r] & C^\oo(M) \ar[r] & \ms{X}_{Ham}(M,\omega) \ar[r] & 0 \\
&&& \g \ar[u] \ar[ul, dotted, "\tilde{\Phi}"] 
\end{tikzcd}
\]

The choice of a linear map $\tilde{\Phi}:\g\to C^\oo(M)$ (lifting $\xi \to \xi_M)$ is called a \underline{weak moment map}, or sometimes a comoment map. 

In general, this won't be a Lie algebra homomorphism. 

The comoment map can be regarded as a map $\Phi:M \to \g^*$, $\langle \Phi(m),\xi\rangle = \tilde{\Phi}(\xi)(m)$

This is called the (weak) moment map. 

\defn

A symplectic $G$-action on a symplectic manifold $(M,\omega)$ is called \underline{Hamiltonian} if there exists a $G$-equivariant $\Phi:M\to\g^*$ such that $\iota(\xi_M)\omega = -d\langle\Phi,\xi\rangle$

The map $\tilde{\Phi}:G\to C^\oo(M)$, $\tilde{\Phi}(\xi)(m) = \langle \Phi(m),\xi\rangle$ is the comoment map. 

\underline{Remarks:}
\begin{enumerate}
	\item $G$ acts on $\g$ by $g\cdot\xi = Ad_g\xi$, hence on $\g^*$ by $g\cdot\mu = (Ad_{g^{-1}})^*\mu$. So equivariance means $\Phi(g\cdot m) = g\cdot\Phi(m)$.

Alternatively, $\tilde{\Phi}(Ad_g\xi) = g\cdot\tilde{\Phi}(\xi)$, so $(g\cdot f)(m) = f(g^{-1}\cdot m)$

Infinitesimally, $\g$ acts on $\g^*$ by $\xi\cdot\mu = -(ad_\xi)^*\mu$

Now, $\Phi(g\cdot m) = g\cdot \Phi(m)$ implie $\ms{L}_{\zeta_M}\Phi = -(ad_\zeta)^*\Phi$

\item Name comes from French ``application moment" (Souriaux (sp?)), but ``Moment map" is an incorrect translation. Correct is ``momentum map". 

\item If $G$ is abelian (e.g. $\R^n$, torii, or products thereof), equivariance means invariance. 
\end{enumerate}

\prop

For Hamiltonian $G$-action, the comoment map 
\[
\tilde{\Phi}:\g\to C^\oo(M)
\]
is a Lie algebra homomorphism. 

\proof

Denote $\Phi^\xi = \tilde{\Phi}(\xi) = \langle \Phi, \xi\rangle$. Then $\pois{\Phi^\xi}{\Phi^\eta} = \underbrace{\ms{L}_{\xi_M}\Phi^\eta}_{\ms{L}_{\xi_M}\langle\Phi,\eta\rangle} = \langle -ad_\xi^*\Phi,g\rangle = -\langle\Phi,[\xi,\eta]\rangle = -\Phi^{[\xi,\eta]}$ (there is a sign error in here somewhere...)

\qed

\prop

A weakly Hamiltonian $G$-action on $(M,\omega)$ is Hamiltonian in following cases:
\begin{enumerate}[label=(\alph*)]

\item $G$ compact

\item $M$ compact

\end{enumerate}

\proof

$\Phi$ is $G$-equivariant if and only if $(g\cdot\Phi)(m) = \Phi(g\cdot m)$. Given a \underline{weak} moment map, define
\[
(g\cdot\Phi)(m) = g\cdot(\Phi(g^{-1}\cdot m))
\]

This is again a weak moment map. If $G$ is compact, can average 
\[
\tilde{\Phi}(m) = \int_G(g\cdot\Phi)(m)|dg|
\]
This is a $G$-equivariant map. This proves (a).

Now, assume $M$ is compact. Note: weak moment maps $\Phi:M\to\g^*$ are unique up to a constant. If $M$ is compact, we can fix a normalization by 
\[
\int_M\Phi\frac{\omega^n}{n!} = 0
\]
This uniquely determines $\Phi$. If $\Phi$ is normalized then $g\cdot\Phi$ is again normalized, hence $g\cdot\Phi=\Phi$.

\qed

Remark: 

If we have a lie algebra homomorphism $\g\to \ms{X}_{Ham}(M,\omega)$, we want to lift to a Lie algebra homomorphism $\tilde{\Phi}.$ We have the central extension $\hat{\g}$
\[
\begin{tikzcd}
0 \ar[r] & \R \ar[r] & C^\oo(M) \ar[r] & \ms{X}_{Ham}(M,\omega) \ar[r] & 0 \\
0 \ar[r] & \R \ar[r] & \hat{\g} \ar[u]\ar[r] & \ar[l,bend left = 30, dotted] \g\ar[u] \ar[r] & 0 
\end{tikzcd}
\]
So the weak Hamiltonian action is Hamiltonian if $G$ is connected and the central extension $\hat{\g}$ is ``trivial" (splits), E.G. $G = \SL(n,\R)$.

\exm

Recall: Cotangent bundles $M = T^*Q, \omega = -d\theta$ (in coordinates $\theta = \sum p_idq_i$). 

For $Y \in \ms{X}(Q)$, the cotangent lift $Y_{T^*}\in\ms{X}(T^*Q)$ is Hamiltonian vector field, with function 
\[
H = \langle \theta, Y_{T^*} = \iota(Y_{T^*})\theta \in C^\oo(T^*Q)
\]

This gives a Lie algebra morphism (exercise)
\[
\ms{X}(Q) \to C^\oo(T^*Q), Y \mapsto \iota(Y_{T^*})\theta
\]

This is $\Diff(Q)$-equivariant (Exercise). 

If $Q$ has a $G$-action, $\ms{A}:G\to\Diff(Q) \curvearrowright T^*Q$

we gete a $G$-action on $T^*Q$ by composition, with comoment map
\[
\tilde{\Phi}:\g\to C^\oo(T^*Q), \xi \mapsto \iota(\xi_{T^*Q})\theta
\]
In coordinates, if 
\[
Y = \sum Y_j(q)\pp{}{q_j} \implies H = \sum Y_j(q)p_j
\]

Special case: $Q = \R^n, G = \GL(n,\R)$, $G\curvearrowright T^*Q = T^*\R^n = \R^n\times\R^n$

Generating vector fields are given by the formula: for $A \in \gl(n,\R) = Mat_{\R}(n)$, 
\[
A_{\R^n}|_{x} = -\sum_{i,j}A_{ij}x_j\pp{}{x_i}
\]

Comoment map: $\g\to C^\oo(T^*\R^n), A \mapsto - \sum A_{ij}q_jp_i$. 

To get moment map, first identify $\gl(n,\R) \simeq \gl(n,\R)^*$ by the bilinear form 
\[
\langle A, B\rangle = \tr(AB)
\]
This form is $Ad$-invariant:
\[
\langle Ad_g A, Ad_gB \rangle = \langle A, B \rangle
\]

In terms of this identity (?) 
$\Phi:T^*\R^n\to\gl(n,\R)$, $(q, p) \mapsto \Phi(q,p)_{ij} = -q_jp_i$

Same kind of calculation for any $G \subseteq \GL(n,\R)$ such that $\g$ invariant under $A \mapsto A^T$.

E.G $G = \SO(n)$ rotation group, with $\g = so(n) = \{A \mid A^T = -A\}$, the Skew-adjoint matrices.

The moment map becomes
\[
\Phi:T^*\R^n \to so(n)^* =so(n)
\]
\begin{align*}
(q,p) \mapsto \Phi(q,p)_{ij} & =-\frac12(q_jp_i - q_ip_j) \\
& = \frac12(q_ip_j - q_jp_i)
\end{align*}

If $h = 3$ this is $\vec{q}\times\vec{p}$ (up to factor). In physics this is angular momentum, hence momentum map.

\exm

\underline{Special case:} $Q = \R^n$, $M = T^*\R^n$, $G = \R^n$ acting by $x \mapsto x - b$.

Generating vector field: $b_{\R^n} = \sum b_i\pp{}{q_i}$ 

Hamiltonian: $\sum b_ip_i$

Comoment map: $\tilde{\Phi}:\R^n\to C^\oo(T^*\R^n), b \mapsto \tilde{\Phi}(q,p) = b\cdot p$. 

$\Phi:T^*\R^n \to (\R^n)^* \cong \R^n$ (by dot product), $(q, p) \mapsto p$ (linear momentum). 

These calculations generalize to ``exact symplectic manifolds," which are symplectic manifolds whose symplectic form is also exact, i.e. $\omega = -d\theta$ (for some $\theta \in \Omega^1(M)$)

Whenever $\ms{A}:G \to \Diff(M)$ preserves $\theta$, (i.e. $\ms{A}^*_g\theta = \theta$), then the generating vector field $\xi_M$ are Hamiltonian, with obvious candidate for the moment map 
\[
\langle\Phi,\xi\rangle = \iota(\xi_M)\theta
\]
Check: 
\begin{align*}
d\langle\Phi,\xi\rangle & = d\iota(\xi_M)\theta \\
& = (\ms{L}(\xi_M)-\iota(\xi_M)d)\theta \\
& = \iota(\xi_M)\omega
\end{align*}
There is a sign error here somewhere as well...

\exm: Symplectic representation

Let $(E,\omega)$ be a symplectic vector space. Then the symplectic group $G = \spew$ acts on $E$

Generating vector fields for this action: $\xi_E|_V = -\xi\cdot v$ (identifying $T_vE \cong E$)

$\Phi:E\to\g^*$, $\langle \Phi(v),\xi\rangle =(\cdots) \frac12\omega(\xi v, v)$. Taking the exterior derivative, 
\begin{align*}
\langle d \langle \Phi,\xi \rangle |_V,w \rangle & = \dd{}{t}|_{t=0}\langle\Phi(v + tw), \xi \rangle \\
& = \dd{}{t}|_{t=0}\omega(\xi\cdot(v + tw), v + tw) \\
& = \omega(\xi\cdot w, v) + \omega(\xi\cdot v, w) \\
& = \omega(\xi\cdot v, w) \\ 
& = -\langle\iota(\xi_E|_V)w,w\rangle
\end{align*}

\underline{Remark:} If $G\curvearrowright (M,\omega)$ is a Hamiltonian action, and $\varphi:H\to G$ is a Lie group homomorphism, then we get an action $H \curvearrowright (M,\omega)$ just by composition. Then the action of $H$ is again Hamiltonian, with moment map
\[
\h \to \g \to C^\oo(M)
\]
and moment map
\[
M \to \g^* \to \h^*
\]

\section*{Lecture 17, 11/14/24}

Suppose $G \curvearrowright (M,\omega)$ is a \underline{Hamiltonian} $G$-action. Then there exists a $G$-equivariant $\Phi:M\to\g^*$ such that
\[
\iota(\zeta_M)\omega = -d\langle\Phi,\zeta\rangle
\]
(i.e. $\langle \Phi,\zeta\rangle$ are Hamiltonian for $\zeta_M$). 

Then $\g \to C^\oo(M)$ given by $\zeta \mapsto \langle\Phi,\zeta\rangle$ is a Lie algebra map. 

\underline{Exact Case:}

Let $\omega = -d\theta$. If $G$ preserves $\theta$ then 
\[
\langle\Phi,\zeta\rangle = -\iota(\zeta_M)\theta
\]

\exm

Let $(E,\omega)$ be a symplectic vector space, and let $G = \spew\curvearrowright E$ is Hamiltonian with moment map 
\[
\langle\Phi(v),\zeta\rangle = -\frac12\omega(v,\zeta v)
\]

\exm

Pick compatible complex structure $\J$ on $(E,\omega)$. Then $E$ becomes a \underline{complex} vector space with Hamiltonian metric  (inner prod) given by 
\[
h(v,w) = \underbrace{g(v,w)}_{=\omega(v,\J w)} + i\omega(v,w)
\]

Now, $U(E) \subseteq\spew$ preserves $\omega$. If $\varphi:H\to G$  is a group homomorphism then 
\[
\Phi_H = \varphi^*\circ\Phi
\]
where $\varphi^*:\g^*\to\h^*$ is the dual map.

\underline{Notation:} For inner product spaces, and linear map $A:E\to F$, let $A^\dagger:F\to E$ be the adjoint. Then 
\begin{align*}
U(E) & = \{A:F\to E \mid A^\dagger = A^{-1}\} \\
u(E) & = \{\xi:E\to E \mid \xi^\dagger = -\xi\}
\end{align*}

So 
\begin{align*}
h(v,\xi v) & = -h(\xi v, v) = - \bar{h(v,\xi v)}
\end{align*}
So $h(v,\xi v) = i\omega(v,\xi v)$, and so 
\[
\langle\Phi(v),\xi\rangle = \frac{i}{2}h(v,\xi v)
\]
$u(E)$ has positive definite, $U(E)$-invariant metric given by 
\[
(\xi,\eta) = \tr(\xi^\dagger\eta) = -\tr(\xi,\eta)
\]
So we identify $u(E)\cong u(E)^*$ by this metric. 

Write 
\begin{align*}
\langle\Phi(v),\xi\rangle & = \frac{i}{2}h(v,\xi v) \\
& = \frac{i}{2} v^\dagger\xi v \\
& = \frac{i}{2}\tr(v^\dagger\xi v) \\
& = \frac{i}{2}\tr(vv^\dagger \xi) \\
& = \tr(\frac{i}{2}vv^\dagger\xi)
\end{align*}
where we view $v$ as a linear map $\C\to E$. So 
\[
\Phi(v) = -\frac{i}{2}vv^\dagger
\]

For sepcial case $E = \C^n$, $v =\begin{pmatrix} z_1 \\ \vdots \\ z_n \end{pmatrix}, v^\dagger = (\bar{z_1},\cdots,\bar{z_n})$, so 
\[
\Phi(z) = \frac{1}{2i}zz^\dagger
\]
\[
(\Phi(z))_{ij} = \frac{1}{2\sqrt{-1}}z_i\bar{z_j}
\]

\exm

Let $M = \CP(n),$ with Fubini-Study form $\omega_{FS}$ and with action of $U(n+1)$, $[z] = [z_0:\dots:z_n], A\cdot[z] = [Az]$.

This action is Hamiltonian, with moment map 
\[
\Phi:\CP(n)\to u(n+1)^* \geq u(n + 1)
\]
\[
[z_0:\dots:z_n] \mapsto \frac{1}{2\sqrt{-1}}\frac{zz^\dagger}{\norm{z}^2}
\]

We will see a proof of this later. 

\subsection*{\underline{Coadjoint orbits}}

Recall: A homogeneous $G$-space is a manifold $M$ with \underline{transitive} $G$-actio n (i.e. only one orbit). 

Given $m\in M$, with stabilizer $H = G_m$, get $M = G/H$.

Consider homogeneous Hamiltonian $G$-space $(M,\omega),\Phi:M\to\g^*$. 

Since the generating vector fields $\xi_M$ span tangent space to $M$ everywhere, the 2-form $\omega$ is determined entirely by moment map condition !
\[
\omega(\xi_M,\cdot) = -d\langle\Phi,\xi\rangle
\]
\begin{align*}
\omega(\xi_M,\eta_M) & = -\iota(\eta_M)d\langle\Phi,\xi\rangle \\
& = -\ms{L}(\eta_M)\langle\Phi,\xi\rangle \\
& = -\langle\ms{L}(\eta_M)\Phi,\xi\rangle \\
& = \langle\eta\cdot\Phi,\xi\rangle \\
& = -\langle\Phi,\eta\cdot\xi\rangle \\
& = \langle\Phi,[\xi,\eta]\rangle
\end{align*}
where the $\cdot$ is the coadjoint action. 

So $\omega(\xi_M,\zeta_M) = \langle\Phi,[\xi,\eta]\rangle$

Note also: $\Phi(M)\subseteq\g^*$ is a single coadjoint orbit. 

\thm[Kostant, Kirillov, Souriau (Sp? all)]

Let $\ms{O}\subseteq\g^*$ be an orbit under the coadjoint action. Then $\ms{O}$ has a unique symplectic $2$-form $\omega$ such that the $G$-action is Hamiltonian, with $\Phi:\ms{O}\to\g^*$ the inclusion. 

\proof

Recall generating vector field for $G$-action on $\g^*$ are $\xi_{\g^*}|_{\mu} = -\xi\cdot\mu$ (coadjoint representation). 

This is also formula for $\xi_{\ms{O}}|_{M}$. Only candidate for $\omega$ (taking care of uniqueness) is
\begin{align*}
\omega|_{\mu}(\xi_{\ms{O}}|_{M}, \eta_{\ms{O}}|_{M}) & = \langle\mu,[\xi,\eta]\rangle\\
&=-\langle\xi\cdot\mu,\eta\rangle\\
&=\langle\eta\cdot\mu,\xi\rangle
\end{align*}
This is well-defined. Now, $\xi_G|_M\in\ker\omega|_M \iff \omega_M(\xi_{\ms{O}}|_M,\eta_{\ms{O}}|_M) = 0$ for all $\eta \iff$ $\langle\xi\cdot\mu,\eta\rangle=0$ for all $\eta \iff \xi\cdot\mu = 0$, and gives nondegenerate.

If $\omega$ closed:
\begin{align*}
\iota(\xi_{\ms{O}})d\omega & = \underbrace{\ms{L}(\xi_{\ms{O}})}_{=0}\omega-d\iota(\xi_{\ms{O}})\omega \\
& = dd\langle\Phi,\xi\rangle \\
& = 0
\end{align*}

\qed











\end{document}
