
\documentclass[x11names,reqno,14pt]{extarticle}
% Choomno Moos
% Portland State University
% Choom@pdx.edu


%% stupid experiment %%
%%%%%%%%%%%%% PACKAGES %%%%%%%%%%%%%

%%%% SYMBOLS AND MATH %%%%
\let\oldvec\vec
\usepackage{authblk}	% author block customization
\usepackage{microtype}	% makes stuff look real nice
\usepackage{amssymb} 	% math symbols
\usepackage{siunitx} 	% for SI units, and the degree symbol
\usepackage{mathrsfs}	% provides script fonts like mathscr
\usepackage{mathtools}	% extension to amsmath, also loads amsmath
\usepackage{esint}		% extended set of integrals
\mathtoolsset{showonlyrefs} % equation numbers only shown when referenced
\usepackage{amsthm}		% theorem environments
\usepackage{relsize}	%font size commands
\usepackage{bm}			% provides bold math
\usepackage{bbm}		% for blackboard bold 1

%%%% FIGURES %%%%
\usepackage{graphicx} % for including pictures
\usepackage{float} % allows [H] option on figures, so that they appear where they are typed in code
\usepackage{caption}
\usepackage{hyperref}
%\usepackage{titling}
\usepackage{tikz} % for drawing
\usetikzlibrary{shapes,arrows,chains,positioning,cd,decorations.pathreplacing,decorations.markings,hobby,knots,braids}
\usepackage{subcaption}	% subfigure environment in figures

%%%% MISC %%%%
\usepackage{enumitem} % for lists and itemizations
\setlist[enumerate]{leftmargin=*,label=\bf \arabic*.}

\usepackage{multicol}
\usepackage{multirow}
\usepackage{url}
\usepackage[symbol]{footmisc}
\renewcommand{\thefootnote}{\fnsymbol{footnote}}
\usepackage{lastpage} % provides the total number of pages for the "X of LastPage" page numbering
\usepackage{fancyhdr}
\usepackage{manfnt}
\usepackage{nicefrac}
%\usepackage{fontspec}
%\usepackage{polyglossia}
%\setmainlanguage{english}
%\setotherlanguages{khmer}
%\newfontfamily\khmerfont[Script=Khmer]{Khmer Busra}

%%% Khmer script commands for math %%%
%\newcommand{\ka}{\text{\textkhmer{ក}}}
%\newcommand{\ko}{\text{\textkhmer{ត}}}
%\newcommand{\kha}{\text{\textkhmer{ខ}}}

%\usepackage[
%backend=biber,
% numeric
%style=numeric,
% APA
%bibstyle=apa,
%citestyle=authoryear,
%]{biblatex}

\usepackage[explicit]{titlesec}
%%%%%%%% SOME CODE FOR REDECLARING %%%%%%%%%%

\makeatletter
\newcommand\RedeclareMathOperator{%
	\@ifstar{\def\rmo@s{m}\rmo@redeclare}{\def\rmo@s{o}\rmo@redeclare}%
}
% this is taken from \renew@command
\newcommand\rmo@redeclare[2]{%
	\begingroup \escapechar\m@ne\xdef\@gtempa{{\string#1}}\endgroup
	\expandafter\@ifundefined\@gtempa
	{\@latex@error{\noexpand#1undefined}\@ehc}%
	\relax
	\expandafter\rmo@declmathop\rmo@s{#1}{#2}}
% This is just \@declmathop without \@ifdefinable
\newcommand\rmo@declmathop[3]{%
	\DeclareRobustCommand{#2}{\qopname\newmcodes@#1{#3}}%
}
\@onlypreamble\RedeclareMathOperator
\makeatother

\makeatletter
\newcommand*{\relrelbarsep}{.386ex}
\newcommand*{\relrelbar}{%
	\mathrel{%
		\mathpalette\@relrelbar\relrelbarsep
	}%
}
\newcommand*{\@relrelbar}[2]{%
	\raise#2\hbox to 0pt{$\m@th#1\relbar$\hss}%
	\lower#2\hbox{$\m@th#1\relbar$}%
}
\providecommand*{\rightrightarrowsfill@}{%
	\arrowfill@\relrelbar\relrelbar\rightrightarrows
}
\providecommand*{\leftleftarrowsfill@}{%
	\arrowfill@\leftleftarrows\relrelbar\relrelbar
}
\providecommand*{\xrightrightarrows}[2][]{%
	\ext@arrow 0359\rightrightarrowsfill@{#1}{#2}%
}
\providecommand*{\xleftleftarrows}[2][]{%
	\ext@arrow 3095\leftleftarrowsfill@{#1}{#2}%
}
\makeatother

%%%%%%%% NEW COMMANDS %%%%%%%%%%

% settings
\newcommand{\N}{\mathbb{N}}                     	% Natural numbers
\newcommand{\Z}{\mathbb{Z}}                     	% Integers
\newcommand{\Q}{\mathbb{Q}}                     	% Rationals
\newcommand{\R}{\mathbb{R}}                     	% Reals
\newcommand{\C}{\mathbb{C}}                     	% Complex numbers
\newcommand{\K}{\mathbb{K}}							% Scalars
\newcommand{\F}{\mathbb{F}}                     	% Arbitrary Field
\newcommand{\E}{\mathbb{E}}                     	% Euclidean topological space
\renewcommand{\H}{{\mathbb{H}}}                   	% Quaternions / Half space
\newcommand{\RP}{{\mathbb{RP}}}                       % Real projective space
\newcommand{\CP}{{\mathbb{CP}}}                       % Complex projective space
\newcommand{\Mat}{{\mathrm{Mat}}}						% Matrix ring
\newcommand{\M}{\mathcal{M}}
\newcommand{\GL}{{\mathrm{GL}}}
\newcommand{\SL}{{\mathrm{SL}}}

\newcommand{\tgl}{\mathfrak{gl}}
\newcommand{\tsl}{\mathfrak{sl}}                  % Lie algebras; i.e., tangent space of SO/SL/SU
\newcommand{\tso}{\mathfrak{so}}
\newcommand{\tsu}{\mathfrak{sl}}


% typography
\newcommand{\noi}{\noindent}						% Removes indent
\newcommand{\tbf}[1]{\textbf{#1}}					% Boldface
\newcommand{\mc}[1]{\mathcal{#1}}               	% Calligraphic
\newcommand{\ms}[1]{\mathscr{#1}}               	% Script
\newcommand{\mbb}[1]{\mathbb{#1}}               	% Blackboard bold


% (in)equalities
\newcommand{\eqdef}{\overset{\mathrm{def}}{=}}		% Definition equals
\newcommand{\sub}{\subseteq}						% Changes default symbol from proper to improper
\newcommand{\psub}{\subset}						% Preferred proper subset symbol

% Categories
\newcommand{\catname}[1]{{\text{\sffamily {#1}}}}

\newcommand{\Cat}{{\catname{C}}}
\newcommand{\cat}[1]{{\catname{\ifblank{#1}{C}{#1}}}}
\newcommand{\CAT}{{\catname{Cat}}}
\newcommand{\Set}{{\catname{Set}}}

\newcommand{\Top}{{\catname{Top}}}
\newcommand{\Met}{{\catname{Met}}}
\newcommand{\PL}{{\catname{PL}}}
\newcommand{\Man}{{\catname{Man}}}
\newcommand{\Diff}{{\catname{Diff}}}

\newcommand{\Grp}{{\catname{Grp}}}
\newcommand{\Grpd}{{\catname{Grpd}}}
\newcommand{\Ab}{{\catname{Ab}}}
\newcommand{\Ring}{{\catname{Ring}}}
\newcommand{\CRing}{{\catname{CRing}}}
\newcommand{\Mod}{{\mhyphen\catname{Mod}}}
\newcommand{\Alg}{{\mhyphen\catname{Alg}}}
\newcommand{\Field}{{\catname{Field}}}
\newcommand{\Vect}{{\catname{Vect}}}
\newcommand{\Hilb}{{\catname{Hilb}}}
\newcommand{\Ch}{{\catname{Ch}}}

\newcommand{\Hom}{{\mathrm{Hom}}}
\newcommand{\End}{{\mathrm{End}}}
\newcommand{\Aut}{{\mathrm{Aut}}}
\newcommand{\Obj}{{\mathrm{Obj}}}
\newcommand{\op}{{\mathrm{op}}}

% Norms, inner products
\delimitershortfall=-1sp
\newcommand{\widecdot}{\, \cdot \,}
\newcommand\emptyarg{{}\cdot{}}
\DeclarePairedDelimiterX{\norm}[1]{\Vert}{\Vert}{\ifblank{#1}{\emptyarg}{#1}}
\DeclarePairedDelimiterX{\abs}[1]\vert\vert{\ifblank{#1}{\emptyarg}{#1}}
\DeclarePairedDelimiterX\inn[1]\langle\rangle{\ifblank{#1}{\emptyarg,\emptyarg}{#1}}
\DeclarePairedDelimiterX\cur[1]\{\}{\ifblank{#1}{\emptyarg,\emptyarg}{#1}}
\DeclarePairedDelimiterX\pa[1](){\ifblank{#1}{\emptyarg}{#1}}
\DeclarePairedDelimiterX\brak[1][]{\ifblank{#1}{\emptyarg}{#1}}
\DeclarePairedDelimiterX{\an}[1]\langle\rangle{\ifblank{#1}{\emptyarg}{#1}}
\DeclarePairedDelimiterX{\bra}[1]\langle\vert{\ifblank{#1}{\emptyarg}{#1}}
\DeclarePairedDelimiterX{\ket}[1]\vert\rangle{\ifblank{#1}{\emptyarg}{#1}}

% mathmode text operators
\RedeclareMathOperator{\Re}{\operatorname{Re}}		% Real part
\RedeclareMathOperator{\Im}{\operatorname{Im}}		% Imaginary part
\DeclareMathOperator{\Stab}{\mathrm{Stab}}
\DeclareMathOperator{\Orb}{\mathrm{Orb}}
\DeclareMathOperator{\Id}{\mathrm{Id}}
\DeclareMathOperator{\vspan}{\mathrm{span}}			% Vector span
\DeclareMathOperator{\tr}{\mathrm{tr}}
\DeclareMathOperator{\adj}{\mathrm{adj}}
\DeclareMathOperator{\diag}{\mathrm{diag}}
\DeclareMathOperator{\eq}{\mathrm{eq}}
\DeclareMathOperator{\coeq}{\mathrm{coeq}}
\DeclareMathOperator{\coker}{\mathrm{coker}}
\DeclareMathOperator{\dom}{\mathrm{dom}}
\DeclareMathOperator{\cod}{\mathrm{codom}}
\DeclareMathOperator{\im}{\mathrm{im}}
\DeclareMathOperator{\Dim}{\mathrm{dim}}
\DeclareMathOperator{\codim}{\mathrm{codim}}
\DeclareMathOperator{\Sym}{\mathrm{Sym}}
\DeclareMathOperator{\lcm}{\mathrm{lcm}}
\DeclareMathOperator{\Inn}{\mathrm{Inn}}
\DeclareMathOperator{\sgn}{sgn}						% sgn operator
\DeclareMathOperator{\intr}{\text{int}}             % Interior
\DeclareMathOperator{\co}{\mathrm{co}}				% dual/convex Hull
\DeclareMathOperator{\Ann}{\mathrm{Ann}}
\DeclareMathOperator{\Tor}{\mathrm{Tor}}


% misc symbols
\newcommand{\divides}{\big\lvert}
\newcommand{\grad}{\nabla}
\newcommand{\veps}{\varepsilon}						% Preferred epsilon
\newcommand{\vphi}{\varphi}
\newcommand{\del}{\partial}							% Differential/Boundary
\renewcommand{\emptyset}{\text{\O}}					% Traditional emptyset symbol
\newcommand{\tril}{\triangleleft}					% Quandle operation
\newcommand{\nabt}{\widetilde{\nabla}}				% Contravariant derivative
\newcommand{\later}{$\textcolor{red}{\blacksquare}$}% Laziness indicator

% misc
\mathchardef\mhyphen="2D							% mathomode hyphen
\renewcommand{\mod}[1]{\ (\mathrm{mod}\ #1)}
\renewcommand{\bar}[1]{\overline{#1}}				% Closure/conjugate
\renewcommand\qedsymbol{$\blacksquare$} 			% Changes default qed in proof environment
%%%%% raised chi
\DeclareRobustCommand{\rchi}{{\mathpalette\irchi\relax}}
\newcommand{\irchi}[2]{\raisebox{\depth}{$#1\chi$}}
\newcommand\concat{+\kern-1.3ex+\kern0.8ex}

% Arrows
\newcommand{\weak}{\rightharpoonup}					% Weak convergence
\newcommand{\weakstar}{\overset{*}{\rightharpoonup}}% Weak-star convergence
\newcommand{\inclusion}{\hookrightarrow}			% Inclusion/injective map
\renewcommand{\natural}{\twoheadrightarrow}				% Natural map

% Environments
\theoremstyle{plain}
\newtheorem{thm}{Theorem}[section]
%\newtheorem{lem}[thm]{Lemma}
\newtheorem{lem}{Lemma}
\newtheorem*{lems}{Lemma}
\newtheorem{cor}[thm]{Corollary}
\newtheorem{prop}{Proposition}
\newtheorem*{claim}{Claim}
\newtheorem*{cors}{Corollary}
\newtheorem*{props}{Proposition}
\newtheorem*{conj}{Conjecture}

\theoremstyle{definition}
\newtheorem{defn}{Definition}[section]
\newtheorem*{defns}{Definition}
\newtheorem{exm}{Example}[section]
\newtheorem{exer}{Exercise}[section]

\theoremstyle{remark}
\newtheorem*{rem}{Remark}

\newtheorem*{solnx}{Solution}
\newenvironment{soln}
    {\pushQED{\qed}\renewcommand{\qedsymbol}{$\Diamond$}\solnx}
    {\popQED\endsolnx}%

% Macros
\newcommand{\restr}[1]{_{\mkern 1mu \vrule height 2ex\mkern2mu #1}}
\newcommand{\Upushout}[5]{
    \begin{tikzcd}[ampersand replacement = \&]
    \&#2\ar[rd,"\iota_{#2}"]\ar[rrd,bend left,"f"]\&\&\\
    #1\ar[ur,"#4"]\ar[dr,"#5"]\&\&#2\oplus_{#1} #3\ar[r,dashed,"\vphi"]\&Z\\
    \&#3\ar[ur,"\iota_{#3}"']\ar[rru,bend right,"g"']\&\&
    \end{tikzcd}
}
\newcommand{\exactshort}[5]{
		\begin{tikzcd}[ampersand replacement = \&]
			0\ar[r]\&#1\ar[r,"#2"]\& #3 \ar[r,"#4"]\& #5 \ar[r]\&0
		\end{tikzcd}
}
\newcommand{\product}[6]{
		\begin{tikzcd}[ampersand replacement = \&]
			#1 \& #2 \ar[l,"#4"'] \\
			#3 \ar[u,"#5"] \ar[ur,"#6"']
		\end{tikzcd}
}
\newcommand{\coproduct}[6]{
		\begin{tikzcd}[ampersand replacement = \&]
			#1 \ar[r,"#4"] \ar[d,"#5"'] \& #2 \ar[dl,"#6"] \\
			#3
		\end{tikzcd}
}
%%%%%%%%%%%% PAGE FORMATTING %%%%%%%%%

\usepackage{geometry}
    \geometry{
		left=15mm,
		right=15mm,
		top=15mm,
		bottom=15mm	
		}

\usepackage{color} % to do: change to xcolor
\usepackage{listings}
\lstset{
    basicstyle=\ttfamily,columns=fullflexible,keepspaces=true
}
\usepackage{setspace}
\usepackage{setspace}
\usepackage{mdframed}
\usepackage{booktabs}
\usepackage[document]{ragged2e}
\usepackage{epsfig}
\usepackage{dynkin-diagrams}

\pagestyle{fancy}{
	\fancyhead[L]{Fall 2024}
	\fancyhead[C]{MAT1344F}
	\fancyhead[R]{John White}
  
  \fancyfoot[R]{\footnotesize Page \thepage \ of \pageref{LastPage}}
	\fancyfoot[C]{}
	}
\fancypagestyle{firststyle}{
     \fancyhead[L]{}
     \fancyhead[R]{}
     \fancyhead[C]{}
     \renewcommand{\headrulewidth}{0pt}
	\fancyfoot[R]{\footnotesize Page \thepage \ of \pageref{LastPage}}
}
\newcommand{\pmat}[4]{\begin{pmatrix} #1 & #2 \\ #3 & #4 \end{pmatrix}}
\newcommand{\A}{\mathbb{A}}
\newcommand{\B}{\mathbb{B}}
\newcommand{\fin}{``\in"}
\newcommand{\mk}[1]{\mathfrak{#1}}
\newcommand{\g}{\mk{g}}
\newcommand{\h}{\mk{h}}
\newcommand{\tphi}{\tilde{\phi}}
\DeclareMathOperator{\Perm}{Perm}
\DeclareMathOperator{\pdim}{pdim}
\DeclareMathOperator{\gldim}{gldim}
\DeclareMathOperator{\lgldim}{lgldim}
\DeclareMathOperator{\rgldim}{rgldim}
\DeclareMathOperator{\idim}{idim}
\DeclareMathOperator{\SU}{SU}
\DeclareMathOperator{\SO}{SO}
\DeclareMathOperator{\Ad}{Ad}
\DeclareMathOperator{\ad}{ad}
\DeclareMathOperator{\gr}{gr}
\newcommand{\Rmod}{R-\text{mod}}
\newcommand{\RMod}{R-\text{Mod}}
\newcommand{\onto}{\twoheadrightarrow}
\newcommand{\into}{\hookrightarrow}
\newcommand{\barf}{\bar{f}}
\newcommand{\dd}[2]{\frac{d#1}{d#2}}
\newcommand{\pp}[2]{\frac{\partial #1}{\partial #2}}
\newcommand{\gl}{\mk{g}\mk{l}}
\renewcommand{\P}{\mathbb{P}}
\renewcommand{\E}{\mathbb{E}}
\DeclareMathOperator{\Ext}{Ext}
\DeclareMathOperator{\Rank}{Rank}
\DeclareMathOperator{\Sp}{Sp}
\DeclareMathOperator{\ann}{ann}
\DeclareMathOperator{\Lag}{Lag}
\DeclareMathOperator{\Riem}{Riem}

\newcommand{\exactlon}[5]{
		\begin{tikzcd}
			0\ar[r]&#1\ar[r,"#2"]& #3 \ar[r,"#4"]& #5 \ar[r]&0
		\end{tikzcd}
}

\title{MAT 1344}
\author{John White}
\date{Fall 2024}


\begin{document}

\section*{Lecture 1 - 3/5/24}

A good starting point is Newton's equation $V(q_1, \dots, q_n)$ for a particle: 
\[
m\ddot{q_i} = -\pp{V}{q_i}
\]

The first observation is that if energy is 
\[
E = \frac{m}{2}\dot{q}^2 + V(q)
\]
then $E$ is constant along solution curves (take the $t$ derivative). 

A classic physics trick is to reduce $n$th order to first order by letting higher derivatives be introduced as new variables. Introduce $p_i = m\dot{q}_i$. We have the equations
\begin{align*}
\dot{q_i} & = \frac{1}{m}p_i  & \dot{p_i} = -\pp{V}{q_i}
\end{align*}
The energy becomes ``Hamiltonian." 
\[
H(q, p) = \frac{1}{2m}\sum_{i=1}^n p_i^2 + V(q)
\]

We can write these equations from earlier quite nicely in terms of the Hamiltonian (if you know the potential you know the Hamiltonian, and vice-versa) as 
\begin{align*}
\dot{q_i} & = \pp{H}{p_i},  & \dot{p_i} = -\pp{H}{q_i}
\end{align*}

Hamilton's equation. This looks similar to $\dot{X_i} = -\pp{V}{x_i}$, the equation of a gradient flow. 

One advantage of these Hamiltonian equations is that we have \underline{\underline{lots}} of symmetry, i.e. for coordinate changes 
\begin{align*}
\tilde{q_i} = f_i(q, p), \,\,\,& \tilde{p_i} = g_i(q, p); & \tilde{H}(\tilde{q},\tilde{p}) = H(q, p)
\end{align*}
then in new coordinates, $\dot{\tilde{q_i}} = \pp{\tilde{H}}{\tilde{p_i}}, \dot{\tilde{p_i}} = -\pp{\tilde{H}}{\tilde{q_i}}$

\exm

$\tilde{p_i} = -q_i, \tilde{q_i} = p_i$

\exm

$\tilde{q_i} = q_i, \tilde{p_i} = p_i + \varphi_i(q_1, \dots, q_n)$

We think of this Hamiltonian as having a very large infintie dimensional symmetry group, in contrast to the earlier graident flow, which has a very small symmetry group. 

A \underline{Hamiltonian vector field} 
\[
X_H = \sum_{i=1}^n \left(\pp{H}{q_i}\pp{}{p_i} - \pp{H}{p_i}\pp{}{q_i}\right)
\]

If we take the exterior derivative, 
\[
dH = \sum_{i=1}^n\left(\pp{H}{q_i}dq_i + \pp{H}{p_i}dp_i\right)
\]
We can write Hamilton's equation as $\iota(X_H)\omega = - dH$, with $\omega = \sum_{i=1}^ndq_i\wedge dp_i$ where $\iota$ means contraction. 

Often we take this equation as the definition of the Hamiltonian vector field, which defines a differential equation, which defines a flow, et cetera. 

\underline{Remark:} The word ``Symplectic" was introduced by Hermann Weyl in the theory of Lie groups. 

Symplectic manifolds were introduced by Charles Ehrsmann and Paulette Libermann around 1948. 

In the 60s and 70s, Souriau, Kostant (SP?), others did more work such as trying to phrase classical mechanics in this language. Many others, such as Arnold, Thurston, who showed that there symplectic and complex manifolds are not the same. Arnold initiated a program of symplectic topology in 74(?). Weinstein, Steinberg, Guillemain...

\section*{\underline{Part 1: Symplectic Linear Algebra}}

\defn

A \underline{symplectic structure} on a finite dimensional (real for now) vector space $E$ is a bilinear form
\[
\omega:E\times E \to \R
\]
which is
\begin{enumerate}[label=(\roman*)]

\item Skew-symmetric, meaning $\omega(v, w) = \omega(w, v)$

\item Nondegenerate, meaning $\ker \omega \eqdef \{v \in E \mid \omega(v, w) = 0$ for all $w \in E\}$ is trivial. (Every vector has a friend).

\end{enumerate}

In terms of $\omega^\flat:E\to E^*$, $v \mapsto \omega(v, \cdot)$. 

Skew-symmetry means $(\omega^\flat)* = -\omega^\flat$. 

Non-degeneracy means $\ker(\omega^\flat) = 0$

\exm
\,

\begin{enumerate}

\item ``Standard symplectic structure" For $E = \R^{2n}$ with basis $e_1, \dots, e_n, f_1, \dots, f_n$, if we set 
\begin{align*}
\omega(e_i,e_j) = 0,\omega(f_i,f_j)=0,\,&\omega(e_i,f_j)=\delta_{ij}
\end{align*}

\item For $V$ any finite dimensional vector space, setting $E = V \oplus V^*$, and 
\[
\omega((v_i,\alpha_i),(v_2,\alpha_2)) = \langle\alpha_1,v_2\rangle - \langle \alpha_2,v_1\rangle
\]

\item If $V$ is any finite-dimensional \underline{complex} inner product space $h:V\times V \to \C$. 

If we take $E = V$, $\omega(v, w) = \Im(h(v, w))$ is symplectic.

\end{enumerate}

Note that these three are all actually the same example. 

\defn

A \underline{symplectomorphism} between symplectic vector spaces $(E_i, \omega_i)$, $(i = 1, 2)$ is a linear isomorphism $A: E_1\to E_2$, such that
\[
\omega_2(Av,Aw) = \omega_1(v,w)
\]
for all $v, w \in E_1$ (I.e $\omega_1 = A^*\omega_2$).

\underline{Remark:} stipulating that it is an isomorphism is a little overkill, because anything satisfying the second condition is injective. 

Symplectomorphisms of $(E,\omega)$ to itself are denoted $\Sp(E,\omega)$, and is called the 

\underline{symplectic group}.

In this sense, it is easy to see that those three examples are all symplectomorphic. 


\section*{Lecture 2 - 9/10/24}

\subsection*{\underline{Subspace of symplectic vector space}}

Let $F \subseteq E$. Define $F^\omega = \{v\in E \mid \omega(v, w) = 0$ for all $w \in F\}$, the ``$w$-orthogonal" space. 

In terms of $\ann(F) = \{\alpha\in E^* \mid \alpha(v) = 0$ for all $v \in F\}$

Note that $\omega^\flat:F^\omega \to \ann(F)$ is an isomorphism.

\prop
\,
\begin{itemize}

\item $\dim F^\omega = \dim E - \dim F$

\item $(F^\omega)^\omega = F$, $(F_1 \cap F_2)^\omega = F_1^\omega + F_2^\omega$, $(F_1 + F_2)^\omega = (F_1)^\omega \cap (F_2)^\omega$

\end{itemize}

\proof
\,
\begin{itemize}

\item $\dim F^\omega = \dim(\ann(F)) = \cdots$

\item Since elements of $F$ are orthogonal to elements of $F^\omega$, we have $F\subseteq (F^\omega)^\omega;$ by dimension count have equality. Etc

\end{itemize}

\defn

A subspace $F \subseteq E$ is called
\begin{itemize}

\item \underline{isotropic} if $F \subseteq F^\omega$

\item \underline{coisotropic} if $F^\omega \subseteq F$

\item \underline{Lagrangian} if $F^\omega = F$. 

\end{itemize}

Note $F$ is isotropic if and only if $\omega|_{F\times F} = 0$

Note: 

If $F$ is isotropic, then $\dim F \leq \frac12\dim E$

If $F$ is coisotropic, then $\dim F \geq \frac12\dim E$

In both cases, if equality holds, $F$ is Lagrangian. 

So if $F$ is Lagrangian, then $\dim F = \frac12 \dim E$.

\defn The set of Lagrangian subspaces is denoted $\Lag(E,\omega)$, called the ``\underline{Lagrangian Grassmannian}". 

\prop
\,
\begin{enumerate}[label=(\alph*)]

\item $\Lag(E, \omega) \neq \varnothing$

\item For every $M \in \Lag(E, \omega)$, there exists $L \in \Lag(E,\omega)$ with $L \cap M = \{0\}$ (i.e. $E = L \oplus M$).

\end{enumerate}

\proof
\,
\begin{enumerate}[label=(\alph*)]

\item By induction: Suppose $F \subseteq E$ is isotropic. If $F$ is Lagrangian, we're done. Otherwise, $F \subset F^\omega$ is a proper subspace. Pick $v \in F^\omega \setminus F$. Then $F' = F + \operatorname{Span}\{v\}$ is again isotropic. The process ends when it becomes Lagrangian.

\item By induction: suppose $F \subseteq E$ is isotropic, with $F\cap M = \{0\}$. If $F$ is Lagrangian, we are done. Otherwise, $F + (F^\omega \cap M) \subseteq F^\omega$ is an isotropic subspace, hence is a proper subspace of $F^\omega$. Pick $v \in F^\omega\setminus (F + (F^\omega \cap M))$; $F' = F + span\{v\}$

\claim $F' \cap M = \{0\}.$

\proof

Indeed: if $y \in F'\cap M$, write $y = x + tv$, $x \in F, t \in \R$. Then
\[
tv = y - x \in (F + M) \cap F^\omega = F + (F^\omega \cap F)
\]

\qed

\end{enumerate}

\qed

\underline{Exercise:} Given $L \in \Lag(E, \omega)$. Let $F \subseteq E$ be any complement, i.e. $E = L \oplus F$. 
\begin{enumerate}[label=(\roman*)]

\item Show that there exists a unique linear map $A:F\to L$ such that $F^\omega = \{v + Av\mid v \in F\}$

\item Show that all $F_t = \{v + tAv \mid v \in F\}$ is a complement to $L$.

\item $F_{\frac12}$ is Lagrangian

\end{enumerate}

Given $(E, \omega)$, we can choose a Lagrangian splitting $E = L \oplus M$. 

\prop The choice of splitting identifies $M\cong L^*$ and determines a symplectomorphism 
\[
E\to L\oplus L^*
\]

\proof

Every $w \in M$ defines a linear functional $\alpha_\omega \in L^*$, $\alpha_\omega(v) = \omega(v, w)$

The map $M \to L^*$, $w \mapsto \alpha_w$ is an isomorphism, using the non-degeneracy of the symplectic structure. The resulting map $E \cong L \oplus M \to L \oplus L^*$ is a symplectomorphism (by formula for sympletic structure on $L \oplus L^*$).

\qed

\prop

For every symplectic $(E, \omega)$, there exists a symplectomorphism $E \to \R^{2n}$, where $\R^{2n}$ has the standard symplectic structure. 

\proof

Choose a Lagrangian splitting $E = L \oplus L^*$. Now pick basis of $L, $ dual basis of $L^*$ to identify $E \cong \R^{2n}$

\underline{Remark:}

$\Lag(\R^2) = \RP(1) \cong S^1$

$\Lag(\R^4) = ?$

\underline{Exercise:}

Given $L \in \Lag(E,\omega)$, show that the set of all 
\begin{itemize}

\item complement to $L$ is an affine space with corresponding linear space $\Hom(E/L,L)$ (note if $L$ is lagrangian, then $E/L$ is naturally identified with $L^*$).

\item Lagrangian complements to $L$ is an affine space with corresponding linear space the self-adjoint maps $L^* \to L$.

\end{itemize}

In general, $\dim \Lag(\R^{2n}) = \frac{n(n + 1)}{2}$

\subsection*{\underline{Linear Reduction:}}

Let $(E, \omega)$ be symplectic. $F \subseteq E$ is \underline{symplectic} if $\omega|_{F\times F}$ is nondegenerate. Equivalently, $\underbrace{F\cap F^\omega}_{=\ker \omega|_{F\times F}} = \{0\}$

Note that $F$ is symplectic if and only if $F^\omega$, and $E = F \oplus F^\omega$.

In general, if $F$ is not symplectic, we can make it symplectic by quotienting by $\ker(\omega|_{F\times F}) = F\cap F^\omega$. 

\prop

For any subspace $F$, the quotient $E_F = F/(F\cap F^\omega)$ inherits a symplectic structure: 
\[
\omega_F(\pi(v), \pi(w)) = \omega(v, w)
\]
where $\pi$ is the quotient map $\pi:F\to F/(F\cap F^\omega)$

\proof

It's well defined: E.g, if $\pi(v) = 0$, then $v \in F \cap F^\omega$, so $\omega(v, w) = 0$ for all $w\in F$. 

It's non-degenerate: If $\pi(v)\in \ker(\omega_F)$, then $\omega(v, w) = 0$ for all $w \in F$, so $v \in F \cap F^\omega$, so $\pi(v) = 0$. 

Note: For $F$ coisotropic, $E_F = F/F^\omega$

\prop

For $F$ coisotropic, $L \subseteq E$ Lagrangian, the subspace $\pi(L\cap F) = L_F$ is again Lagrangian.

\proof

Clearly, $L_F$ is isotropic. To show $L_F$ is Lagrangian, count dimension: $(L_F) = (L \cap F)/(L \cap F^\omega)$.
\begin{align*}
\dim (L \cap F^\omega) & = \dim E - \dim (L \cap F^\omega)^\omega \\
& = \dim E - \dim (L + F) \\
& = \underbrace{\dim E}_{2\dim L} - \dim L - \dim F + \dim(L\cap F)\\
& = \dim L - \dim F + \dim (L \cap F)
\end{align*}

So 
\begin{align*}
\dim (L_F) & = \dim(L\cap F) - \dim(L\cap F^\omega) \\
& = \dim F - \dim L \\
\dim E_f & = \dim F - \dim F^\omega \\& = 2\dim F - \dim E \\ & = 2(\dim F - \dim L) 
\end{align*}

\qed

So, we have constructed a map $\Lag(E, \omega) \to \Lag(E_F, \omega_F), L \mapsto L_F$

\underline{Warning:} This map is not continuous!

It is discontinuous at the set of $L$'s where $L, F$ are not transverse. Away from this set, it's smooth. 

\underline{Exercise:}

Let $E = \R^4$ with standard symplectic basis. Take 
\[
F = \operatorname{span}\{e_1, e_2, f_1\}
\]
Then $F^\omega = \operatorname{span}\{e_2\}$. 

$F/F^\omega \cong \R^2 = \operatorname{span}\{e_1, f_1\}$. 

Let $L_t = \operatorname{span}\{e_1 + tf_2, e_2 + tf_1\}$. 
\begin{enumerate}[label=(\alph*)]

\item Check $L_t$ are Lagrangian

\item Compute $(L_t)_F \subseteq \R^2$ and find it's discontinuous at $t = 0$. 

\end{enumerate}

\subsection*{\underline{Compatible complex structures}}

\underline{Recall:} 

Given a complex vector space $V$, we can always regard it as a real vector space of twice the dimension. ``Multiplication by $\sqrt{-1}"$ becomes a real linear transformation $\mathcal{J} \in \Hom_{\R}(V,V)$, $\mathcal{J}^2 = -I$. 

Conversely, a real vector space with such a $\mathcal{J}$ is called a \underline{complex structure}, and we can imbue it with complex multiplication by defining
\[
(a + ib)v = av + b(\mathcal{J}v)
\]

\defn

Let $(E, \omega)$ be a symplectic vector space. A complex structure $\mathcal{J}$ (meaning $\mathcal{J}^2 = -I)$ is \underline{$\omega$-compatible} if 
\[
g(v, w) \eqdef \omega(v, \mathcal{J}w)
\]
defines an inner product. Denote by $\mathcal{J}(E, \omega) \eqdef \{\omega-$compatible complex structure $\}$.

Given $j \in \mathcal{J}(E, \omega)$, we get a complex inner product by 
\[
h(v, w) \eqdef g(v, w) + \sqrt{-1}\omega(v, w)
\]

\underline{Remark:} $\mathcal{J} \in \mathcal{J}(E,\omega)$ is a symplectomorphism: 
\begin{align*}
\omega(\mathcal{J}v, \mathcal{J}w) & = g(\mathcal{J}v, w) \\
& = g(w, \mc{J}v) \\
& = \omega(w, \mc{J}^2v) \\
& = -\omega(w, v)\\
& = \omega(v, w) \\
\end{align*}

\underline{Remark:} For $\R^{2n}$, there is a standard complex structure given by $\mc{J}(e_i) = f_i, \mc{J}(f_i) = -e_i$.

This identifies $\R^{2n} \cong \C^n$. We can come up with more complex structures by picking an $A \in \Sp(E, \omega)$ and considering $\mc{J} \mapsto A\mc{J}A^{-1}$. 

We have a map $\mc{J}(E,\omega) \to \operatorname{Riem}(E)$ (real inner products) given by $\mc{J} \mapsto g$, where $g$ is as above. 

There is a canonical left inverse $\varphi:\operatorname{Riem}(E) \to \mc{J}(E,\omega)$ as follows: 

\prop There is a canonical retraction (in the sense of topology) from $\operatorname{Riem}(E) \to \mc{J}(E, \omega)$.

\proof

Given $k \in \Riem(E)$, define $A \in \GL(E)$ by $k(v, w) = \omega(v, Aw)$. 

$A$ is not a complex structure in general, but it's skew-symmetric with respect to $h$: 
\[
A^T = -A
\]
Define $|A| = (A^TA)^\frac12 = (-A^2)^\frac12$. This commutes with $A$ by functional calculus, and define $\mc{J} = A|A|^{-1}$. This will do the job. 

\qed


































\end{document}
