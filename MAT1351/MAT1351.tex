
\documentclass[x11names,reqno,14pt]{extarticle}
\input{preamble}
\usepackage[document]{ragged2e}
\usepackage{epsfig}
\usepackage{dynkin-diagrams}

\pagestyle{fancy}{
	\fancyhead[L]{Fall 2024}
	\fancyhead[C]{MAT1344F}
	\fancyhead[R]{John White}
  
  \fancyfoot[R]{\footnotesize Page \thepage \ of \pageref{LastPage}}
	\fancyfoot[C]{}
	}
\fancypagestyle{firststyle}{
     \fancyhead[L]{}
     \fancyhead[R]{}
     \fancyhead[C]{}
     \renewcommand{\headrulewidth}{0pt}
	\fancyfoot[R]{\footnotesize Page \thepage \ of \pageref{LastPage}}
}
\newcommand{\pmat}[4]{\begin{pmatrix} #1 & #2 \\ #3 & #4 \end{pmatrix}}
\newcommand{\A}{\mathbb{A}}
\newcommand{\B}{\mathbb{B}}
\newcommand{\fin}{``\in"}
\newcommand{\mk}[1]{\mathfrak{#1}}
\newcommand{\g}{\mk{g}}
\newcommand{\h}{\mk{h}}
\newcommand{\tphi}{\tilde{\phi}}
\DeclareMathOperator{\Perm}{Perm}
\DeclareMathOperator{\pdim}{pdim}
\DeclareMathOperator{\gldim}{gldim}
\DeclareMathOperator{\lgldim}{lgldim}
\DeclareMathOperator{\rgldim}{rgldim}
\DeclareMathOperator{\idim}{idim}
\DeclareMathOperator{\SU}{SU}
\DeclareMathOperator{\SO}{SO}
\DeclareMathOperator{\Ad}{Ad}
\DeclareMathOperator{\ad}{ad}
\DeclareMathOperator{\gr}{gr}
\newcommand{\Rmod}{R-\text{mod}}
\newcommand{\RMod}{R-\text{Mod}}
\newcommand{\onto}{\twoheadrightarrow}
\newcommand{\into}{\hookrightarrow}
\newcommand{\barf}{\bar{f}}
\newcommand{\dd}[2]{\frac{d#1}{d#2}}
\newcommand{\pp}[2]{\frac{\partial #1}{\partial #2}}
\newcommand{\gl}{\mk{g}\mk{l}}
\renewcommand{\P}{\mathbb{P}}
\renewcommand{\E}{\mathbb{E}}
\DeclareMathOperator{\Ext}{Ext}
\DeclareMathOperator{\Rank}{Rank}
\DeclareMathOperator{\Sp}{Sp}

\newcommand{\exactlon}[5]{
		\begin{tikzcd}
			0\ar[r]&#1\ar[r,"#2"]& #3 \ar[r,"#4"]& #5 \ar[r]&0
		\end{tikzcd}
}

\title{MAT 1344}
\author{John White}
\date{Fall 2024}


\begin{document}

\section*{Lecture 2 - 3/5/24}

\underline{Let's try setting notation}

\underline{Notation:}

We denote by $[X,Y]$ the homotopy classes of maps from $X \to Y$. 

We denote by $[(X,A),(Y,B)]$ the homotopy classes of maps relative to $A$ $f:X\to Y$ such that $f(A)\subseteq B$

\defn

The \underline{$n$th homotopy group of $X$ based at $x$}, $\pi_n(X, x)$, is defined as 
\[
\pi_n(X,x) \eqdef \underbrace{[(I^n,\partial I^n), (X, x_0)]}_{\text{more useful for technical lemmas}} \cong \underbrace{[(S^n,*),(X,*)]}_{\text{more useful for intuition}}
\]

\lem

For $n \geq 2$, $\pi_n(X)$ is Abelian. 

\proof

\includegraphics[scale=.2]{1351P1}

\qed

\defn

The \underline{relative homotopy group} $\pi_n(X, A, x_0)$ is
\[
[(I^n, \partial I^n, \partial I^n\setminus I^{n-1}),(X,A,x_0)] \cong [(D^n, \partial D^n, *),(X,A,x_0)]
\]

\underline{Exercise:}

Show that for $n \geq 3$, this is an Abelian group. 

INSERT PICTURE HERE

We have a long exact sequence for homotopy groups:
\[
\begin{tikzcd}
\cdots \ar[r] & \pi_n(A) \ar[r, "i_*"] & \pi_n(X) \ar[r, "tautological"] & \ar[r, "restriction"] \pi_n(X,A) \ar[r] & \pi_{n-1}(A) \ar[r] & \cdots 
\end{tikzcd}
\]

\underline{Exercise:} Convince yourself its exact at $\pi_n(X)$. 

\defn

The \underline{Hurewicz homomorphism $H:\pi_n(X)\to H_n(X)$, $\pi_n(X,A) \to H_n(X, A)$} is defined as follows: 

Given $f:(I,\del I^n) \to (X, A)$, define a singular cycle using a homeomorphism from $I^n$ to the $n$-simplex. 

\underline{Exercise:} Check that this is a homomorphism. 

\thm

$H:\pi_n(S^n) \to H_n(S^n)$ is an isomorphism, and therefore $\pi_n(S^n) \cong \Z$. 

\proof

To show it's onto, we need a map of degree $d$. To do this, pick $d$ little balls inside the sphere, map each onto the whole sphere, and map everything outside these balls to the basepoint. No problems arise so long as we correctly consider orientation. 

Now we have to show it's injective. We do this by showing every map of degree 0 is nullhomotopic. 

Take $f:S^n\to S^n$ of degree 0. By some homotopy, we can assume it is smooth. Pick a regular value, and take the preimage of a ball around it. Its preimage is a bunch of balls with local degrees which cancel out. 

Next, take a homotopy $H:S^n\times[0,1]\to S^n$ which spreads the ball around the regular point around the entire sphere, sending everything outside it to the basepoint. Call this a homotopy from $\Id$ to $H_1$. 

Now consider $H \circ (g\times \Id_{[0,1]})$, which is a homotopy from $g$ to $H_1\circ g$. 

Now we have a map $H_1\circ g$ sending everything but these discs in the domain to the basepoint. 

Then, consulting the picture, we connect the balls using spaghettis. 

\includegraphics[scale=.2]{1351P2}
\qed

\thm[Whitehead Theorem]

Let $X, Y$ be CW complexes, $Y$ connected. Let $f:X\to Y$ such that $f_*:\pi_n(X) \to \pi_n(Y)$ is an isomorphism for all $n \geq0$. 

Then $f$ is a homotopy equivalence. 

\proof
\,

\underline{note:}
\begin{enumerate}[label=(\roman*)]

\item It is not enough for $\pi_n(X), \pi_n(Y)$ to be abstractly isomorphic, there must be a map. For example, for every $n$, $\pi_n(S^2) \cong \pi_n(\CP^2\times S^3)$, but we know from their homology groups that they're not homotopy equivalent. 

\item They need to be CW complexes. For example,if we take the map from the point to the warsaw circle, this is an iso on all $\pi_n$, but the warsaw circle is not contractible. 

\end{enumerate}

We now proceed with the proof

\underline{Case 1:} 

We begin by showing that if $\pi_n(Y) \cong 0$ for all $n \geq 0$, then $X$ is contractible. The idea is that we contract skeleton by skeleton. 

\underline{Step 0:} Homotope $Y^{(0)}$ to the basepoint (since $Y$ is connected).

Recall that CW satisfy the \underline{homotopy extension property}, so a nullhomotopy of $Y^{(k)}$ is a map on $Y\times\{0\} \cup Y^{(k)} \times [0, 1]$ which extends to $Y\times[0,1]$. 

\underline{Step $k + 1$:} We've built a map $f_k:Y\to Y$ such that $Y^{(k)}$ goes to the basepoint. So every $(k + 1)$-cell becomes a map $(D^{k + 1}, S^k) \to (Y, y_0)$. Since $\pi_{k + 1}(Y) = 0$, we can contract the $(k + 1)$-skeleton and extend this map to the rest of $Y$. 

\underline{Step $\oo$:}

Do step $0$ on $[0,\frac12]$, step $1$ on $[\frac12,\frac34]$, et cetera. The image of every point is eventually constant, so the homotopy is continuous at 1 (exercise in using the weak topology).

\underline{Case 2:} Suppose $X \into Y$ is a subcomplex (and the inclusion induces $\cong$ on all $\pi_n$). Then

$\pi_n(Y, X)\cong 0$ by the LES from above. We can do the same thing except instead of contracting into a point, we deformation retract onto $X$.

\underline{Case 3:} General case. Let $X \to Y$ be some map. We know $Y \cong M_f = \frac{Y \coprod X\times[0,1]}{(x,1)\sim f(x)}$

This reduces the general case to case 2. 

\qed

\thm[Hurewicz Theorem] 

Let $X$ be a CW complex. If $\pi_k(X) \cong 0$ for $k < n$ (and $n\geq2$), then $H_k(X)\cong 0$ for $k \leq n$, and $H:\pi_n(X) \to H_n(X)$ is an isomorphism. 

\proof

Suppose by induction that $X^{(k-1)}$ is a point, $k < n$. We find a $Y \simeq X$ such that $Y^{(k)}$ is a point. 

\underline{Step 1:} Attach $(k + 1)$-cells to fill each $k$-cell $e_i$. For each $e_i$ there is a map $f_i:D^{k + 1} \to X$ such that $f_i|_{\del}$ is the inclusion of $e_i$.

These together form a map from $S^{k+1}$ along which we can attach a $(k + 2)$-cell. We get a space $Z$ which obviously deformation retracts to $X$. 

\underline{Step 2:} Go from $Z$ to $Y$. 

Inside $Z$ we have a subcomplex $A$ which is contractible, so $Z/A\simeq Z$. So set $Y = Z/A$.

By this ``cell-trading" method, we've shown the first part of the theorem (we have shown that $X$ is homotopy equivalent to something with a trivial $k$-skeleton, so by cellular homology the homology groups vanish. 

\includegraphics[scale=.2]{1351P3}

Now we need to show that if $X^{(n-1)}$ is a point, then $\pi_n(X) \cong H_n(X)$.

Surjectivity is clear (because every homology generator is represented by a sphere). 

For injectivity, we need to show that if a certain sum of $n$-cells is homologically trivial, then it is homotopically trivial. 

If it is homologically trivial, then there's a sum of $(n + 1)$-cells whose $\del$ is $\sum \alpha_ie_i$. 

UNTIL NEXT TIME. 




\end{document}




