
\documentclass[x11names,reqno,14pt]{extarticle}
% Choomno Moos
% Portland State University
% Choom@pdx.edu


%% stupid experiment %%
%%%%%%%%%%%%% PACKAGES %%%%%%%%%%%%%

%%%% SYMBOLS AND MATH %%%%
\let\oldvec\vec
\usepackage{authblk}	% author block customization
\usepackage{microtype}	% makes stuff look real nice
\usepackage{amssymb} 	% math symbols
\usepackage{siunitx} 	% for SI units, and the degree symbol
\usepackage{mathrsfs}	% provides script fonts like mathscr
\usepackage{mathtools}	% extension to amsmath, also loads amsmath
\usepackage{esint}		% extended set of integrals
\mathtoolsset{showonlyrefs} % equation numbers only shown when referenced
\usepackage{amsthm}		% theorem environments
\usepackage{relsize}	%font size commands
\usepackage{bm}			% provides bold math
\usepackage{bbm}		% for blackboard bold 1

%%%% FIGURES %%%%
\usepackage{graphicx} % for including pictures
\usepackage{float} % allows [H] option on figures, so that they appear where they are typed in code
\usepackage{caption}
\usepackage{hyperref}
%\usepackage{titling}
\usepackage{tikz} % for drawing
\usetikzlibrary{shapes,arrows,chains,positioning,cd,decorations.pathreplacing,decorations.markings,hobby,knots,braids}
\usepackage{subcaption}	% subfigure environment in figures

%%%% MISC %%%%
\usepackage{enumitem} % for lists and itemizations
\setlist[enumerate]{leftmargin=*,label=\bf \arabic*.}

\usepackage{multicol}
\usepackage{multirow}
\usepackage{url}
\usepackage[symbol]{footmisc}
\renewcommand{\thefootnote}{\fnsymbol{footnote}}
\usepackage{lastpage} % provides the total number of pages for the "X of LastPage" page numbering
\usepackage{fancyhdr}
\usepackage{manfnt}
\usepackage{nicefrac}
%\usepackage{fontspec}
%\usepackage{polyglossia}
%\setmainlanguage{english}
%\setotherlanguages{khmer}
%\newfontfamily\khmerfont[Script=Khmer]{Khmer Busra}

%%% Khmer script commands for math %%%
%\newcommand{\ka}{\text{\textkhmer{ក}}}
%\newcommand{\ko}{\text{\textkhmer{ត}}}
%\newcommand{\kha}{\text{\textkhmer{ខ}}}

%\usepackage[
%backend=biber,
% numeric
%style=numeric,
% APA
%bibstyle=apa,
%citestyle=authoryear,
%]{biblatex}

\usepackage[explicit]{titlesec}
%%%%%%%% SOME CODE FOR REDECLARING %%%%%%%%%%

\makeatletter
\newcommand\RedeclareMathOperator{%
	\@ifstar{\def\rmo@s{m}\rmo@redeclare}{\def\rmo@s{o}\rmo@redeclare}%
}
% this is taken from \renew@command
\newcommand\rmo@redeclare[2]{%
	\begingroup \escapechar\m@ne\xdef\@gtempa{{\string#1}}\endgroup
	\expandafter\@ifundefined\@gtempa
	{\@latex@error{\noexpand#1undefined}\@ehc}%
	\relax
	\expandafter\rmo@declmathop\rmo@s{#1}{#2}}
% This is just \@declmathop without \@ifdefinable
\newcommand\rmo@declmathop[3]{%
	\DeclareRobustCommand{#2}{\qopname\newmcodes@#1{#3}}%
}
\@onlypreamble\RedeclareMathOperator
\makeatother

\makeatletter
\newcommand*{\relrelbarsep}{.386ex}
\newcommand*{\relrelbar}{%
	\mathrel{%
		\mathpalette\@relrelbar\relrelbarsep
	}%
}
\newcommand*{\@relrelbar}[2]{%
	\raise#2\hbox to 0pt{$\m@th#1\relbar$\hss}%
	\lower#2\hbox{$\m@th#1\relbar$}%
}
\providecommand*{\rightrightarrowsfill@}{%
	\arrowfill@\relrelbar\relrelbar\rightrightarrows
}
\providecommand*{\leftleftarrowsfill@}{%
	\arrowfill@\leftleftarrows\relrelbar\relrelbar
}
\providecommand*{\xrightrightarrows}[2][]{%
	\ext@arrow 0359\rightrightarrowsfill@{#1}{#2}%
}
\providecommand*{\xleftleftarrows}[2][]{%
	\ext@arrow 3095\leftleftarrowsfill@{#1}{#2}%
}
\makeatother

%%%%%%%% NEW COMMANDS %%%%%%%%%%

% settings
\newcommand{\N}{\mathbb{N}}                     	% Natural numbers
\newcommand{\Z}{\mathbb{Z}}                     	% Integers
\newcommand{\Q}{\mathbb{Q}}                     	% Rationals
\newcommand{\R}{\mathbb{R}}                     	% Reals
\newcommand{\C}{\mathbb{C}}                     	% Complex numbers
\newcommand{\K}{\mathbb{K}}							% Scalars
\newcommand{\F}{\mathbb{F}}                     	% Arbitrary Field
\newcommand{\E}{\mathbb{E}}                     	% Euclidean topological space
\renewcommand{\H}{{\mathbb{H}}}                   	% Quaternions / Half space
\newcommand{\RP}{{\mathbb{RP}}}                       % Real projective space
\newcommand{\CP}{{\mathbb{CP}}}                       % Complex projective space
\newcommand{\Mat}{{\mathrm{Mat}}}						% Matrix ring
\newcommand{\M}{\mathcal{M}}
\newcommand{\GL}{{\mathrm{GL}}}
\newcommand{\SL}{{\mathrm{SL}}}

\newcommand{\tgl}{\mathfrak{gl}}
\newcommand{\tsl}{\mathfrak{sl}}                  % Lie algebras; i.e., tangent space of SO/SL/SU
\newcommand{\tso}{\mathfrak{so}}
\newcommand{\tsu}{\mathfrak{sl}}


% typography
\newcommand{\noi}{\noindent}						% Removes indent
\newcommand{\tbf}[1]{\textbf{#1}}					% Boldface
\newcommand{\mc}[1]{\mathcal{#1}}               	% Calligraphic
\newcommand{\ms}[1]{\mathscr{#1}}               	% Script
\newcommand{\mbb}[1]{\mathbb{#1}}               	% Blackboard bold


% (in)equalities
\newcommand{\eqdef}{\overset{\mathrm{def}}{=}}		% Definition equals
\newcommand{\sub}{\subseteq}						% Changes default symbol from proper to improper
\newcommand{\psub}{\subset}						% Preferred proper subset symbol

% Categories
\newcommand{\catname}[1]{{\text{\sffamily {#1}}}}

\newcommand{\Cat}{{\catname{C}}}
\newcommand{\cat}[1]{{\catname{\ifblank{#1}{C}{#1}}}}
\newcommand{\CAT}{{\catname{Cat}}}
\newcommand{\Set}{{\catname{Set}}}

\newcommand{\Top}{{\catname{Top}}}
\newcommand{\Met}{{\catname{Met}}}
\newcommand{\PL}{{\catname{PL}}}
\newcommand{\Man}{{\catname{Man}}}
\newcommand{\Diff}{{\catname{Diff}}}

\newcommand{\Grp}{{\catname{Grp}}}
\newcommand{\Grpd}{{\catname{Grpd}}}
\newcommand{\Ab}{{\catname{Ab}}}
\newcommand{\Ring}{{\catname{Ring}}}
\newcommand{\CRing}{{\catname{CRing}}}
\newcommand{\Mod}{{\mhyphen\catname{Mod}}}
\newcommand{\Alg}{{\mhyphen\catname{Alg}}}
\newcommand{\Field}{{\catname{Field}}}
\newcommand{\Vect}{{\catname{Vect}}}
\newcommand{\Hilb}{{\catname{Hilb}}}
\newcommand{\Ch}{{\catname{Ch}}}

\newcommand{\Hom}{{\mathrm{Hom}}}
\newcommand{\End}{{\mathrm{End}}}
\newcommand{\Aut}{{\mathrm{Aut}}}
\newcommand{\Obj}{{\mathrm{Obj}}}
\newcommand{\op}{{\mathrm{op}}}

% Norms, inner products
\delimitershortfall=-1sp
\newcommand{\widecdot}{\, \cdot \,}
\newcommand\emptyarg{{}\cdot{}}
\DeclarePairedDelimiterX{\norm}[1]{\Vert}{\Vert}{\ifblank{#1}{\emptyarg}{#1}}
\DeclarePairedDelimiterX{\abs}[1]\vert\vert{\ifblank{#1}{\emptyarg}{#1}}
\DeclarePairedDelimiterX\inn[1]\langle\rangle{\ifblank{#1}{\emptyarg,\emptyarg}{#1}}
\DeclarePairedDelimiterX\cur[1]\{\}{\ifblank{#1}{\emptyarg,\emptyarg}{#1}}
\DeclarePairedDelimiterX\pa[1](){\ifblank{#1}{\emptyarg}{#1}}
\DeclarePairedDelimiterX\brak[1][]{\ifblank{#1}{\emptyarg}{#1}}
\DeclarePairedDelimiterX{\an}[1]\langle\rangle{\ifblank{#1}{\emptyarg}{#1}}
\DeclarePairedDelimiterX{\bra}[1]\langle\vert{\ifblank{#1}{\emptyarg}{#1}}
\DeclarePairedDelimiterX{\ket}[1]\vert\rangle{\ifblank{#1}{\emptyarg}{#1}}

% mathmode text operators
\RedeclareMathOperator{\Re}{\operatorname{Re}}		% Real part
\RedeclareMathOperator{\Im}{\operatorname{Im}}		% Imaginary part
\DeclareMathOperator{\Stab}{\mathrm{Stab}}
\DeclareMathOperator{\Orb}{\mathrm{Orb}}
\DeclareMathOperator{\Id}{\mathrm{Id}}
\DeclareMathOperator{\vspan}{\mathrm{span}}			% Vector span
\DeclareMathOperator{\tr}{\mathrm{tr}}
\DeclareMathOperator{\adj}{\mathrm{adj}}
\DeclareMathOperator{\diag}{\mathrm{diag}}
\DeclareMathOperator{\eq}{\mathrm{eq}}
\DeclareMathOperator{\coeq}{\mathrm{coeq}}
\DeclareMathOperator{\coker}{\mathrm{coker}}
\DeclareMathOperator{\dom}{\mathrm{dom}}
\DeclareMathOperator{\cod}{\mathrm{codom}}
\DeclareMathOperator{\im}{\mathrm{im}}
\DeclareMathOperator{\Dim}{\mathrm{dim}}
\DeclareMathOperator{\codim}{\mathrm{codim}}
\DeclareMathOperator{\Sym}{\mathrm{Sym}}
\DeclareMathOperator{\lcm}{\mathrm{lcm}}
\DeclareMathOperator{\Inn}{\mathrm{Inn}}
\DeclareMathOperator{\sgn}{sgn}						% sgn operator
\DeclareMathOperator{\intr}{\text{int}}             % Interior
\DeclareMathOperator{\co}{\mathrm{co}}				% dual/convex Hull
\DeclareMathOperator{\Ann}{\mathrm{Ann}}
\DeclareMathOperator{\Tor}{\mathrm{Tor}}


% misc symbols
\newcommand{\divides}{\big\lvert}
\newcommand{\grad}{\nabla}
\newcommand{\veps}{\varepsilon}						% Preferred epsilon
\newcommand{\vphi}{\varphi}
\newcommand{\del}{\partial}							% Differential/Boundary
\renewcommand{\emptyset}{\text{\O}}					% Traditional emptyset symbol
\newcommand{\tril}{\triangleleft}					% Quandle operation
\newcommand{\nabt}{\widetilde{\nabla}}				% Contravariant derivative
\newcommand{\later}{$\textcolor{red}{\blacksquare}$}% Laziness indicator

% misc
\mathchardef\mhyphen="2D							% mathomode hyphen
\renewcommand{\mod}[1]{\ (\mathrm{mod}\ #1)}
\renewcommand{\bar}[1]{\overline{#1}}				% Closure/conjugate
\renewcommand\qedsymbol{$\blacksquare$} 			% Changes default qed in proof environment
%%%%% raised chi
\DeclareRobustCommand{\rchi}{{\mathpalette\irchi\relax}}
\newcommand{\irchi}[2]{\raisebox{\depth}{$#1\chi$}}
\newcommand\concat{+\kern-1.3ex+\kern0.8ex}

% Arrows
\newcommand{\weak}{\rightharpoonup}					% Weak convergence
\newcommand{\weakstar}{\overset{*}{\rightharpoonup}}% Weak-star convergence
\newcommand{\inclusion}{\hookrightarrow}			% Inclusion/injective map
\renewcommand{\natural}{\twoheadrightarrow}				% Natural map

% Environments
\theoremstyle{plain}
\newtheorem{thm}{Theorem}[section]
%\newtheorem{lem}[thm]{Lemma}
\newtheorem{lem}{Lemma}
\newtheorem*{lems}{Lemma}
\newtheorem{cor}[thm]{Corollary}
\newtheorem{prop}{Proposition}
\newtheorem*{claim}{Claim}
\newtheorem*{cors}{Corollary}
\newtheorem*{props}{Proposition}
\newtheorem*{conj}{Conjecture}

\theoremstyle{definition}
\newtheorem{defn}{Definition}[section]
\newtheorem*{defns}{Definition}
\newtheorem{exm}{Example}[section]
\newtheorem{exer}{Exercise}[section]

\theoremstyle{remark}
\newtheorem*{rem}{Remark}

\newtheorem*{solnx}{Solution}
\newenvironment{soln}
    {\pushQED{\qed}\renewcommand{\qedsymbol}{$\Diamond$}\solnx}
    {\popQED\endsolnx}%

% Macros
\newcommand{\restr}[1]{_{\mkern 1mu \vrule height 2ex\mkern2mu #1}}
\newcommand{\Upushout}[5]{
    \begin{tikzcd}[ampersand replacement = \&]
    \&#2\ar[rd,"\iota_{#2}"]\ar[rrd,bend left,"f"]\&\&\\
    #1\ar[ur,"#4"]\ar[dr,"#5"]\&\&#2\oplus_{#1} #3\ar[r,dashed,"\vphi"]\&Z\\
    \&#3\ar[ur,"\iota_{#3}"']\ar[rru,bend right,"g"']\&\&
    \end{tikzcd}
}
\newcommand{\exactshort}[5]{
		\begin{tikzcd}[ampersand replacement = \&]
			0\ar[r]\&#1\ar[r,"#2"]\& #3 \ar[r,"#4"]\& #5 \ar[r]\&0
		\end{tikzcd}
}
\newcommand{\product}[6]{
		\begin{tikzcd}[ampersand replacement = \&]
			#1 \& #2 \ar[l,"#4"'] \\
			#3 \ar[u,"#5"] \ar[ur,"#6"']
		\end{tikzcd}
}
\newcommand{\coproduct}[6]{
		\begin{tikzcd}[ampersand replacement = \&]
			#1 \ar[r,"#4"] \ar[d,"#5"'] \& #2 \ar[dl,"#6"] \\
			#3
		\end{tikzcd}
}
%%%%%%%%%%%% PAGE FORMATTING %%%%%%%%%

\usepackage{geometry}
    \geometry{
		left=15mm,
		right=15mm,
		top=15mm,
		bottom=15mm	
		}

\usepackage{color} % to do: change to xcolor
\usepackage{listings}
\lstset{
    basicstyle=\ttfamily,columns=fullflexible,keepspaces=true
}
\usepackage{setspace}
\usepackage{setspace}
\usepackage{mdframed}
\usepackage{booktabs}
\usepackage[document]{ragged2e}
\usepackage{epsfig}
\usepackage{dynkin-diagrams}

\pagestyle{fancy}{
	\fancyhead[L]{Fall 2024}
	\fancyhead[C]{MAT1344F}
	\fancyhead[R]{John White}
  
  \fancyfoot[R]{\footnotesize Page \thepage \ of \pageref{LastPage}}
	\fancyfoot[C]{}
	}
\fancypagestyle{firststyle}{
     \fancyhead[L]{}
     \fancyhead[R]{}
     \fancyhead[C]{}
     \renewcommand{\headrulewidth}{0pt}
	\fancyfoot[R]{\footnotesize Page \thepage \ of \pageref{LastPage}}
}
\newcommand{\pmat}[4]{\begin{pmatrix} #1 & #2 \\ #3 & #4 \end{pmatrix}}
\newcommand{\A}{\mathbb{A}}
\newcommand{\B}{\mathbb{B}}
\newcommand{\fin}{``\in"}
\newcommand{\mk}[1]{\mathfrak{#1}}
\newcommand{\g}{\mk{g}}
\newcommand{\h}{\mk{h}}
\newcommand{\tphi}{\tilde{\phi}}
\DeclareMathOperator{\Perm}{Perm}
\DeclareMathOperator{\pdim}{pdim}
\DeclareMathOperator{\gldim}{gldim}
\DeclareMathOperator{\lgldim}{lgldim}
\DeclareMathOperator{\rgldim}{rgldim}
\DeclareMathOperator{\idim}{idim}
\DeclareMathOperator{\SU}{SU}
\DeclareMathOperator{\SO}{SO}
\DeclareMathOperator{\Ad}{Ad}
\DeclareMathOperator{\ad}{ad}
\DeclareMathOperator{\gr}{gr}
\newcommand{\Rmod}{R-\text{mod}}
\newcommand{\RMod}{R-\text{Mod}}
\newcommand{\onto}{\twoheadrightarrow}
\newcommand{\into}{\hookrightarrow}
\newcommand{\barf}{\bar{f}}
\newcommand{\dd}[2]{\frac{d#1}{d#2}}
\newcommand{\pp}[2]{\frac{\partial #1}{\partial #2}}
\newcommand{\gl}{\mk{g}\mk{l}}
\renewcommand{\P}{\mathbb{P}}
\renewcommand{\E}{\mathbb{E}}
\DeclareMathOperator{\Ext}{Ext}
\DeclareMathOperator{\Rank}{Rank}
\DeclareMathOperator{\Sp}{Sp}

\newcommand{\exactlon}[5]{
		\begin{tikzcd}
			0\ar[r]&#1\ar[r,"#2"]& #3 \ar[r,"#4"]& #5 \ar[r]&0
		\end{tikzcd}
}

\title{MAT 1344}
\author{John White}
\date{Fall 2024}


\begin{document}

\section*{Lecture 2 - 3/5/24}

\underline{Let's try setting notation}

\underline{Notation:}

We denote by $[X,Y]$ the homotopy classes of maps from $X \to Y$. 

We denote by $[(X,A),(Y,B)]$ the homotopy classes of maps relative to $A$ $f:X\to Y$ such that $f(A)\subseteq B$

\defn

The \underline{$n$th homotopy group of $X$ based at $x$}, $\pi_n(X, x)$, is defined as 
\[
\pi_n(X,x) \eqdef \underbrace{[(I^n,\partial I^n), (X, x_0)]}_{\text{more useful for technical lemmas}} \cong \underbrace{[(S^n,*),(X,*)]}_{\text{more useful for intuition}}
\]

\lem

For $n \geq 2$, $\pi_n(X)$ is Abelian. 

\proof

\includegraphics[scale=.2]{1351P1}

\qed

\defn

The \underline{relative homotopy group} $\pi_n(X, A, x_0)$ is
\[
[(I^n, \partial I^n, \partial I^n\setminus I^{n-1}),(X,A,x_0)] \cong [(D^n, \partial D^n, *),(X,A,x_0)]
\]

\underline{Exercise:}

Show that for $n \geq 3$, this is an Abelian group. 

INSERT PICTURE HERE

We have a long exact sequence for homotopy groups:
\[
\begin{tikzcd}
\cdots \ar[r] & \pi_n(A) \ar[r, "i_*"] & \pi_n(X) \ar[r, "tautological"] & \ar[r, "restriction"] \pi_n(X,A) \ar[r] & \pi_{n-1}(A) \ar[r] & \cdots 
\end{tikzcd}
\]

\underline{Exercise:} Convince yourself its exact at $\pi_n(X)$. 

\defn

The \underline{Hurewicz homomorphism $H:\pi_n(X)\to H_n(X)$, $\pi_n(X,A) \to H_n(X, A)$} is defined as follows: 

Given $f:(I,\del I^n) \to (X, A)$, define a singular cycle using a homeomorphism from $I^n$ to the $n$-simplex. 

\underline{Exercise:} Check that this is a homomorphism. 

\thm

$H:\pi_n(S^n) \to H_n(S^n)$ is an isomorphism, and therefore $\pi_n(S^n) \cong \Z$. 

\proof

To show it's onto, we need a map of degree $d$. To do this, pick $d$ little balls inside the sphere, map each onto the whole sphere, and map everything outside these balls to the basepoint. No problems arise so long as we correctly consider orientation. 

Now we have to show it's injective. We do this by showing every map of degree 0 is nullhomotopic. 

Take $f:S^n\to S^n$ of degree 0. By some homotopy, we can assume it is smooth. Pick a regular value, and take the preimage of a ball around it. Its preimage is a bunch of balls with local degrees which cancel out. 

Next, take a homotopy $H:S^n\times[0,1]\to S^n$ which spreads the ball around the regular point around the entire sphere, sending everything outside it to the basepoint. Call this a homotopy from $\Id$ to $H_1$. 

Now consider $H \circ (g\times \Id_{[0,1]})$, which is a homotopy from $g$ to $H_1\circ g$. 

Now we have a map $H_1\circ g$ sending everything but these discs in the domain to the basepoint. 

Then, consulting the picture, we connect the balls using spaghettis. 

\includegraphics[scale=.2]{1351P2}
\qed

\thm[Whitehead Theorem]

Let $X, Y$ be CW complexes, $Y$ connected. Let $f:X\to Y$ such that $f_*:\pi_n(X) \to \pi_n(Y)$ is an isomorphism for all $n \geq0$. 

Then $f$ is a homotopy equivalence. 

\proof
\,

\underline{note:}
\begin{enumerate}[label=(\roman*)]

\item It is not enough for $\pi_n(X), \pi_n(Y)$ to be abstractly isomorphic, there must be a map. For example, for every $n$, $\pi_n(S^2) \cong \pi_n(\CP^2\times S^3)$, but we know from their homology groups that they're not homotopy equivalent. 

\item They need to be CW complexes. For example,if we take the map from the point to the warsaw circle, this is an iso on all $\pi_n$, but the warsaw circle is not contractible. 

\end{enumerate}

We now proceed with the proof

\underline{Case 1:} 

We begin by showing that if $\pi_n(Y) \cong 0$ for all $n \geq 0$, then $X$ is contractible. The idea is that we contract skeleton by skeleton. 

\underline{Step 0:} Homotope $Y^{(0)}$ to the basepoint (since $Y$ is connected).

Recall that CW satisfy the \underline{homotopy extension property}, so a nullhomotopy of $Y^{(k)}$ is a map on $Y\times\{0\} \cup Y^{(k)} \times [0, 1]$ which extends to $Y\times[0,1]$. 

\underline{Step $k + 1$:} We've built a map $f_k:Y\to Y$ such that $Y^{(k)}$ goes to the basepoint. So every $(k + 1)$-cell becomes a map $(D^{k + 1}, S^k) \to (Y, y_0)$. Since $\pi_{k + 1}(Y) = 0$, we can contract the $(k + 1)$-skeleton and extend this map to the rest of $Y$. 

\underline{Step $\oo$:}

Do step $0$ on $[0,\frac12]$, step $1$ on $[\frac12,\frac34]$, et cetera. The image of every point is eventually constant, so the homotopy is continuous at 1 (exercise in using the weak topology).

\underline{Case 2:} Suppose $X \into Y$ is a subcomplex (and the inclusion induces $\cong$ on all $\pi_n$). Then

$\pi_n(Y, X)\cong 0$ by the LES from above. We can do the same thing except instead of contracting into a point, we deformation retract onto $X$.

\underline{Case 3:} General case. Let $X \to Y$ be some map. We know $Y \cong M_f = \frac{Y \coprod X\times[0,1]}{(x,1)\sim f(x)}$

This reduces the general case to case 2. 

\qed

\thm[Hurewicz Theorem] 

Let $X$ be a CW complex. If $\pi_k(X) \cong 0$ for $k < n$ (and $n\geq2$), then $H_k(X)\cong 0$ for $k \leq n$, and $H:\pi_n(X) \to H_n(X)$ is an isomorphism. 

\proof

Suppose by induction that $X^{(k-1)}$ is a point, $k < n$. We find a $Y \simeq X$ such that $Y^{(k)}$ is a point. 

\underline{Step 1:} Attach $(k + 1)$-cells to fill each $k$-cell $e_i$. For each $e_i$ there is a map $f_i:D^{k + 1} \to X$ such that $f_i|_{\del}$ is the inclusion of $e_i$.

These together form a map from $S^{k+1}$ along which we can attach a $(k + 2)$-cell. We get a space $Z$ which obviously deformation retracts to $X$. 

\underline{Step 2:} Go from $Z$ to $Y$. 

Inside $Z$ we have a subcomplex $A$ which is contractible, so $Z/A\simeq Z$. So set $Y = Z/A$.

By this ``cell-trading" method, we've shown the first part of the theorem (we have shown that $X$ is homotopy equivalent to something with a trivial $k$-skeleton, so by cellular homology the homology groups vanish. 

\includegraphics[scale=.2]{1351P3}

Now we need to show that if $X^{(n-1)}$ is a point, then $\pi_n(X) \cong H_n(X)$.

Surjectivity is clear (because every homology generator is represented by a sphere). 

For injectivity, we need to show that if a certain sum of $n$-cells is homologically trivial, then it is homotopically trivial. 

If it is homologically trivial, then there's a sum of $(n + 1)$-cells whose $\del$ is $\sum \alpha_ie_i$. 

UNTIL NEXT TIME. 




\end{document}




